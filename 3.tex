\documentclass[pmath441]{subfiles}

%% ========================================================
%% document

\begin{document}

    \section{Prime Factorization}
    
    Let $K$ be a number field and let $R=\mO_K$. Let's recall some important properties of $R$ as a ring.
    \begin{enumerate}
        \item Every nonzero prime ideal of $R$ is maximal.
        \item If $I$ is a nonzero ideal, then $R /I$ is finite.
        \item $R$ is Noetherian.
    \end{enumerate}
    
    \subsection{Some Useful Ring Theory}
    
    \begin{prop}{}
        Let $R$ be a ring.\footnotemark[1] The following are equivalent.
        \begin{enumerate}
            \item $R$ is Noetherian.
            \item Every ascending chain of ideals stabilizes.\footnotemark[2] \hfill\textit{ascending chain condition (acc)}
            \item Every nonempty collection of ideals of $R$ has a maximal (with respect to inclusion) element.
        \end{enumerate}
        
        \noindent
        \begin{minipage}{\textwidth}
            \footnotetext[1]{Let us recall that a ring is always commutative and unital in our course.}
            \footnotetext[2]{This is the \textit{usual} definition of Noetherian ring in commutative algebra.}
        \end{minipage}
    \end{prop}
    
    \placeqed[Proof is left as an exercise]

    \np The idea for (b) $\implies$ (a) is that, given an ascending chain of ideals, the union is also an ideal. For this ideal to be finitely generated, it must be the case that the chain stabilizes. 

    For (b) $\implies$ (c), if we assume (c) is false, then we can construct an ascending chain of ideals that does not stabilize.
    
    \begin{prop}{A Glimpse of Prime Factorization}
        Let $R$ be a Noetherian ring and let $I$ be a proper ideal of $R$. Then there exists prime ideals $P_1,\ldots,P_n$ of $R$ such that
        \begin{enumerate}
            \item $I\subseteq P_i$ for $i$;
            \item $P_1\cdots P_n \subseteq I$.
        \end{enumerate}
    \end{prop}

    \rruleline

    \np We know that prime factorization of numbers does not work well in a ring of integers. After all, a ring of integers need not be a UFD! Hence, instead of factoring numbers, we are going to \textit{factor ideals} in $\mO_K$. This will work well, and introduce us the notion of \textit{Dedekind domains}. 
    
    Note that Proposition 3.2 is bit more general than we require, that it works for any \textit{Noetherian ring}. Indeed, any ring of integer is a Noetherian ring (Corollary 1.10.4, \textit{the} result of Chapter 1).

    \begin{boxyproof}{Proof of Proposition 3.2}
        Let $X$ be the collection of proper ideals of $R$ not having the property. Assume for contradiction that $X$ is nonempty. Let $I\in X$ be an maximal \textit{element} of $X$ (we do not insist that $I$ is a maximal \textit{ideal} in $R$).

        Clearly $I$ is not prime. If not, then take $P_1=I$ and observe that $I$ has the property. Since $I$ is not prime, we may find $a,b\in R$ such that $ab\in I$ but $a,b\notin I$. By maximality of $I$, $I+\left< a \right>,I+\left< b \right>\notin X$. Note that, for any ideal $J$, $IJ\subseteq I$ (this is a property of ideal product; check this!). Moreover, $ab\in I$ and $\left< a \right>,\left< b \right>$ are principal ideals, so that $\left< a \right>\left< b \right> = \left< ab \right> \subseteq I$. Hence it follows that     
        \begin{equation*}
            \left( I+\left< a \right>  \right)\left( I+\left< b \right>  \right) \subseteq I.
        \end{equation*}
        Hence $I+\left< a \right>, I+\left< b \right>\neq R$ (since $JR=RJ=J$ for any ideal $J$). Therefore, there are prime ideals $P_1,\ldots,P_n,Q_1,\ldots,Q_m$ such that
        \begin{enumerate}
            \item $I+\left< a \right>\subseteq P_i, I+\left< b \right>\subseteq Q_j$ for $i,j$ $\implies$ $I\subseteq I+\left< a \right>\subseteq P_i$, $I\subseteq I+\left< b \right>\subseteq Q_j$ for $i,j$; and
            \item $P_1\cdots P_n\subseteq I+\left< a \right>, Q_1\cdots Q_m\subseteq I+\left< b \right>$ $\implies$ $P_1\cdots P_nQ_1\cdots Q_m\subseteq\left( I+\left< a \right>  \right)\left( I+\left< b \right>  \right)\subseteq I$.
        \end{enumerate}
        Thus $I\notin X$, which is a contradiction.
    \end{boxyproof}
    
    \begin{definition}{\textbf{Coprime} Ideals}
        Let $R$ be a ring and let $I,J\subseteq R$ be prime ideals. We say $I,J$ are \emph{coprime} if and only if $I+J=R$.
    \end{definition}

    \np A motivation for the above definition comes from the Bezout lemma.

    \begin{prop}{}
        Let $R$ be a ring and let $I,J$ be coprime ideals of $R$. Then for any $n,m\in\N$, $I^n,J^m$ are coprime.
    \end{prop}

    \begin{proof}
        Since $I,J$ are proper, so are $I^n\subseteq I, J^m\subseteq J$. Suppose for contradiction that
        \begin{equation*}
            I^n+J^m \neq R.
        \end{equation*}
        Then $I^n+J^m\subseteq M$ for some maximal ideal $M$, which means $I^n,J^m\subseteq M$. But any maximal ideal is a prime ideal, so that $M$ is a prime ideal. Recall that, 
        \begin{equation*}
            \text{\textit{given two ideals $\tilde{I},\tilde{J}$ and a prime ideal $P$ such that $\tilde{I},\tilde{J}\subseteq P$, $\tilde{I}\subseteq P$ or $\tilde{J}\subseteq P$.}}
        \end{equation*}
        In particular, $I,J\subseteq M$. This means $I+J\subseteq M\neq R$, a contradiction.
    \end{proof}

    \np Recall the following theorem from ring theory.

    \begin{theorem}{Chinese Remainder Theorem}
        Let $R$ be a ring and let $I,J$ be coprime ideals of $R$. Then $R /IJ \iso R /I\times R /J$.
    \end{theorem}

    \begin{proof}
        "\textit{When we want two algebraic objects to be isomorphic, 99.9\% of the time we want to find an isomorphism.}" - Blake

        Since we are working with quotient rings, we resort to the first isomorphism theorem. Let
        \begin{equation*}
            \begin{aligned}
                \phi:R&\to R /I\times R /J \\
                x & \mapsto \left( x+I,x+J \right).
            \end{aligned} 
        \end{equation*}
        Then
        \begin{equation*}
            \ker\left( \phi \right) = I\cap J.
        \end{equation*}
        Now observe that,
        \begin{equation*}
            IJ\subseteq I\cap J = \left( I\cap J \right)R = \left( I\cap J \right)\left( I+J \right) = \underbrace{\left( I\cap J \right)I}_{\subseteq IJ} + \underbrace{\left( I\cap J \right)J}_{\subseteq IJ} \subseteq IJ,\footnotemark[1]
        \end{equation*} 
        so that
        \begin{equation*}
            IJ \subseteq I.
        \end{equation*}
        Hence we conclude
        \begin{equation*}
            \ker\left( \phi \right) = IJ.
        \end{equation*}
        To invoke the first isomorphism theorem, we want to show that $\phi$ is surjective. Take $a\in I, b\in J$ such that $a+b = 1$ (since $I+J=R$). For $x,y\in R$
        \begin{equation*}
            \begin{aligned}
                \phi\left( ax+by \right) & = \left( \underbrace{ax}_{\in I}+by+I,ax+\underbrace{by}_{\in J}+J \right) = \left( by+I,ax+J \right) \\
                                         & = \left( b+I,a+J \right)\left( y+I,x+J \right) = \left( 1+I,1+J \right)\left( y+I,x+J \right) = \left( y+I,x+J \right).
            \end{aligned} 
        \end{equation*}
        Note that we are using $a+b=1$ but $a+I=0+I, b+J=0+J$ to obtain the second-last equality.

        Thus $\phi$ is surjective and
        \begin{equation*}
            R /IJ \iso R /I\times R /J
        \end{equation*}
        by the first isomorphism theorem.
        
        \noindent
        \begin{minipage}{\textwidth}
            \footnotetext[1]{Note that the above argument worked because of the \textit{coprimeness} of $I,J$: $R=I+J$.}
        \end{minipage}
    \end{proof}
    
    \clearpage
    
    \begin{theorem}{Generalized Chinese Remainder Theorem}
        Let $R$ be a ring and let $I_1,\ldots,I_n$ be \textit{pairwise} coprime ideals. Then $R /I_1\cdots I_n \iso R /I_1 \times \cdots \times R /I_n$.
    \end{theorem}

    \rruleline

    \begin{prop}{}
        Let $R$ be a finite ring. Then 
        \begin{equation*}
            R \iso R /P_1^{n_1} \times \cdots \times R /P_m^{n_m}
        \end{equation*}
        for some distinct prime ideals $P_1,\ldots,P_m$ and $n_1,\ldots,n_m\in\N$.
    \end{prop}

    \rruleline

    \np In case $R$ is an integral domain, we can simply take $P_1 = \left\lbrace 0 \right\rbrace$ and \textit{call it a day!} In fact, the key idea for the general case is to identify $R$ with $R / \left\lbrace 0 \right\rbrace$.

    \begin{boxyproof}{Proof of Proposition 3.6}
        Note that
        \begin{equation*}
            R\text{ is finite} \implies R\text{ is Noetherian}.\footnotemark[1]
        \end{equation*}
        So we may find prime ideals $Q_1,\ldots,Q_k\subseteq R$ such that $Q_1\cdots Q_k=\left\lbrace 0 \right\rbrace$. \textit{Graping} the $Q_i$'s we obtain distinct prime ideals $P_1,\ldots,P_m$ such that
        \begin{equation*}
            P_1^{n_1}\cdots P_m^{n_m} = \left\lbrace 0 \right\rbrace.
        \end{equation*}
        For each $P_i$,
        \begin{equation*}
            R\text{ is finite and }P_i\text{ is prime}\implies R /P_i\text{ is finite integral domain} \implies R /P_i\text{ is a field}.
        \end{equation*}
        Hence each $P_i$ is maximal, which imply
        \begin{equation*}
            P_i+P_j = R, \hspace{2cm}\forall i\neq j.
        \end{equation*}
        It follows $P_i^{n_i}+P_j^{n_j}=R$. Hence $P_1,\ldots,P_m$ are pairwise coprime ideals, so by the generalized Chinese remainder theorem,
        \begin{equation*}
            R \iso R /\left\lbrace 0 \right\rbrace = R /P_1^{n_1}\cdots P_m^{n_m} \iso R /P_1^{n_1}\times\cdots R /P_m^{n_m}.
        \end{equation*}

        \noindent
        \begin{minipage}{\textwidth}
            \footnotetext[1]{"\textit{Good luck in finding an infinite ascending chain in a finite ring!}" - Blake}
        \end{minipage}
    \end{boxyproof}

    \subsection{Prime Ideals of a Ring of Integers}

    \begin{boxyrecall}{}
    Once again, let $K$ be a number field of degree $n$ and let $R = \mO_K$. 
        \begin{enumerate}
            \item $R$ is Noetherian. 
            \item $R /I$ is finite for any nonzero proper ideal $I$.
            \item Every ideal $\overline{J}$ of $R /I$ is of the form $\overline{J} = J /I$, where $J\subseteq R$ is an ideal such that $I\subseteq J$; moreover, $\overline{J}$ is prime if and only if $J$ is prime.\footnote{In fact, this is true for any ring!} \hfill\textit{correspondence theorem}
            \item $R /I \iso \left( R /I \right) / \left( P_1^{n_1} / I \right) \times \cdots \times \left( R /I \right) / \left( P_m^{n_m} / I \right) \iso R /P^{n_1}\times\cdots\times R /P_m^{n_m}$, where each $P_i\subseteq R$ is prime with $I\subseteq P_i$.
        \end{enumerate}
    \end{boxyrecall}
    Here are some bing ideas for this section:
    \begin{enumerate}
        \item To understand $I$, we study the prime ideals $P$ containing $I$. Turns out, for a prime ideal $P$,
            \begin{equation*}
                I\subseteq P \iff P\text{ is a \textit{prime factor} of $I$}.
            \end{equation*}
        \item The prime ideals of $R /I$ are $P /I$, where $P$ is a prime ideal containing $I$.
        \item Say $P$ is a prime ideal containing $I$. Then $\left| R /P \right| = p^m$ for some prime $p$ and $m\in\N$. Now,
            \begin{equation*}
                p^m + P = p^m\left( 1+P \right) = 0+P
            \end{equation*}
            by Lagrange's theorem, which imply that $p^m\in P$. Since $P$ is a prime ideal, it follows $p\in P$. Hence we have
            \begin{equation*}
                \left< p \right> \subseteq P. 
            \end{equation*}
            That is, any prime ideal containing $I$ also contains a principal ideal generated by \textit{an old-school prime number}. Because of this, we first search for ideals of the form $\left< p \right>$ to find candidates for prime factorization of $I$. 
    \end{enumerate}
    
    \begin{example}{}
        Let $K=\Q\left( \sqrt{2} \right), R = \mO_K = \Z\left[ \sqrt{2} \right]$. Find all prime ideals $P$ of $R$ containing $\left< 5 \right>$. 
    \end{example}

    \begin{answer}
        Observe that
        \begin{equation*}
            R /\left< 5 \right> = \Z\left[ \sqrt{2} \right] /\left< 5 \right> \iso \Z\left[ x \right] / \left< x^{2}-2,5 \right>  = \Z\left[ x \right] / \left< 5,x^{2}-2 \right> \iso \Z_5\left[ x \right] / \left< x^{2}-2 \right> .
        \end{equation*}
        But $x^{2}-2$ is irreducible over $\Z_5$, which means $\left< x^{2}-2 \right>$ is a maximal ideal of $\Z_5\left[ x \right]$. Therefore, $\Z_5\left[ x \right] /\left< x^{2}-2 \right>$ is a field, and so is $R /\left< 5 \right>$. Hence $\left< 5 \right>$ is a maximal ideal of $R$, which means the only prime ideal containing $\left< 5 \right>$ is $\left< 5 \right>$ itself.  
    \end{answer}

    \begin{example}{}
        Let $K=\Q\left( \sqrt{2} \right), R = \mO_K = \Z\left[ \sqrt{2} \right]$. Find all prime ideals $P$ of $R$ containing $\left< 7 \right>$. 
    \end{example}

    \begin{answer}
        Observe
        \begin{equation*}
            R /\left< 7 \right> = \Z\left[ x \right] / \left< x^{2}-2,7 \right> = \Z_7\left[ x \right] / \left< x^{2}-2 \right>.   
        \end{equation*}
        But $x^{2}-2$ is reducible over $\Z_7$, namely
        \begin{equation*}
            x^{2}-2 = \left( x+3 \right)\left( x+4 \right).
        \end{equation*}
        It follows $\left< x^{2}-2 \right> = \left< x+3 \right>\left< x+4 \right>$, and the two ideals $\left< x+3 \right>,\left< x+4 \right>$ are coprime. It follows by the Chinese remainder theorem that
        \begin{equation}
            \Z_7\left[ x \right] / \left< x^{2}-2 \right> \iso \Z_7\left[ x \right] / \left< x+3 \right> \times \Z_7\left[ x \right] / \left< x+4 \right> \iso \Z_7\times\Z_7,  
        \end{equation}
        where the last isomorphism is due to the first isomorphism theorem (or, we can intuitively think that we can replace $x$ by $-3,-4$ and retain every element of $\Z_7$ from $\Z_7\left[ x \right]$, respectively).

        The prime ideals of $\Z_7\times\Z_7$ are
        \begin{equation*}
            P_1 = \left< \left( 1,0 \right) \right>, P_2 = \left< \left( 0,1 \right) \right>. 
        \end{equation*}
        Now, given an isomorphism $\phi$, $\phi\left( \left< a \right>  \right) = \left< \phi\left( a \right) \right>$. Hence we have to \textit{undo} isomorphisms in [3.1] with elements $\left( 1,0 \right),\left( 0,1 \right)$ to figure out the prime ideals containing $\left< 7 \right>$:
        \begin{flalign*}
            && \left( 1,0 \right) & \mapsto \left( 1+\left< x+3 \right>,0+\left< x+4 \right>   \right) &&\\
            &&                   & \mapsto x+4+\left< x^{2}-2 \right> && \text{since $x+4$ is $1$ modulo $x+3$ and $0$ modulo $x+4$} \\
            &&                   & \mapsto x+4+\left< x^{2}-2,7 \right> && \\
            &&                   & \mapsto \sqrt{2}+4+\left< 7 \right>
        \end{flalign*}
        and
        \begin{equation*}
            \left( 0,1 \right) \mapsto \left( 0+\left< x+3 \right>,1+\left< x+4 \right>   \right)  
                               \mapsto \left( -x-3 \right) + \left< x^{2}-2 \right> 
                               \mapsto -x-3 + \left< x^{2},7 \right> 
                               \mapsto -\sqrt{2}-3 + \left< 7 \right> .
        \end{equation*}
        Therefore, the prime ideals in $R$ containing $7$ are $Q_1=\left< \sqrt{2}+4,7 \right>, Q_2=\left< -\sqrt{2}-3,7 \right>$. Note that we are including $7$ in each ideal in addition to $\sqrt{2}+4,-\sqrt{2}-3$, respectively, in order to mod out by $\left< 7 \right>$.  In fact, $\left< -\sqrt{2}-3,7 \right> = \left< \sqrt{2}+3,7 \right>$ and $\left( \sqrt{2}+3 \right)\left( \sqrt{2}-3 \right) = -7$, so that $Q_2 =  \left< \sqrt{2}+3 \right>$. 
        
        Note that $\left( \sqrt{2}+3 \right)\left( \sqrt{4} \right) = 14+7\sqrt{2}\in\left< 7 \right>$, so that $Q_1Q_2 = \left< 7 \right>$. That is, we factored $\left< 7 \right>$ into prime ideals!  
    \end{answer}

    \clearpage

    \begin{example}{}
        Let $K = \Q\left( \sqrt{2} \right), R=\mO_K=\left[ \sqrt{2} \right]$. Find all prime ideals $P$ of $R$ containing $\left< 2 \right>$.  
    \end{example}
    
    \begin{answer}
        We have
        \begin{equation*}
            R /\left< 2 \right> \iso \Z\left[ x \right] / \left< x^{2}-2,2 \right> \iso \Z_2\left[ x \right] / \left< x^{2}-2 \right> = \Z_2\left[ x \right] / \left< x^{2} \right>,
        \end{equation*}
        since $x^{2}-2 \equiv x^{2} \mod 2$. Since $\Z_2\left[ x \right] / \left< x^{2} \right>$ is very small,
        \begin{equation*}
            \Z_2\left[ x \right] / \left< x^{2} \right> = \left\lbrace 0+\left< x^{2} \right>,1+\left< x^{2} \right>,x+\left< x^{2} \right>,x+1+\left< x^{2} \right> \right\rbrace ,
        \end{equation*}
        given an ideal of $\Z_2\left[ x \right] / \left< x^{2} \right>$, we can explicitly write down the elements. 
        
        Let $P$ be a prime ideal of $\Z_2\left[ x \right] /\left< x^{2} \right>$. Since $P$ is an ideal, $0+\left< x^{2} \right>\in P$. Since $P$ is prime so proper, $1+\left< x^{2} \right>\notin P$. Also,
        \begin{equation*}
            \left( x+1+\left< x^{2} \right>  \right)^{2} = \left( x^{2}+2x+1+\left< x^{2} \right>  \right) = 1+\left< x^{2} \right>\notin P,
        \end{equation*}
        so that $x+1+\left< x^{2} \right>\notin P$, since $P$ is prime. Hence $P = \left< 0+\left< x^{2} \right>  \right>$ or $P = \left< x+\left< x^{2} \right>  \right>$. But $\Z_2\left[ x \right] / \left< x^{2} \right>$ is not an integral domain, since $x+\left< x^{2} \right> $ is a zero divisor. It follows that
        \begin{equation*}
            P = \left< x+\left< x^{2} \right>  \right>. 
        \end{equation*}

        Retracing the isomorphisms,
        \begin{equation*}
            x+\left< x^{2} \right> \mapsto x+\left< x^{2}-2,2 \right> \mapsto \sqrt{2} + \left< 2 \right>.   
        \end{equation*}
        Hence the only prime $Q\subseteq R$ with $2\in Q$ is
        \begin{equation*}
            Q = \left< \sqrt{2},2 \right> = \left< \sqrt{2} \right>.  
        \end{equation*}
        Note that
        \begin{equation*}
            \left< 2 \right> = \left< \sqrt{2} \right>^{2}.  
        \end{equation*}
        Hence we have a prime factorization of $\left< 2 \right>$ with \textit{multiplicity}. 
    \end{answer}
    
    \begin{prop}{}
        Let $K$ be a number field with $\left[ K:\Q \right]$ with $K=\Q\left( \alpha \right)$ such that $\mO_K=\Z\left[ \alpha \right]$.\footnotemark[1] Let $m\in\Z\left[ x \right]$ be the minimal polynomial for $\alpha$. If $p$ is prime and
        \begin{equation*}
            m = q_1^{n_1}\cdots q_k^{n_k}\in\Z_p\left[ x \right]\footnotemark[2]
        \end{equation*}
        for some distinct irreducible $q_1,\ldots,q_k\in\Z_p\left[ x \right]$, then
        \begin{enumerate}
            \item the prime ideals $P\subseteq\mO_K$ such that $p\in P$ are exactly of the form $P=\left< q_i\left( \alpha \right),p \right>$; and
            \item $\left< p \right> = \left< q_1\left( \alpha \right),p \right>^{n_1} \cdots \left< q_k\left( \alpha \right),p \right>^{n_k}$ in $\mO_K$.
        \end{enumerate}
        
        \noindent
        \begin{minipage}{\textwidth}
            \footnotetext[1]{Observe that $K=\Q\left( \alpha \right)$ does not add any assumption, since every number field is of the form due to the primitive element theorem.}
            \footnotetext[2]{To be more precise, we are referring to the polynomial $\overline{m}\in\Z_p\left[ x \right]$ we obtain by replacing every coefficient of $m$ by its equivalence class in $\Z_p$.}
        \end{minipage}
    \end{prop}

    \placeqed[We shall treat this as a fact for now!]
    
    \begin{example}{}
        Consider $\alpha\in\CC$ with $\alpha^{2}+\alpha+1 = 0$. Then $m = x^{2}+x+1$ is the minimal polynomial for $\alpha$ over $\Q$ and $\mO_{\Q\left( \alpha \right)} = \Z\left[ \alpha \right]$.

        Over $\Z_3$,
        \begin{equation*}
            m = \left( x+2 \right)\left( x+2 \right),
        \end{equation*}
        so that
        \begin{equation*}
            \left< 3 \right> = \left< \alpha+2,3 \right>^{2}.  
        \end{equation*}

        On the other hand, over $\Z_2$, $m$ is irreducible, so that
        \begin{equation*}
            \left< 2 \right> = \left< \alpha^{2}+\alpha+1,2 \right>. 
        \end{equation*}
    \end{example}

    \rruleline
    
    \subsection{Dedekind Domains}

    Dedekind domains are the rings where the ideal prime factorization happens.

    \begin{boxyrecall}{}
        Let $R,S$ be integral domains, $R\subseteq S$.
        \begin{enumerate}
            \item Let $\alpha\in S$. Then
                \begin{equation*}
                    \alpha\text{ is integral over $R$} \iff \text{there is monic $f\in R\left[ x \right]$ such that $f\left( \alpha \right)=0$} \iff \text{$R\left[ \alpha \right]$ is a finitely generated $R$-module}.
                \end{equation*}

            \item We say $S$ is integral over $R$ if and only if every element of $S$ is integral over $R$.
        \end{enumerate}
    \end{boxyrecall}
    
    \begin{definition}{\textbf{Integral Closure}}
        Let $R,S$ be integral domains, $R\subseteq S$.
        \begin{enumerate}
            \item The \emph{integral closure} of $R$ in $S$ is
                \begin{equation*}
                    \left\lbrace \alpha\in S: \alpha\text{ integral over $R$} \right\rbrace.
                \end{equation*}
            \item $R$ is \emph{integrally closed} if and only if the integral closure of $R$ in its field of fractions is $R$.
        \end{enumerate}
    \end{definition}

    \begin{example}{}
        $\Z$ is integrally closed.
    \end{example}

    \rruleline

    \np Let $K$ be a number field and let $R=\mO_K$. Let $F$ be the field of fractions of $R$. Given $\alpha\in K$, since $\alpha$ is an algebraic number, there is a polynomial $f\in\Z\left[ x \right]$ annihilating $\alpha$. Taking the leading coefficient $N\in\Z$ of $f$, it follows $N\alpha\in R$. Hence $\alpha\in F$, which imply that $K\subseteq F$.

    But $F$ is the smallest field containing $R$, so that $K=F$.

    \begin{prop}{}
        Let $K$ be a number field. Then $\mO_K$ is algebraically closed.
    \end{prop}
    
    \begin{proof}
        Let
        \begin{equation*}
            f = x^n+a_{n-1}x^{n-1}+\cdots+a_0\in\mO_K\left[ x \right]
        \end{equation*}
        and supose $f\left( \alpha \right) = 0$ for some $\alpha\in K$. Then each $a_i$ is an algebraic integer, so $\Z\left[ a_i \right]$ is a finitely generated $\Z$-module. Hence $\Z\left[ a_{n-1},\ldots,a_0 \right]$ is also finitely generated. Also,
        \begin{equation*}
            \alpha^n = -\sum^{n-1}_{j=0} a_j\alpha^j.
        \end{equation*}
        It follows that $\Z\left[ \alpha,a_{n-1},\ldots,a_0 \right]$ is finitely generated. Since $\Z$ is Noetherian and $\Z\left[ \alpha \right]\subseteq\Z\left[ \alpha,a_{n-1},\ldots,a_0 \right]$, $\Z\left[ \alpha \right]$ is finitely generated. Thus $\alpha$ is an algebraic integer, as required.
    \end{proof}
    
    \begin{definition}{\textbf{Dedekind Domain}}
        Let $R$ be an integral domain. We say $R$ is a \emph{Dedekind domain} if
        \begin{enumerate}
            \item $R$ is Noetherian;
            \item $R$ is integrally closed; and
            \item every nonzero prime ideal of $R$ is maximal.
        \end{enumerate}
    \end{definition}
    
    \begin{example}{}
        Let $K$ be a number field. Then $\mO_K$ is a Dedekind domain.
    \end{example}

    \rruleline

    \clearpage
    
    \np Here is a question for the section:
    \begin{equation*}
        \text{\textit{why is Def'n 3.3 the right definition for prime factorization?}}
    \end{equation*}
    It turns out (\textit{spoiler alert})\ldots
    \begin{enumerate}
        \item $\implies$ existence of prime factorization;
        \item $\implies$ prime ideals cannot be factored further; and
        \item $\implies$ uniqueness of prime factorization.
    \end{enumerate}
    Let us first explore the third implication. The following lemma will be \textit{the contradiction getter}, according to Blake.

    \begin{lemma}{}
        Let $R$ be a Dedekind domain and let $I$ be a proper nontrivial ideal of $R$. Let $F$ be the field of fractions of $R$. Then there is $\lambda\in F\setminus R$ such that $\lambda I\subseteq R$.
    \end{lemma}

    \begin{proof}
        Let $a\in I$ be nonzero. Since $R$ is Noetherian, we may find nonzero prime ideals $P_1,\ldots,P_r$ such that $P_1\cdots P_r\subseteq\left< a \right>$ by Proposition 3.2. Moreover, assume $r$ is minimal (i.e. there does not exist fewer prime ideals $Q_1,\ldots,Q_k$ such that $Q_1\cdots Q_k\subseteq\left< a \right>$). Let $M$ be a maximal ideal containing $I$.

        Since $P_1\cdots P_r\subseteq\left< a \right>\subseteq I\subseteq M$ and $M$ is prime, some $P_i$ is contained in $M$. Without loss of generality, suppose $P_1\subseteq M$. Since $P_1$ is a nonzero prime ideal of a Dedekind domain, it is maximal. Hence $P_1 = M$.

        \begin{case}
            \textit{$r=1$.}

            In this case,
            \begin{equation*}
                P_1\subseteq \left< a \right> \subseteq I \subseteq M = P_1, 
            \end{equation*}
            so that $I = P_1$ is a prime ideal. Take $\lambda = a^{-1}$, so that
            \begin{equation*}
                \lambda\left< a \right> = a^{-1}\left< a \right> = R \subseteq R.  
            \end{equation*}
            A quick note: $a^{-1}\notin R$, since if $a^{-1}\in R$, then $a$ is a unit in $R$, so that the principal ideal $\left< a \right>$ \textit{blows up to} $R$, contradicting the fact that $\left< a \right> \subseteq I\neq R$.  
        \end{case}

        \begin{case}
            \textit{$r>1$.}

            By minimality of $r$, $P_2\cdots P_r\nsubseteq\left< a \right>$, so choose
            \begin{equation*}
                b \in P_2\cdots P_r \setminus \left< a \right>. 
            \end{equation*}
            Note that $bP_1\subseteq\left< a \right>$, since, given any $c\in P_1$, $bc \in \left( P_2\cdots P_r \right)P_1 = P_1\cdots P_r\subseteq\left< a \right>$. Then
            \begin{equation}
                bI \subseteq bM = bP_1 \subseteq \left< a \right>. 
            \end{equation}
            Since $b\notin\left< a \right>$, $\lambda = \frac{b}{a}\notin R$. By [3.2], given any $x\in I$, $bx = ar$ for some $r\in R$, so that
            \begin{equation*}
                \lambda x = \frac{b}{a}x = \frac{ar}{a} = r\in R.
            \end{equation*}
        \end{case}
    \end{proof}
    
    \begin{prop}{Invertibility of the Ideals of a Dedekind Domain}
        Let $R$ be a Dedekind domain and let $I$ be an ideal of $R$. Then there exists a nonzero ideal $J\subseteq R$ such that $IJ$ is principal.
    \end{prop}
    
    \begin{proof}
        The case where $I = \left\lbrace 0 \right\rbrace$ or $I=R$ is trivial. Hence suppose $I$ is a nontrivial proper ideal.

        Let $a\in I$ be nonzero. Consider
        \begin{equation*}
            J = \left\lbrace x\in R: xI\subseteq\left< a \right>  \right\rbrace,
        \end{equation*}
        which is a nonzero ideal of $R$ (check this!). Note $IJ\subseteq\left< a \right>$ by definition. 

        Let
        \begin{equation*}
            A = \frac{1}{a}IJ.
        \end{equation*}
        Since $IJ\subseteq\left< a \right>$, it follows $A\subseteq R$. 

        Suppose for contradiction $A\neq R$. Observe that $A$ is a nonzero ideal of $R$ (again, check this!). From Lemma 3.9, \textit{the contradiction getter}, there is $\lambda\in F\setminus R$ such that $\lambda A\subseteq R$. Here $F$ is the field of fractions of $R$. We note two things.
        \begin{enumerate}
            \item \textit{Stupidly}, $J = \frac{1}{a}aJ$. Since $a\in I$ and $A=\frac{1}{a}IJ$, this means $J\subseteq A$, so that
                \begin{equation*}
                    \lambda J\subseteq\lambda A\subseteq R.
                \end{equation*}
            \item Observe that $\lambda A = \frac{\lambda}{a}IJ\subseteq R$. This means $\lambda IJ \subseteq aR = \left< a \right>$. 
        \end{enumerate}
        But by the definition of $J$,
        \begin{equation*}
            J = \left\lbrace x\in R: xI\subseteq\left< a \right>  \right\rbrace,
        \end{equation*}
        it follows $\lambda J\subseteq J$. Say $J$ is generated by $\alpha_1,\ldots,\alpha_m$. Then we may find $B\in R^{m\times m}$ such that
        \begin{equation*}
            \begin{bmatrix} \lambda\alpha_1 \\ \vdots \\ \lambda\alpha_m \end{bmatrix} = B \begin{bmatrix} \alpha_1 \\ \vdots \\ \alpha_m \end{bmatrix}.
        \end{equation*}
        That is, every $\lambda\alpha_j$ can be written as a $R$-linear combination of $\alpha_1,\ldots,\alpha_m$. This means
        \begin{equation*}
            \left( \lambda I-B \right) \begin{bmatrix} \alpha_1 \\ \vdots \\ \alpha_m \end{bmatrix} = 0,
        \end{equation*}
        where at least one of $\alpha_j$ is nonzero as $J = \left< \alpha_1,\ldots,\alpha_m \right>$. Hence
        \begin{equation*}
            \det\left( \lambda I-B \right) = 0.
        \end{equation*}
        This means $\lambda$ is a root of a monic polynomial over $R$, which contradicts the fact that $R$ is integrally closed and $\lambda\notin R$.

        Thus $A=R$, so that
        \begin{equation*}
            IJ = aR = \left< a \right>,
        \end{equation*}
        as required.
    \end{proof}
    
    \begin{cor}{}
        Let $R$ be a Dedekind domain and let
        \begin{equation*}
            X = \left\lbrace I\subseteq R:\text{$I$ is a nonzero ideal of $R$} \right\rbrace.
        \end{equation*}
        Define an equivalence relation $\sim$ on $X$ by
        \begin{equation*}
            I\sim J\iff \exists \alpha,\beta\in R\setminus \left\lbrace 0 \right\rbrace \left[ \alpha I = \beta J \right].
        \end{equation*}
        Then
        \begin{equation*}
            \mG = \left\lbrace \left[ I \right]_{\sim}: I\in X \right\rbrace
        \end{equation*}
        is a group with multiplication
        \begin{equation*}
            \left[ I \right]\left[ J \right] = \left[ IJ \right].
        \end{equation*}
    \end{cor}	

    \begin{proof}
        This follows from Proposition 3.10 and Assignment 2.
    \end{proof}

    \begin{definition}{\textbf{Ideal Class Group} of a Dedekind Domain}
        Consider the setting of Corollary 3.10.1. We call $\mG$ the \emph{ideal class group} of $R$.
    \end{definition}

    \clearpage
    
    \begin{prop}{Cancellation of Ideals of Dedekind Domains}
        Let $R$ be a Dedekind domain and let $A,B,C\subseteq R$ be nontrivial ideals. Then
        \begin{equation*}
            AB = AC \implies B=C.
        \end{equation*}
    \end{prop}

    \vspace{-\preskip}
    
    \begin{proof}
        Let $J$ be a nontrivial ideal of $R$ such that
        \begin{equation*}
            JA = \left< a \right> 
        \end{equation*}
        for some nonzero $a\in A$. Then
        \begin{equation*}
            AB = AC \implies JAB = JAC \implies \left< a \right>B = \left< a \right>C \implies aB = aC \implies B=C,  
        \end{equation*}
        where the last implication uses the fact that $R$ is an integral domain.
    \end{proof}
    
    \begin{definition}{\textbf{Ideal Divisibility}}
        Let $R$ be a ring and let $AB$ be ideals of $R$. We say $A$ \emph{divides} $B$, denoted as $A|B$, if and only if there is an ideal $C$ of $R$ such that $B= AC$.
    \end{definition}
    
    \begin{prop}{Characterization of Ideal Divisibility for Dedekind Domains}
        Let $R$ be a Dedekind domain and let $A,B$ be ideals of $R$. Then
        \begin{equation*}
            A|B \iff B\subseteq A.
        \end{equation*}
    \end{prop}

    \vspace{-\preskip}

    \begin{proof}
        The case involving $\left\lbrace 0 \right\rbrace$ or $R$ is trivial. Hence assume $A,B\neq\left\lbrace 0 \right\rbrace,R$.
        
        ($\implies$) Clearly $B=AC\subseteq A$.

        ($\impliedby$) Suppose $B\subseteq A$. Let $J$ be a nonzero ideal such that $JA = \left< a \right>$ for some $a\in A$. Then $JB \subseteq \left< a \right>$, which means
        \begin{equation*}
            C = \frac{1}{a}JB
        \end{equation*}
        is an ideal of $R$ (again, we can \textit{multiply} by $\frac{1}{a}$ since $JB\subseteq\left< a \right>$). This means
        \begin{equation*}
            JAC = \left< a \right> \frac{1}{a}JB = JB.  
        \end{equation*}
        Using cancellation (Proposition 3.11), we obtain
        \begin{equation*}
            AC = B.
        \end{equation*}
        That is, $A|B$, as required.
    \end{proof}
    
    \np Proposition 3.12 is \textit{nice}, since checking containment is easier than checking divisibility.
    
    \begin{theorem}{Prime Factorization of Ideals of a Dedekind Domain}
        Let $R$ be a Dedekind domain and let $I$ be a proper nontrivial\footnotemark[1] ideal of $R$. Then $I$ can be uniquely\footnotemark[2] written as a product of prime ideals.
        
        \noindent
        \begin{minipage}{\textwidth}
            \footnotetext[1]{"\textit{With $R$ we can never get existence and with $\left\lbrace 0 \right\rbrace$ we can never get uniqueness, so we rule those cases out.}" - Blake}
            \footnotetext[2]{Unique up to reordering.}
        \end{minipage}
    \end{theorem}

    \begin{proof}[Proof of Existence]\qedplacedtrue
        Let $X$ be the set of proper nontrivial ideals of $R$ which cannot be written as a product of prime ideals. For contradiction, $X\neq\emptyset$. Let $I\in X$ be an maximal element of $X$. We know $I$ is not prime, so is not maximal, since $R$ is a Dedekind domain. Let $P$ be a maximal ideal containing $I$. Since $P$ is prime, $I\neq P$. Hence there is a proper ideal $J$ such that $I = PJ$. Then
        \begin{equation*}
            I = PJ \subseteq J.
        \end{equation*}
        If $I = J$, then observe that
        \begin{equation*}
            RJ = RI = I = PJ,
        \end{equation*}
        so by cancelling $J$, we obtain $R=P$, which is a contradiction. Hence $I\neq J$, so that $J\notin X$. This means $J$ is a product of prime ideals, so that $I = PJ$ is also a product of prime ideals, which is a contradiction.

        Thus we conclude $X=\emptyset$, which means every proper nontrivial ideal of $R$ can be written as a product of prime ideals.
    \end{proof}

    \begin{proof}[Proof of Uniqueness]
        Suppose we have two factorizations of a proper nontrivial ideal $I$,
        \begin{equation*}
            I = P_1\cdots P_n = Q_1\cdots Q_m,
        \end{equation*}
        where $P_1,\ldots,P_n,Q_1,\ldots,Q_m$ are prime. This means
        \begin{equation*}
            Q_1\cdots Q_m \subseteq P_1.
        \end{equation*}
        Since $P_1$ is prime, it follows one of $Q_j$'s is contained in $P_1$. Without loss of generality, assume $Q_1\subseteq P_1$. But $Q_1$ is also prime and $R$ is a Dedekind domain, so that $Q_1$ is maximal. This means $P_1 = Q_1$. So by cancellation,
        \begin{equation*}
            P_2\cdots P_n = Q_2\cdots Q_m.
        \end{equation*}
        By induction, we obtain uniqueness.
    \end{proof}

    \np Now that we know prime factorization exists and is unique, our next question is
    \begin{equation*}
        \text{\textit{how do we actually factor an ideal?}}
    \end{equation*}
    This question will be answered in the following two sections.

    \subsection{Ideal Norm}
    
    \begin{definition}{\textbf{Norm} of an Ideal}
        Let $K$ be a number ring and let $R=\mO_K$. If $I$ is a nontrivial ideal of $R$, then we define the \emph{norm} of $I$ as
        \begin{equation*}
            N\left( I \right) = \left| R /I \right|.
        \end{equation*}
    \end{definition}
    
    \np Let's see where definition can be handy. \textit{Assume} that the norm is multiplicative:
    \begin{equation*}
        N\left( IJ \right) = N\left( I \right)N\left( J \right).
    \end{equation*}
    Let $I$ be a nontrivial proper ideal of $R$ and let
    \begin{equation*}
        n = N\left( I \right) = \left| R /I \right|.
    \end{equation*}
    We know that $I$ can be factored into product of prime ideals
    \begin{equation*}
        I = P_1^{n_1}\cdots P_k^{n_k}.
    \end{equation*}
    This means
    \begin{equation}
        N\left( I \right) = N\left( P_1 \right)^{n_1}\cdots N\left( P_k \right)^{n_k}.
    \end{equation}
    Recall that
    \begin{equation*}
        N\left( P_i \right) = \left| R /P_i \right| = p_i^{m_i}
    \end{equation*}
    where $p_i\in P_i$ is prime and $m_i\in\N$. Consequently,
    \begin{equation*}
        n = p_1^{n_1m_1}\cdots p_k^{n_km_k},
    \end{equation*}
    implying that
    \begin{equation*}
        p\in\N\text{ is prime with }p|n \implies p = p_i\text{ for some $i$}.
    \end{equation*}
    But
    \begin{equation*}
        p = p_i\in P_i \implies \left< p \right>\subseteq P_i \implies P_i|\left< p \right>.  
    \end{equation*}
    Hence \textit{if} we can factor each $\left< p_i \right>$, then we can find the candidates for $P_i$'s and hence factor $I$. Also, due to [3.3], $N\left( I \right)$ helps us find $n_i$ as well.

    \np Therefore, here are the goals in order for the above story to work out.
    \begin{formula}{Goals}
        \begin{enumerate}
            \item Prove that ideal norm is multiplicative.
            \item Show $\left< p \right>$ is easily factored for \textit{almost all}\footnotemark[1] prime $p\in\N$. 
        \end{enumerate}
        
        \noindent
        \begin{minipage}{\textwidth}
            \footnotetext[1]{What does \textit{almost all} mean? We shall see this later.}
        \end{minipage}
    \end{formula}

    \np Suppose
    \begin{equation*}
        I = P_1^{n_1}\cdots P_k^{n_k}\subseteq\mO_K
    \end{equation*}
    with $P_i\neq P_j$ for $i\neq j$. Since $P_i$'s are coprime, it follows that
    \begin{equation*}
        R /I \iso R /P_1^{n_1} \times \cdots \times R /P_k^{n_k}
    \end{equation*}
    by the Chinese remainder theorem. Hence
    \begin{equation*}
        N\left( I \right) = N\left( P_1^{n_1} \right)\cdots N\left( P_k^{n_k} \right).
    \end{equation*}
    Hence it suffices to show that
    \begin{equation}
        N\left( P^n \right) = N\left( P \right)^n \text{ for $n\in\N$, prime $P$}.
    \end{equation}

    \np Here are the tools to prove [3.4]:
    \begin{enumerate}
        \item localization;
        \item local rings; and
        \item discrete valuation ring.
    \end{enumerate}
    
    \np Suppose $R = \mO_K$ with an integral basis $\left\lbrace v_1,\ldots,v_n \right\rbrace$, and let $I$ be a nonzero ideal of $R$. Then by Assignment 2,
    \begin{equation*}
        \disc\left( w_1,\ldots,w_n \right) = \left[ R:I \right]^{2} \disc\left( v_1,\ldots,v_n \right) = N\left( I \right)^{2}\disc\left( K \right).
    \end{equation*}
    In the special case $I$ is principal,
    \begin{equation*}
        I = \left< \alpha \right> 
    \end{equation*}
    for some $\alpha\neq 0$, $\left\lbrace \alpha v_1,\ldots,\alpha v_n \right\rbrace$ is an integral basis for $I$. Then
    \begin{equation}
        \disc\left( \alpha v_1,\ldots,\alpha v_n \right) = N\left( I \right)^{2}\disc\left( K \right).
    \end{equation}
    On the other hand,
    \begin{equation}
        \disc\left( \alpha v_1,\ldots,\alpha v_n \right) = \det \left(\left[ \sigma_i\left( \alpha v_j \right) \right]^n_{i,j=1}\right)^{2} = \left( \prod^{n}_{j=1}\sigma_j\left( \alpha \right) \right)^{2} \det \left( \left[ \sigma_i\left( v_j \right) \right]^n_{i,j=1} \right)^{2} = N_{K /\Q}\left( \alpha \right)^{2} \disc\left( K \right).
    \end{equation}
    It follows from [3.5], [3.6] that
    \begin{equation*}
        N\left( I \right)^{2} = N_{K /\Q}\left( \alpha \right)^{2} \implies N\left( \left< \alpha \right>  \right) = \left| N_{K /\Q}\left( \alpha \right) \right|.
    \end{equation*}
    
    \subsection{Localization}

    Recall that the goal is to prove multiplicativity of ideal norm by showing
    \begin{equation*}
        N\left( P^n \right) = N\left( P \right)^n
    \end{equation*}
    for a prime ideal $P$.
    
    \begin{definition}{\textbf{Local Ring}}
        A \emph{local ring} is a ring $R$ which has a unique maximal ideal.
    \end{definition}

    \np How do we spot a local ring? Here is Blake's favorite way.

    \begin{prop}{}
        Let $R$ be a ring. Then
        \begin{equation*}
            R\text{ is local} \iff R\setminus R^{\times} \text{ is an ideal of $R$}.
        \end{equation*}
        In this case, $R\setminus R^{\times}$ is the unique maximal ideal of $R$.
    \end{prop}

    \begin{proof}
        Let $I = R\setminus R^{\times}$.

        ($\implies$) Suppose $R$ is local with a unique maximal ideal $M$. Since $M$ is proper, $M$ does not have any units, so that
        \begin{equation*}
            M \subseteq I.
        \end{equation*}
        But $I \subseteq \left< I \right> \subseteq M$, since $I$ does not have any units and $M$ is the unique maximal ideal. 

        ($\impliedby$) Suppose $I$ is an ideal. Then for any maximal ideal $M\subseteq R$, $M\subseteq I$, since $M$ does not have any unit. But $M$ is maximal, so $M = I$.
    \end{proof}

    \begin{example}{}
        Fields are local.
    \end{example}

    \rruleline

    \begin{example}{}
        Consider $\Z_{p^n}$ with $n>1$. Then
        \begin{equation*}
            x\notin\Z_{p^n}^{\times} \iff \gcd\left( x,p^n \right)\neq 1 \iff p|x \iff x\in\left< p \right>,  
        \end{equation*}
        so $\left< p \right>$ is the unique maximal ideal for $\Z_{p^n}$. Thus $\Z_{p^n}$ is local.
    \end{example}

    \rruleline

    \np How can we construct local integral domains? The answer is \textit{localization}.
    \begin{equation*}
        \text{\textit{"Localization is a process of making a local ring." - Blake}}
    \end{equation*}
    There are three ingredients to localization: an integral domain, the field of fractions and a prime ideal.

    \begin{definition}{\textbf{Localization}}
        Let $R$ be an integral domain, let $K$ be the field of fractions and let $P$ be a prime ideal. The \emph{localization} of $R$ at $P$ is
        \begin{equation*}
            R_P = \left\lbrace \frac{a}{b}\in K: b\notin P \right\rbrace.
        \end{equation*}
    \end{definition}

    \np There's more general version of localization, but let's leave that to commutative algebraists.

    \np Observe that we are using a \textit{lazy notation}. In fact, we can have $\frac{a}{b}\in R_P$ when $b\in P$. What we need is for there to exist $c,d\in R$ such that $\frac{a}{b} = \frac{c}{d}$ but $d\notin P$. The following example demonstrates this remark.

    \begin{example}{}
        Consider $R=\Z,P=\left< 2 \right>$. Then $\frac{4}{6}$ \textit{looks like} it should not belong to $\Z_{\left< 2 \right>}$, since $2|6$. However, $\frac{4}{6} = \frac{2}{3}$ and $2\nmid 3$, so that $\frac{4}{6}\in \Z_{\left< 2 \right>}$.
    \end{example}

    \rruleline

    \np Let $\frac{a}{b}, \frac{c}{d}\in K$ with $b,d\notin P$. Then
    \begin{equation*}
        \frac{a}{b}+\frac{c}{d} = \frac{ad+bc}{bd}\in R_P,
    \end{equation*}
    since $bd\notin P$.\footnote{\textit{"The complement of a prime ideal is multiplicatively closed."} - Blake} In a similar manner
    \begin{equation*}
        \frac{a}{b} \frac{c}{d} = \frac{ac}{bd}\in R_P.
    \end{equation*}
    Hence $R_P$ is a subring of $K$.
    
    \np Observe that
    \begin{equation}
        R_P \setminus R_P^{\times} = PR_P = \left\lbrace \sum^{n}_{j=1}a_jr_j: a_j\in P, r_j\in R_P \right\rbrace.
    \end{equation}
    Proving [3.7] is left as an exercise.

    In particular, $R_P\setminus R_P^{\times}$ is an ideal, so by Proposition 3.14 $R_P$ is local.

    \np Since we are going to refer $PR_P$ often, let's give it a notation.

    \begin{notation}{$P_P$}
        We write $P_P$ to denote $PR_P$.
    \end{notation}
    
    \np It turns out
    \begin{equation*}
        P_P = \left\lbrace \frac{a}{b}: a\in P, b\notin P \right\rbrace,
    \end{equation*}
    which is also left as an exercise.

    \np We know that
    \begin{equation*}
        R\text{ is an integral domain} \implies R_P \text{ is local}.
    \end{equation*}
    Well, Dedekind domains are \textit{much better} than integral domain, so it must be the case that
    \begin{equation*}
        R\text{ is a Dedekind domain} \implies R_P \text{ is local} + \text{???}.
    \end{equation*}

    \subsection{Discrete Valuation Rings (DVRs)}
    
    \begin{definition}{\textbf{Discrete Valuation Ring (DVR)}}
        A \emph{DVR} is an integral doamin which is
        \begin{enumerate}
            \item not a field;
            \item Noetherian;
            \item local; and
            \item such that the unique maximal ideal is principal.
        \end{enumerate}
        A generator $\pi$ for the unique maximal ideal is called a \emph{uniformizer}.
    \end{definition}
    
    \np Here's another goal:
    \begin{equation}
        \text{$R$ is Dedekind and $P$ is a nontrivial proper ideal of $R$} \implies R_P\text{ is a DVR}.
    \end{equation}
    And indeed, [3.8] is why DVR's are created.

    \np We are ruling out the case $P = \left\lbrace 0 \right\rbrace$, since $P = \left\lbrace 0 \right\rbrace$ implies $R_P$ is the field of fractions of $R$, so not a DVR.

    \clearpage
    
    \begin{lemma}{Nakayama}
        Let $R$ be a ring and let $I$ be a nonzero proper ideal of $R$. Let $M$ be a finitely generated $R$-module with $IM=M$. Then there exists $a\in R$ such that
        \begin{enumerate}
            \item $a+I=1+I$; and
            \item $aM = 0$.
        \end{enumerate}
    \end{lemma}

    \begin{proof}
        Since $M$ is finitely generated,
        \begin{equation*}
            M = Rx_1 + \cdots + Rx_n
        \end{equation*}
        for some $x_1,\ldots,x_n\in M$. But $IM = M$, so that we may write
        \begin{equation*}
            x_i = a_{i,1}x_1 + \cdots + a_{i,n}x_n
        \end{equation*}
        for some $a_{i,1},\ldots,a_{i,n}\in I$. Consider the matrix
        \begin{equation*}
            A = \left[ a_{i,j} \right]^n_{i,j=1}\in I^{n\times n}.
        \end{equation*}
        Let
        \begin{equation*}
            v = \begin{bmatrix} x_1 \\ \vdots \\ x_n \end{bmatrix}.
        \end{equation*}
        Then by construction
        \begin{equation*}
            Av=v.
        \end{equation*}
        Also, consider
        \begin{equation*}
            f = \det\left( xI_n-A \right).
        \end{equation*}
        Then by the Cayley-Hamilton theorem, we have
        \begin{equation*}
            f\left( A \right) = 0.
        \end{equation*}
        Writing $f$ explicitly,
        \begin{equation*}
            f = x^n + c_{n-1}x^{n-1} + \cdots + c_1x + c_0,
        \end{equation*}
        where each $c_i\in I$. Hence
        \begin{equation*}
            0 = f\left( A \right)v = \left( A^n+c_{n-1}A^{n-1}\cdots+c_1A+c_0I_n \right)v = v + c_{n-1}v + \cdots c_1v + c_0v = f\left( 1 \right)v.
        \end{equation*}
        So let $a = f\left( 1 \right)$.

        Now,
        \begin{equation*}
            av = 0 \implies ax_i = 0 \implies aM = a\left( Rx_1+\cdots+R_n \right) = 0.
        \end{equation*}
        Also,
        \begin{equation*}
            a = f\left( 1 \right) = 1+c_{n-1}+\cdots+c_1+c_0 \equiv 1 \mod I,
        \end{equation*}
        since each $c_i\in I$.
    \end{proof}
    
    \begin{prop}{}
        Let $R$ be a DVR and $M = \left< \pi \right>$ be the unique maximal ideal of $R$. Then every nonzero proper ideal $I$ of $R$ is of the form
        \begin{equation*}
            I = M^n
        \end{equation*}
        for some $n\in\N$.
    \end{prop}
    
    \begin{proof}
        Let $I$ be a nonzero proper ideal and let $J = \frac{1}{\pi}R$. Then
        \begin{equation*}
            JM = R.
        \end{equation*}
        Then
        \begin{equation*}
            I = IR = \underbrace{IJ}_{\clap{\text{\footnotesize$=I_1$}}} M.
        \end{equation*}
        But $I\subseteq M = \left< \pi \right>$, so that $I_1\subseteq R$. Hence
        \begin{equation*}
            I = I_1M \subseteq I_1.
        \end{equation*}

        Suppose $I=I_1$. Then $I = IM$. Also $I$ is finitely generated, since $R$ is a DVR so is Noetherian. Hence by Nakayama's lemma (with the roles of $I,M$ switched) there is $a\in R$ such that $a-1\in M$ and $aI = 0$. Since $R$ is an integral domain,
        \begin{equation*}
            a = 0 \implies -1 \in M \implies M = R.
        \end{equation*}
        This is a contradiction. Hence $I$ is a proper subset of $I_1$.

        If $I_1=R$, then $I=M$, and we are done. Suppose $I_1\neq R$. Then
        \begin{equation*}
            I_1 = I_1R = \underbrace{I_1J}_{\clap{\text{\footnotesize$=I_2$}}}M \subseteq I_2.
        \end{equation*}
        Similarly, we have that $I_1\neq I_2$ due to Nakayama's lemma. 

        If $I_2 = R$, then
        \begin{equation*}
            I_1 = M \implies I = I_1M = M^{2}
        \end{equation*}
        and we are done. If not, we continue the process to obtain an ascending chain of ideals $\left( I_{n} \right)^{\infty}_{n=1}$. Since $R$ is Noetherian, this chain stabilizes, so that we have $n\in\N$ such that $I_n = M$. This means $I$ is a power of $M$, as required.
    \end{proof}
    
    \np Observe that, by Proposition 3.16, every ideal of a DVR is principal. As a consequence, we are going to prove
    \begin{equation*}
        \text{DVR} \iff \text{local PID not a field}.
    \end{equation*}
    
    \np Let $R$ be a DVR and let $M = \left< \pi \right>$ be the unique maximal ideal of $R$. Let $x\in R$ be nonzero. We can classify $x$ into two cases.
    \begin{enumerate}
        \item $x\in R^\times$.
        \item $x\notin R^{\times}$, so that $\left< x \right>$ is a proper nonzero ideal. So by Proposition 3.16, $\left< x \right> = \left< \pi^n \right>$. This means $x,\pi$ are \textit{associates}: $x = u\pi^n$ for some unit $u\in R^{\times}$. This makes every element of $R$ look \textit{quite uniform}, which is why we call $\pi$ a \textit{uniformizer}.
    \end{enumerate}
    
    \begin{prop}{}
        Let $R$ be a Noetherian integral domain and let $P$ be a nonzero prime ideal of $R$. Then $R_P$ is Noetherian.
    \end{prop}

    \begin{proof}
        Let $I\subseteq R_P$ be an ideal and let $J = I\cap R$ be an ideal of $R$. Then $J$ is a finitely generated $R$-module, so that
        \begin{equation*}
            J = Rx_1 + \cdots + Rx_n
        \end{equation*}
        for some $x_1,\ldots,x_n\in R$. Let $x\in I$ with $x=\frac{a}{b}$ for some $a,b\in R$ with $b\notin P$. This means
        \begin{equation*}
            a = bx \in I\cap R = J.
        \end{equation*}
        Thus
        \begin{equation*}
            a = r_1x_1 + \cdots + r_nx_n \implies x = \frac{a}{b} = \frac{r_1}{b}x_1 + \cdots + \frac{r_n}{b}x_n \implies I = R_Px_1 + \cdots + R_Px_n.
        \end{equation*}
    \end{proof}
    
    \begin{theorem}{}
        Let $R$ be a Dedekind domain and let $P$ be a nonzero prime ideal. Then $R_P$ is a DVR.
    \end{theorem}

    \begin{proof}
        Since $P$ is a nonzero ideal, we know $R_P$ is not a field. Also, since $R$ is a Dedekind domain, $R$ is Noetherian, so $R_P$ is Noetherian. Moreover, $R_P$ is local as a localization of a ring. Hence it remains to show that the unique maximal ideal of $R_P$, namely $P_P$ (i.e. the ideal of non-units of $R_P$) is principal.

        Recall that there exists an ideal $I$ such that
        \begin{equation*}
            IP = \left< \alpha \right> 
        \end{equation*}
        for some $\alpha\in P$. Consider $J = \frac{1}{\alpha}I$. Note
        \begin{equation*}
            JP = \frac{1}{\alpha}IP = \frac{1}{\alpha}\left< \alpha \right> = R. 
        \end{equation*}
        Say
        \begin{equation*}
            1 = a_1b_1+\cdots+a_nb_n,
        \end{equation*}
        where each $a_i\in J, b_i\in P$. Take $i$ such that $a_ib_i\notin P$ (such $i$ exists, since otherwise $1\in P$ where $P$ is a prime ideal). This means
        \begin{equation*}
            \frac{1}{a_ib_i}\in R_P.
        \end{equation*}
        Let $x\in P_P$. Then $y=\frac{x}{a_ib_i}\in P_P$, since $x\in P_P$. Moreover
        \begin{equation*}
            x = a_ib_i y.
        \end{equation*}
        Say
        \begin{equation*}
            y = \frac{u}{v}
        \end{equation*}
        for some $u\in P, v\in R\setminus P$. Then
        \begin{equation*}
            x = b_i \frac{a_iu}{v}.
        \end{equation*}
        But $a_i\in J, u\in P$ so that $a_iu\in JP=R$. Hence $\frac{a_iu}{v}\in R_P$, which means
        \begin{equation*}
            x\in \left< \frac{b_i}{1} \right>\subseteq R_P. 
        \end{equation*}
        Since $x$ was arbitrary, it follows
        \begin{equation*}
            P_P = \left< \frac{b_i}{1} \right>, 
        \end{equation*}
        as required.
    \end{proof}

    \np Theorem 3.18 does two awesome things for us.
    \begin{enumerate}
        \item It proves the multiplicativitiy of ideal norm.
        \item It gives a powerful way to prove whether a ring of integers is of the form $\Z\left[ \alpha \right]$.
    \end{enumerate}
    
    \subsection{Multiplicativity of the Ideal Norm}
    
    \begin{prop}{}
        Let $R$ be an integral domain and let $P$ be a nonzero prime ideal. Then for all $n\in\N$,
        \begin{equation*}
            R /P^n \iso R_P /P_P^n.
        \end{equation*}
    \end{prop}

    \begin{proof}[Proof Sketch]
        The isomorphism is given by
        \begin{equation*}
            r+P^n \mapsto \frac{r}{1} + P_P^n.
        \end{equation*}
    \end{proof}

    \clearpage
    
    \begin{boxyrecall}{}
        Let $R$ be an integral domain and suppose an ideal $I\subseteq R$ is such that
        \begin{equation*}
            I = P_1^{n_1}\cdots P_k^{n_k}
        \end{equation*}
        for some pairwise coprime prime ideals $P_1,\ldots,P_k$ of $R$. Then by the CRT,
        \begin{equation*}
            R /I \iso R /P_1^{n_1}\times\cdots\times R /P_k^{n_k}.
        \end{equation*}
        If $R = \mO_K$ for some number field $K$, then
        \begin{equation*}
            N\left( I \right) = N\left( P_1^{n_1} \right)\cdots N\left( P_k^{n_k} \right).
        \end{equation*}
        Hence it suffices to show
        \begin{equation*}
            N\left( P^n \right) = N\left( P \right)^n
        \end{equation*}
        for any prime ideal $P\subseteq R$ and $n\in\N$.
    \end{boxyrecall}
    
    \begin{prop}{}
        Let $R$ be a DVR and let $P$ be a maximal ideal of $R$. If $R /P$ is finite, then
        \begin{equation*}
            \left| R /P^n \right| = \left| R /P \right|^n
        \end{equation*}
        for all $n\in\N$.
    \end{prop}

    \begin{proof}
        We use induction on $n$. 

        Suppose
        \begin{equation*}
            \left| R /P^{n-1} \right| = \left| R /P \right|^{n-1}
        \end{equation*}
        for some $n>1$. Consider
        \begin{equation*}
            \begin{aligned}
                \phi:R /P^{n}&\to R /P^{n-1} \\
                r+P^{n} &\mapsto r+P^{n-1}
            \end{aligned} .
        \end{equation*}
        Since $P^{n}\subseteq P^{n-1}$, $\phi$ is well-defined. Clearly $\phi$ is an epimorphism. Moreover,
        \begin{equation*}
            \ker\left( \phi \right) = P^{n-1} /P^n
        \end{equation*}
        By the first isomorphism theorem on $\phi$ (or the third isomorphism theorem alternatively),
        \begin{equation*}
            \left( R /P^n \right) / \left( P^{n-1} /P^n \right) \iso R /P^n.
        \end{equation*}
        This implies
        \begin{equation*}
            \left| R /P^n \right| = \left| P^{n-1} /P^n \right|\left| R /P \right|^n.
        \end{equation*}
        Hence it remains to show $\left| P^{n-1} /P^n \right| = \left| R /P \right|$.

        Since $P$ is a maximal ideal, $F = R /P$ is a field. Consider $V = P^{n-1} /P^n$ as a $F$-vector space with the scalar multiplication
        \begin{equation*}
            \left( r+P \right)\left( a+P^n \right) = ra+P^n.
        \end{equation*}
        Say $P = \left< \pi \right>$. Let $x\in V$. Then
        \begin{equation*}
            x = a+P^n
        \end{equation*}
        for some $a\in P^{n-1}$. That is, $a \in\left< \pi^{n-1} \right>$, so that $a=c\pi^{n-1}$ for some $c\in R$. Hence
        \begin{equation*}
            x = a+P^n = c\pi^{n-1}+P^n = \left( c+P \right)\left( \pi^{n-1}+P^n \right).
        \end{equation*}
        Since $x$ was arbitrary, it follows $\pi^{n-1}+P^n$ spans $V$, so that $\dim_F\left( V \right) = 1$. That is, $V\iso F$ as $F$-vector spaces. Thus
        \begin{equation*}
            \left| P^{n-1} /P^n \right| = \left| V \right| = \left| F \right| = \left| R /P \right|.
        \end{equation*}
    \end{proof}
    
    \clearpage

    \begin{theorem}{Multiplicativity of the Ideal Norm}
        Let $R = \mO_K$ for some number field $K$. If $I,J$ are nonzero ideals of $R$, then
        \begin{equation*}
            N\left( IJ \right) = N\left( I \right)N\left( J \right).
        \end{equation*}
    \end{theorem}
    
    \begin{proof}
        Let $P$ be a nonzero prime ideal of $R$. It suffices to show
        \begin{equation*}
            N\left( P^n \right) = N\left( P \right)^n.
        \end{equation*}
        But:
        \begin{equation*}
            N\left( P^n \right) = \left| R /P^n \right| = \left| R_P /P_P^n \right| = \left| R_P /P_P \right|^n = \left| R /P \right|^n = N\left( P \right)^n.
        \end{equation*}
    \end{proof}
    
    \subsection{Further Application of DVR's}

    \begin{theorem}{DVR Characterization}
        Let $R=\mO_K$ and let $S\subseteq R$ be a subring such that $\left[ R:S \right] = n < \infty$ (index as an additive subgroup).
        \begin{enumerate}
            \item $S=R$ if and only if $S_P$ is a DVR for all nonzero prime ideal $P\subseteq S$.
            \item Let $P\subseteq S$ be a prime ideal and let $p\in P$ be a prime number.\footnotemark[1] If $p\nmid n$, then $S_P$ is a DVR.
        \end{enumerate}
        
        \noindent
        \begin{minipage}{\textwidth}
            \footnotetext[1]{Again, such a prime exists due to Lagrange.}
        \end{minipage}
    \end{theorem}

    \rruleline

    \np (a) itself alone is not practical, since it is difficult to prove $S_P$ is a DVR for all prime $P\subseteq S$. (b) simplifies things a lot.

    \np Note that (b) is a \textit{huge} generalization of
    \begin{equation*}
        \text{squarefree $\disc\left( \alpha \right)$} \implies \mO_{\Q\left( \alpha \right)}=\Z\left[ \alpha \right].
    \end{equation*}
    Here is an explanation.

    Consider the case
    \begin{equation}
        K = \Q\left( \alpha \right), \alpha\in\mO_K=R, S=\Z\left[ \alpha \right], \rank\left( R \right)=\rank\left( S \right)=\left[ K:\Q \right].
    \end{equation}
    By Assignment 2, we know
    \begin{equation*}
        \left[ R:S \right] < \infty.
    \end{equation*}
    Moreover,
    \begin{equation*}
        \disc\left( \alpha \right) = \left[ R:S \right]^{2}\disc\left( K \right).
    \end{equation*}
    Therefore, 
    \begin{equation*}
        p^{2}\nmid\disc\left( \alpha \right) \implies p\nmid\left[ R:S \right].
    \end{equation*}
    Hence, when $\disc\left( \alpha \right)$ is squarefree in particular, the above implication always holds, so by Theorem 3.22 $S_P$ is a DVR for any prime $P\subseteq S$.
    
    \np But \textit{sometimes} (and by sometimes we mean \textit{always}) we have $p\in P$ such that $p|n$. What should we do in that case?

    \begin{prop}{}
        Let $\alpha\in\A$, let $f\in\Z\left[ x \right]$ be the minimal polynomial for $\alpha$, and let $p\in\Z$ be prime. Say
        \begin{equation*}
            f = p_1^{n_1}\cdots p_k^{n_k}
        \end{equation*}
        is the irreducible factorization of $f$ in $\Z_p\left[ x \right]$. Then the prime ideals of $\Z\left[ \alpha \right]$ which has $p$ are exactly $\left< p_i\left( \alpha \right),p \right>$. 
    \end{prop}

    \placeqed[Assignment 6]

    \clearpage
    
    \np Proposition 3.23 does not say
    \begin{equation*}
        \left< p \right> = \left< p_1\left( \alpha \right),p \right>^{n_1}\cdots\left< p_k\left( \alpha \right),p \right>^{n_k}.   
    \end{equation*}
    A counterexample is when $\alpha=\sqrt{5}$.

    \np Again, consider the case in [3.9]. Let $P\subseteq S$ be a nonzero prime ideal. Then
    \begin{equation*}
        \Z\left[ \alpha \right] \iso \Z\left[ x \right] /\left< f \right>,
    \end{equation*}
    where $f$ is the minimal polynomial for $\alpha$. Now, $\Z\left[ x \right]$ is Noetherian due to \textit{Hilbert's basis theorem},\footnote{Another stolen fact from commutative algebra!} and quotients of a Noetherian ring is Noetherian. Hence $\Z\left[ \alpha \right]$ is Noetherian, so that $S_P$ is local, Noetherian, and not a field.

    Hence in practice, we need only check that $P_P$ is principal.

    \begin{example}{}
        Let $f = x^{4}-5x^{2}+7$, which is irreducible over $\Q$. Then $\disc\left( f \right) = 1008 = 2^{4}3^{2}7^1$. Let $\alpha\in\CC$ be a root of $f$ and let $K=\Q\left( \alpha \right)$. Let $R=\mO_K$ and let $S=\Z\left[ \alpha \right]$. Prove $S=R$.
    \end{example}

    \begin{proof}
        It suffices to show that every prime ideal which has $2$ or $3$ is a DVR.

        \begin{case}
            $p=2$.

            Observe
            \begin{equation*}
                f = x^{4}+x^{2}+1 = \left( x^{2}+x+1 \right)^{2}
            \end{equation*}
            over $\Z_2$. Hence the only prime ideal of $S$ which has $2$ is $P=\left< \alpha^{2}+\alpha+1,2 \right>$. By the above comment, it suffices to check $P_P$ is a principal ideal of $S_P$.

            Dividing $f$ by $x^{2}+x+1$ over $\Z$, we obtain
            \begin{equation*}
                f = \left( x^{2}-x-5 \right)\left( x^{2}+x+1 \right) + \left( 6x+12 \right).
            \end{equation*}
            This means
            \begin{equation*}
                0 = f\left( \alpha \right) = \left( \alpha^{2}+\alpha+1 \right)\left( \alpha^{2}-\alpha-5 \right) + \left( 6\alpha+12 \right) \implies 6\alpha+12\in P.\footnotemark[1]
            \end{equation*}
            Dividing by $2$,
            \begin{equation*}
                2\left( 3\alpha+6 \right)\in \left( \alpha^{2}+\alpha+1 \right)S.
            \end{equation*}

            Suppose for contradiction $3\alpha+6\in P$. Then
            \begin{equation*}
                3\alpha+6\in P \implies 3\alpha\in P\implies \alpha\in P\implies 1\in P,
            \end{equation*}
            since $\alpha$ divides $\alpha^{2}+\alpha$. Since $P$ is prime, this is a contradiction.

            Hence
            \begin{equation*}
                -2\left( 3\alpha+6 \right) = \left( \alpha^{2}+\alpha+1 \right)\left( \alpha^{2}-\alpha-5 \right) \implies 2 = \frac{-1}{3\alpha+6}\left( \alpha^{2}+\alpha+1 \right)\left( \alpha^{2}-\alpha-5 \right)
            \end{equation*}
            in $S_P$, so that
            \begin{equation*}
                2 \in \left( \alpha^{2}+\alpha+1 \right)S_P \implies P_P = 2S_P + \left( \alpha^{2}+\alpha+1 \right)S_P = \left( \alpha^{2}+\alpha+1 \right)S_P.
            \end{equation*}
            Thus $S_P$ is principal.
        \end{case}

        \begin{case}
            $p=3$. 

            Observe
            \begin{equation*}
                f = x^{4}+x^{2}+1 = \left( x+1 \right)^{2}\left( x+2 \right)^{2}
            \end{equation*}
            over $\Z_3$. Hence the prime ideals of $S$ which has $3$ are $\left< \alpha+1,3 \right>,\left< \alpha+2,3 \right>$. Over $\Z$, we have
            \begin{equation*}
                f\left( -2 \right) = f\left( -1 \right) = 3.
            \end{equation*}
            Using the remainder theorem, this means
            \begin{equation*}
                f = \left( x+1 \right)q_1+f\left( -1 \right) = \left( x+1 \right)q_1+3 \implies 0 = f\left( \alpha \right) = \left( \alpha+1 \right)q_1\left( \alpha \right)+3 \implies 3\in\left< \alpha+1 \right>. 
            \end{equation*}
            Similarly $x\in\left< \alpha+2 \right>$. Hence $P_1 = \left( \alpha+1 \right)S, P_2=\left( \alpha+2 \right)S$. Thus
            \begin{equation*}
                \begin{aligned}
                    {P_1}_{P_1} & = \left( \alpha+1 \right)S_{P_1}, \\
                    {P_2}_{P_2} & = \left( \alpha+2 \right)S_{P_2}, \\
                \end{aligned} 
            \end{equation*}
            so that $S_{P_1},S_{P_2}$ are DVR's.
        \end{case}
        
        \noindent
        \begin{minipage}{\textwidth}
            \footnotetext[1]{Of course, we can \textit{easily} see $6\alpha+12\in P$ since $2$ divides it. However, how are we supposed to know it is the \textit{right} multiple of $2$ to look at without this computation?}
        \end{minipage}
    \end{proof}

    \np To practive calculations, visit \texttt{lmfdb.org}.
    
    \begin{boxyrecall}{}
        Let $R$ be a DVR and let $M=\left< \pi \right>$ be the unique maximal ideal of $R$. Let $K$ be the field of fractions of $R$ and let $x\in R$ be nonzero and nonunit. That is, $\left< x \right>$ is a proper ideal of $R$. Then we know for some $m\in\N$
        \begin{equation}
            \left< x \right> = \left< \pi \right>^m = \left< \pi^m \right>. 
        \end{equation}
        In other words, [3.10] is the unique way of factoring any nonzero proper ideal of $R$.

        Moreover, it follows from [3.10] that
        \begin{equation*}
            x = u\pi^m
        \end{equation*}
        for some $u\in R^{\times}$. This is why we called $\pi$ a \textit{uniformizer}. 
    \end{boxyrecall}

    Therefore, for any $y\in K$, there are $m\in\Z,u\in R^{\times}$ such that
    \begin{equation}
        y = u\pi^m \implies y\in R \text{ or } \frac{1}{y} = u^{-1}\pi^{-m}\in R.
    \end{equation}

    \begin{example}{}
        Consider $f=x^{3}+2x-8\in\Q\left[ x \right]$ which is irreducible over $\Q$ with $\disc\left( f \right) = -1760 = -2^55^111^1$. Let $\alpha\in\CC$ be a root of $f$, $K=\Q\left( \alpha \right), R=\mO_K, S=\Z\left[ \alpha \right]$. Then $R\neq S$.
    \end{example}

    \begin{proof}
        Observe that
        \begin{equation*}
            f = x^{3}
        \end{equation*}
        over $\Z_2$, so that $P = \left< \alpha,2 \right>$ is the unique prime ideal of $S$ which has $2$. 

        To show $R\neq S$, it suffices to show that $S_P$ is not a DVR. As always, proving this is equivalent to showing $P_P$ is not principal. Suppose $P_P$ is principal, say $P_P = \left< \pi \right>$ for some $\pi\in S_P$, for contradiction. Then we have
        \begin{equation*}
            \alpha = u_1\pi^n, 2=u_2\pi^m
        \end{equation*}
        for some $u_1,u_2\in R^{\times}$ and $n,m\in\N$. By [3.11], this means $\frac{\alpha}{2}\in S_P$ or $\frac{2}{\alpha}\in S_P$.

        \begin{case}
            \textit{Suppose $\frac{\alpha}{2}\in S_P$.}

            This means
            \begin{equation*}
                \frac{\alpha}{2} = \frac{a+b\alpha+c\alpha^{2}}{d+e\alpha+k\alpha^{2}}
            \end{equation*}
            for some $a,b,c,d,e,k\in\Z$, where $d+e\alpha+k\alpha^{2}\notin P$. So
            \begin{equation*}
                d\alpha+e\alpha^{2}+k\left( -2\alpha+8 \right) = 2a+2b\alpha+2c\alpha^{2}
            \end{equation*}
            using relation $f\left( \alpha \right) = 0$. Since $1,\alpha,\alpha^{2}$ form a basis, it follows
            \begin{equation*}
                d-2k = 2b
            \end{equation*}
            using the coefficients of $\alpha$. This means
            \begin{equation*}
                d = 2k+2b\in P,
            \end{equation*}
            so that $d+e\alpha+k\alpha^{2}\in P$, which is a contradiction.
        \end{case}

        \begin{case}
            \textit{Suppose $\frac{2}{\alpha\in S_P}$.}

            This means 
            \begin{equation*}
                \frac{2}{\alpha} = \frac{a+b\alpha+c\alpha^{2}}{d+e\alpha+k\alpha^{2}}
            \end{equation*}
            for some $a,b,c,d,e,k\in\Z$, where $d+e\alpha+k\alpha^{2}\notin P$. So
            \begin{equation*}
                2d+2e\alpha+2k\alpha^{2} = a\alpha+b\alpha^{2}+c\left( -2\alpha+8 \right) \implies 2d = 8c \implies d = 4c\in P \implies d+e\alpha+k\alpha^{2}\in P,
            \end{equation*}
            which is a contraidction.
        \end{case}
    \end{proof}

    \begin{example}{}
        Consider $f=x^{3}-x^{2}+5x+1\in\Q\left[ x \right]$ which is irreducible over $\Q$ with $\disc\left( f \right) = -2^{2}3^17^2$. Let $\alpha\in\CC$ be a root of $f$, $K=\Q\left( \alpha \right),R=\mO_K,S=\Z\left[ \alpha \right]$. Is $R=S$?
    \end{example}

    \begin{answer}
        Observe
        \begin{equation*}
            f = \left( x+1 \right)^{3}
        \end{equation*}
        over $\Z_2$, so $P=\left< \alpha+1,2 \right>$ is the unique maximal ideal of $S$ which has $2$.

        Over $\Z$, we have
        \begin{equation*}
            f\left( -1 \right) = -6 \implies f = \left( x+1 \right)q-6
        \end{equation*}
        for some $q\in\Z\left[ x \right]$, so that
        \begin{equation*}
            0 = \left( \alpha+1 \right)q\left( \alpha \right)-6 \implies 6\in\left< \alpha+1 \right>. 
        \end{equation*}
        Since $3\notin P$,\footnote{$2\in P$ implies $p\notin P$ for any prime $p\in\N$, since otherwise $1\in P$ so $P$ blows up to $S$.} we have
        \begin{equation*}
            2 = \frac{1}{3}6 \in \left( \alpha+1 \right)S_P \implies P_P = \left( \alpha+1 \right)S_P.
        \end{equation*}

        Moreover, over $\Z_7$,

        \begin{equation*}
            f = \left( x+2 \right)^{3},
        \end{equation*}
        so that $Q= \left< \alpha+2,7 \right>$ is the unique prime ideal of $S$ which has $7$. Over $\Z$,
        \begin{equation*}
            f\left( -2 \right) = -21 \implies 21\in\left< \alpha+2 \right>\subseteq S \implies 7 = \frac{1}{3} 21\in\left( \alpha+2 \right)S_Q \implies Q_Q = \left( \alpha+2 \right)S_Q.
        \end{equation*}
        Hence $P_P$ is principal for any prime $P\subseteq S$, which means $R=S$.
    \end{answer}

    \clearpage
    
    \np We shall prove Theorem 3.22 now:

    \setcounter{stcounter}{21}
    \begin{theorem}{DVR Characterization}
        Let $R=\mO_K$ and let $S\subseteq R$ be a subring such that $\left[ R:S \right] = n < \infty$ (index as an additive subgroup).
        \begin{enumerate}
            \item $S=R$ if and only if $S_P$ is a DVR for all nonzero prime ideal $P\subseteq S$.
            \item Let $P\subseteq S$ be a prime ideal and let $p\in P$ be a prime number.\footnotemark[1] If $p\nmid n$, then $S_P$ is a DVR.
        \end{enumerate}
        
        \noindent
        \begin{minipage}{\textwidth}
            \footnotetext[1]{Again, such a prime exists due to Lagrange.}
        \end{minipage}
    \end{theorem}
    \setcounter{stcounter}{23}
    
    \rruleline
    
    \np Here is a little remark on the assumption $\left[ R:S \right] = n < \infty$. Recall that
    \begin{equation*}
        K = \frc\left( R \right).
    \end{equation*}
    For all $r\in R$, observe
    \begin{equation*}
        nr+S = n\left( r+S \right) = 0+S
    \end{equation*}
    in $R /S$, so that $nr\in S$. This means, given any $\frac{a}{b}\in K$,
    \begin{equation*}
        \frac{a}{b} = \frac{na}{nb}\in\frc\left( S \right),
    \end{equation*}
    so that
    \begin{equation*}
        \frc\left( R \right) = K = \frc\left( S \right).
    \end{equation*}

    It follows that, if $P\subseteq S$ is a prime ideal, then
    \begin{equation*}
        \frc\left( S_P \right) = K.
    \end{equation*}

    \begin{lemma}{Lying-over Theorem}
        Let $S,R$ be integral domains with $S\subseteq R$ and suppose $R$ is integral over $S$. Let $P\subseteq S$ be a prime ideal. Then there is a prime ideal $Q\subseteq R$ such that $P=S\cap Q$.
    \end{lemma}
    
    \begin{proof}[Proof Sketch]
        Consider
        \begin{equation*}
            R_P = \left\lbrace \frac{a}{b}: a\in R, b\in S\setminus P \right\rbrace.
        \end{equation*}

        \begin{claim}
            \textit{$R_P$ is a local ring.}

            Exercise!
        \end{claim}

        Clearly, $S_P\subseteq R_P$. Moreover, using the fintely generated module trick, we can show $R_P$ is integral over $S_P$. Let $M\subseteq R_P$ be the unique maximal ideal of $R_P$ and let $Q = M\cap R$. By Assignment 1, $Q$ is prime. Moreover,
        \begin{equation*}
            Q\cap S = \left( M\cap R \right)\cap S = \left( M\cap S_P \right)\cap S.
        \end{equation*}
        By Assignment 1, it follows $M\cap S_P$ is maximal. That is, $M\cap S_P = P_P$, since $S_P$ is a local ring. Hence
        \begin{equation*}
            Q\cap S = P_P\cap S = P,
        \end{equation*}
        as required.
    \end{proof}
    
    \begin{boxyproof}{Proof of Theorem 3.22 (a)}
        It suffices to prove the reverse direction.

        Suppose $S_P$ is a DVR for all nonzero prime ideal $P\subseteq S$. Observe $R=\mO_K$ is integral over $S$, as $S\supseteq\Z$ and $R$ is integral over $\Z$. Let $P$ be a nonzero prime ideal of $S$, so that there is prime $Q\subseteq R$ such that $P=Q\cap S$ by the lying-over theorem.

        \begin{claim}
            \textit{$S_P = R_Q$.}

            ($\subseteq$) Let $x\in S_P$. This means
            \begin{equation*}
                x = \frac{a}{b}
            \end{equation*}
            for some $a,b\in S, b\notin P$. That is, $a,b\in R, b\notin Q$ (as $P=Q\cap S$), so that
            \begin{equation*}
                x\in R_Q.
            \end{equation*}

            ($\supseteq$) Let $\alpha\in K\setminus S_P$. Then
            \begin{equation*}
                \alpha = u\pi^n,
            \end{equation*}
            where $\pi$ is a uniformizer for $S_P$, $u\in S_P^{\times}$ and $n\in\Z$. Since $\alpha\notin S_P$, it follows $n<0$. This means $-1-n\geq 0$, so that
            \begin{equation*}
                \pi^{-1} = \underbrace{\pi^{-1-n}}_{\in S_P}\underbrace{\pi^n}_{\in S_P\left[ \alpha \right]} \in S_P\left[ \alpha \right] \implies S_P\left[ \alpha \right] = K.
            \end{equation*}
            However, $S_P\subseteq R_Q\subset K$, where the last containment is proper since $Q$ is nonzero. Thus $\alpha\notin R_Q$ (otherwise $R_Q\supseteq S_P\left[ \alpha \right] = K$). 
        \end{claim}

        Let's \textit{unfix} $P,Q$.

        Let $y\in R$ and consider
        \begin{equation*}
            D = \left\lbrace b\in S: by\in S \right\rbrace.
        \end{equation*}
        Immediately, $D$ is an ideal of $S$.

        \begin{claim}
            \textit{$D=S$.}

            Suppose $D\neq S$ and let $P\subseteq S$ be a prime ideal containing $D$. Consider a prime ideal $Q\subseteq R$ with $P=S\cap Q$. From before,
            \begin{equation*}
                S_P = R_Q.
            \end{equation*}
            If $y=\frac{a}{b}$ with $a,b\in S$, then
            \begin{equation*}
                by = a\in S \implies b\in D\subseteq P.
            \end{equation*}
            Hence $y\notin S_P=R_Q$. But $y\in R\subseteq R_Q$, this is a contradiction. Hence $D=S$.
        \end{claim}

        Using $1\in D$,
        \begin{equation*}
            y\in R\implies y\in S \implies S=R.
        \end{equation*}
        \qedplacedtrue
    \end{boxyproof}

    \begin{proof}[Proof of 3.22(b)]
        Let $P\subseteq S$ be a prime ideal and let $p\in P$ be a prime number. Suppose $p\nmid n$.  Since $p\in P$ and $\gcd\left( p,n \right) = 1$, $n\notin P$. As before, consider
        \begin{equation*}
            P = Q\cap S.
        \end{equation*}
        Since $R_Q$ is a DVR, it suffices to prove the following claim.

        \setcounter{claimcounter}{2}
        \begin{claim}
            \textit{$S_P=R_Q$.}

            We know $S_P\subseteq R_Q$.

            Let $x\in R_Q$, so that
            \begin{equation*}
                x = \frac{a}{b}
            \end{equation*}
            for some $a,b\in R, b\notin Q$. Then by Lagrange, $na,nb\in S$. Moreover, $b\notin Q$ implies $b\notin P$, and we know $n\notin P$. Hence $np\notin P$ as $P$ is a prime ideal, so that
            \begin{equation*}
                x = \frac{a}{b} = \frac{na}{nb} \in S_P,
            \end{equation*}
            as required.
        \end{claim}
    \end{proof}
    
    \subsection{Kummer-Dedekind Theorem}

    Recall that our goal was to factor $\left< p \right>$ into prime ideals of a ring of integers. The following theorem does the job.

    \begin{theorem}{Kummer-Dedekind Theorem}
        Let $K = \Q\left( \alpha \right)$ be a number field and let $R=\mO_K$. Let $S=\Z\left[ \alpha \right]$ and let $m\in\Z\left[ x \right]$ be the minimal polynomial for $\alpha$ over $\Q$. Let $p$ be a prime number with $p\nmid\left[ R:S \right]$. Suppose
        \begin{equation*}
            m = p_1^{e_1}\cdots p_k^{e_k}
        \end{equation*}
        is the irreducible factorization of $m$ over $\Z_p$. Let $P_i = \left( p_i\left( \alpha \right),p \right)R$ for all $i\in\left\lbrace 1,\ldots,n \right\rbrace$. Then
        \begin{equation*}
            \left< p \right> = P_1^{e_1}\cdots P_k^{e_k} 
        \end{equation*}
        is the prime factorization of $\left< p \right> $ in $R$.
    \end{theorem}

    \begin{proof}
        Consider the homomorphism
        \begin{equation*}
            \begin{aligned}
                \phi:\Z\left[ x \right] &\to R /P_i \\
                f & \mapsto f\left( \alpha \right) + P_i
            \end{aligned} .
        \end{equation*}
        Immediately, $p_i,p\in\ker\left( \phi \right)$. Moreover,
        \begin{equation*}
            \Z\left[ x \right] /\left< p_i,p \right> \iso \Z_p\left[ x \right] / \left< p_i \right>,  
        \end{equation*}
        but $p_i$ is irreducible so $\left< p_i \right>$ is maximal. It follows $\left< p_i,p \right>$ is also maximal. It follows
        \begin{equation}
            \ker\left( \phi \right) = \left< p_i,p \right> \text{ or } \ker\left( \phi \right) = \Z\left[ x \right].
        \end{equation}

        To apply the first isomorphism theorem, we require the following fact.

        \begin{claim}
            \textit{$\phi$ is surjective.}

            We know $p\nmid\left[ R:S \right]$ and
            \begin{equation*}
                \left[ R:S \right] = \left[ R:\Z\left[ \alpha \right]+pR \right]\left[ \Z\left[ \alpha \right]+pR:S \right],
            \end{equation*}
            so that $p\nmid\left[ R:\Z\left[ \alpha \right]+pR \right],\left[ \Z\left[ \alpha \right]+pR:S \right]$.

            Moreover,
            \begin{equation*}
                \left[ R:pR \right] = \left[ R:\Z\left[ \alpha \right]+pR \right]\left[ \Z\left[ \alpha \right]+pR:pR \right] 
            \end{equation*}
            and
            \begin{equation*}
                \left[ R:pR \right] = \left| R /pR \right| = N\left( pR \right) = \left| N_{K /\Q}\left( p \right) \right| = p^{\left[ K:\Q \right]}.
            \end{equation*}

            Hence it follows $\left[ R:\Z\left[ \alpha \right]+pR \right] = 1$. That is,
            \begin{equation*}
                R = \Z\left[ \alpha \right]+pR.
            \end{equation*}
            Consider $s\in R$. Then
            \begin{equation*}
                s+P_i = f\left( \alpha \right) + pr + P_i\in R /P_i
            \end{equation*}
            for some $f\in\Z\left[ x \right]$ and $r\in R$. But $p\in P_i$, so that
            \begin{equation*}
                s+P_i = f\left( \alpha \right) + P_i
            \end{equation*}
        \end{claim}

        By the first isomorphism theorem and [3.12],
        \begin{equation*}
            \Z\left[ x \right] /\left< p_i,p \right> \iso R /P_i 
        \end{equation*}
        or
        \begin{equation*}
            \Z\left[ x \right] /\Z\left[ x \right] \iso R /P_i \implies P_i = R.
        \end{equation*}

        Without loss of generality assume $P_1,\ldots,P_r$ are such that
        \begin{equation*}
            R /P_i\iso \Z\left[ x \right] / \left< p_i,p \right>, \hspace{2cm}\forall i\leq r 
        \end{equation*}
        and suppose $P_{r+1},\ldots,P_k=R$. For $i\leq r$, let $f_i = \deg\left( p_i \right)$. This means
        \begin{equation*}
            N\left( P_i \right) = \left| R /P_i \right| = \left| \Z\left[ x \right] /\left< p_i,p \right>  \right| = \left| \Z_p\left[ x \right] /\left< p_i \right>  \right| = p^{f_i}.
        \end{equation*}

        \begin{claim}
            $P_1^{e_1}\cdots P_k^{e_k}\subseteq\left< p \right>$. 

            Recall $P_i = \left< p_i\left( \alpha \right),p \right>$ and
            \begin{equation*}
                m = p_1^{e_1}\cdots p_k^{e_k}
            \end{equation*}
            is the irreducible factorization of $m$ over $\Z_p$, so that
            \begin{equation*}
                m\left( \alpha \right) \in \left< p \right>. 
            \end{equation*}
        \end{claim}

        By Claim 2, $P_1^{e_1}\cdots P_r^{e_r}\subseteq \left< P \right>$, so that 
        \begin{equation*}
            \left< p \right> | P_1^{e_1}\cdots P_r^{e_r} .
        \end{equation*}
        Hence the prime factorization of $\left< p \right>$ is  
        \begin{equation*}
            \left< p \right> = P_1^{d_1}\cdots P_r^{d_r} 
        \end{equation*}
        for some $d_1\leq e_1, \ldots, d_r\leq e_r$.

        Finally, taking norms,
        \begin{equation*}
            p^{\left[ K:\Q \right]} = N\left( P_1 \right)^{d_1}\cdots N\left( P_r \right)^{d_r} = p^{f_1d_1}\cdots p^{f_rd_r}.
        \end{equation*}
        That is,
        \begin{equation*}
            \left[ K:\Q \right] = f_1d_1+\cdots+f_rd_r.
        \end{equation*}
        However,
        \begin{equation*}
            \left[ K:\Q \right] = \deg\left( m \right) = f_1e_1+\cdots+f_ke_k.
        \end{equation*}
        It follows
        \begin{equation*}
            f_1d_1+\cdots+f_rd_r=f_1e_1+\cdots+f_ke_k \implies r=k, d_i = e_i.
        \end{equation*}
        Thus
        \begin{equation*}
            \left< p \right> = P_1^{e_1}\cdots P_k^{e_k}. 
        \end{equation*}
    \end{proof}
    
    \np Here is a recap of the work over few sections. Suppose we want to factor an ideal $I$:
    \begin{equation*}
        I = P_1^{e_1}\cdots P_k^{e_k}.
    \end{equation*}
    Then,
    \begin{equation*}
        N\left( I \right) = N\left( P_1 \right)^{e_1}\cdots N\left( P_k \right)^{e_k}
    \end{equation*}
    where $N\left( P_i \right) = \left| R /P_i \right| = p_i^{f_i}$, as $R$ is not only a Dedekind domain, but also a ring of integers; quotients by a prime ideal of a ring of integers are guaranteed to be finite.

    This means
    \begin{equation*}
        p_i\in P_i \implies \left< p_i \right>\subseteq P_i \implies P_i|\left< p_i \right>.
    \end{equation*}
    Moreover, so long as $p_i\nmid\left[ R:S \right]$,
    \begin{equation*}
        \text{Kummer-Dedekind: \textit{$\left< p_i \right>$ factors like the minimal polynomial for $\alpha$ over $\Z_{p_i}$.}}
    \end{equation*}
    Finally, if $P_i = \left< q_i\left( \alpha \right),p_i \right>$, then
    \begin{equation*}
        N\left( P_i \right) = \left| \Z_{p_i}\left[ x \right] /\left< q_i \right>  \right| = p_i^{\deg\left( q_i \right)},
    \end{equation*}
    which helps us determine $e_i$.

    \clearpage

    \begin{example}{}
        Let $f=x^{3}-x^{2}+3\in\Q\left[ x \right]$ which is irreducible and let $\alpha\in\CC$ be a root of $f$. Let $K=\Q\left( \alpha \right)$ and let $R=\mO_K$. Factor $I = \left< \alpha+2 \right>$.
    \end{example}

    \begin{answer}
        We first compute $N\left( I \right)$. By computing the norm, we get candidates for $N\left( P_i \right)$.

        Observe
        \begin{equation*}
            N\left( I \right) = \left| R /\left< \alpha+2 \right>  \right| = \left| N_{K /\Q}\left( \alpha+2 \right) \right|.
        \end{equation*}
        But $K = \Q\left( \alpha \right) = \Q\left( \alpha+2 \right)$. Moreover, $g=f\left( x-2 \right)$ is the minimal polynomial for $\alpha+2$,\footnotemark[1] so that
        \begin{equation*}
            \left| N_{K /\Q}\left( \alpha+2 \right) \right| = \left| N_{\Q\left( \alpha+2 \right) /\Q}\left( \alpha+2 \right) \right| = \left| g\left( 0 \right) \right| = \left| f\left( -2 \right) \right| = 3^{2}.
        \end{equation*}

        Observe that $\disc\left( f \right) = -3^17^111^1$. Since $3^{2}\nmid\disc\left( f \right)$, it follows
        \begin{equation*}
            3\nmid\left[ R:S \right].
        \end{equation*}
        Over $\Z_3$,
        \begin{equation*}
            f = x^{3}-x^{2} = x^{2}\left( x-1 \right).
        \end{equation*}
        By the Kummer-Dedekind theorem, it follows
        \begin{equation*}
            \left< 3 \right> = \left< \alpha,3 \right>^{2}\left< \alpha-1,3 \right>.   
        \end{equation*}

        Observe that $\alpha+2\notin\left< \alpha,3 \right>$, since otherwise $1\in\left< \alpha,3 \right>$, contradicting the fact that $\left< \alpha,3 \right>$ is a prime ideal (which we know by the Kummer-Dedekind theorem). On the other hand,
        \begin{equation*}
            \alpha+2 = \alpha-1+3 \in \left< \alpha-1,3 \right>. 
        \end{equation*}
        Hence
        \begin{equation*}
            I = \left< \alpha-1,3 \right>^{e_1} 
        \end{equation*}
        for some $e_1\in\N$. This means
        \begin{equation*}
            9 = N\left( \left< \alpha-1,3 \right>  \right)^{e_1} = \left( 3^{\deg\left( x-1 \right)} \right)^{e_1} = 3^{e_1} \implies e_1 = 2.
        \end{equation*}
        Thus
        \begin{equation*}
            I = \left< \alpha-1,3 \right>^{2}. 
        \end{equation*}

        \noindent
        \begin{minipage}{\textwidth}
            \footnotetext[1]{This trick works because $x\mapsto x+k$ for some $k\in\Q$ is an isomorphism. In other words, if we have quadratic, cubic, .., principal ideal instead of $I$, we have to find the minimal polynomial in a different way.}
        \end{minipage}
    \end{answer}
    
    \subsection{Ramification}

    Consider the following setting.

    Let $p\in\N$ be a prime number and let $K$ be a number field. Say
    \begin{equation*}
        \left< p \right> = P_1^{e_1}\cdots P_k^{e_k} 
    \end{equation*}
    is the prime factorization of $\left< p \right>$ in $R=\mO_K$. As before,
    \begin{equation*}
        N\left( P_i \right) = p^{f_i}
    \end{equation*}
    for some $f_i\in\N$, since we are factoring $\left< p \right>$ so $p|N\left( P_i \right)$. 

    \begin{definition}{\textbf{Ramification Index}, \textbf{Residue Field Degree}}
        We say $e_i$ is the \emph{ramification index} of $P_i$ over $p$ and $f_i$ the \emph{residue field degree} of $P_i$ over $p$.

        We say $p$ is \emph{ramified} in $K$ if some $e_i>1$. Otherwise, we say $p$ is \emph{unramified}.
    \end{definition}

    \np The idea is that
    \begin{equation*}
        \text{\textit{ramified prime}} \iff \text{\textit{complicated prime}}.
    \end{equation*}

    \np We can compute the residue field degree quite easily as follows:
    \begin{equation*}
        f_i = \left[ R /P_i:\Z_p \right].
    \end{equation*}
    
    \clearpage

    \begin{theorem}{}
        Let $K$ be a number field and let $p\in\N$ be a prime. Then
        \begin{equation*}
            \text{$p$ is ramified in $K$} \iff p|\disc\left( K \right).
        \end{equation*}
    \end{theorem}

    \begin{proof}
        ($\impliedby$) Beyond the scope of the course (a two week of algebra which we can't offer).

        ($\implies$) Let $P\subseteq\mO_K$ be a prime ideal such that $p\in P$ and $P$ is ramified so that $P^{2}|\left< p \right>$. Suppose $\left< p \right>$ factors as
        \begin{equation*}
            \left< p \right> = PI. 
        \end{equation*}
        Note that, if $Q\subseteq\mO_K$ is a prime ideal with $p\in Q$, then $I\subseteq Q$. We also know that $\left< p \right>\neq I$, as $\left< p \right> = PI$. Let $\alpha\in I\setminus\left< p \right>$. Let $\left\lbrace v_1,\ldots,v_n \right\rbrace$ be an integral basis for $\mO_K$ and let $\sigma_1,\ldots,\sigma_n$ be embeddigns of $K$ in $\CC$. Since $\alpha\in I\subseteq\mO_K$, write
        \begin{equation*}
            \alpha = m_1v_1+\cdots+m_nv_n
        \end{equation*}
        for some unique $m_1,\ldots,m_n\in\Z$. This means $p\nmid m_i$ for some $i$, so suppose $p\nmid m_1$ without loss of generality. Then, using elementary column operations,
        \begin{equation*}
            \disc\left( \alpha,v_2,\ldots,v_n \right) = \disc\left( m_1v_1,v_2,\ldots,v_n \right) = m_1^{2}\disc\left( v_1,\ldots,v_n \right) = m_1^{2}\disc\left( K \right).
        \end{equation*}
        Hence it remains to show $p|\disc\left( \alpha,v_2,\ldots,v_n \right)$, as $p\nmid m_1$.

        We may extend each $\sigma_i$ to $\sigma_i:L\to L$, where $L$ is the Galois closure of $K$. Let $S=\mO_L$. Suppose $Q$ is a prime ideal of $S$ such that $p\in Q$. Then $Q\cap\mO_K$ is a prime ideal of $\mO_K$ which has $p$ and so $\alpha\in Q$, as $I\subseteq Q\cap\mO_K$. For $\sigma\in\gal\left( L /\Q \right)$, we also have that $\alpha\in\sigma^{-1}\left( Q \right)$, so that $\sigma\left( \alpha \right)\in Q$. That is,
        \begin{equation*}
            \sigma_i\left( \alpha \right)\in Q
        \end{equation*}
        for all $i\in\left\lbrace 1,\ldots,n \right\rbrace$. Hence
        \begin{equation*}
            \disc\left( \alpha,v_2,\ldots,v_n \right) \in Q\cap\Z = p\Z.
        \end{equation*}
        In particupar, $p|\disc\left( \alpha,v_2,\ldots,v_n \right)$, as required.
    \end{proof}
    
    \np Suppose $K$ is a number field with $\left[ K:\Q \right] = n$. Let $p$ be a prime number with
    \begin{equation*}
        \left< p \right> = P_1^{e_1}\cdots P_k^{e_k} 
    \end{equation*}
    and $N\left( P_I \right) = p^{f_i}$. This means
    \begin{equation*}
        N\left( \left< p \right>  \right) = p^{f_1e_1}\cdots p^{f_ke_k} \implies p^n = p^{\sum^{k}_{i=1}f_ie_i} \implies n = \sum^{k}_{i=1}f_ie_i.
    \end{equation*}
    
    
    
    
    
    
    
    
    
    
    
    
    
    
    
    
    
    
    
    
    
    
    
    
    
    
    
    
    
    
    
    
    
    
    
    
    
    

\end{document}
