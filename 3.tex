\documentclass[pmath441]{subfiles}

%% ========================================================
%% document

\begin{document}

    \section{Prime Factorization}
    
    Let $K$ be a number field and let $R=\mO_K$. Let's recall some important properties of $R$ as a ring.
    \begin{enumerate}
        \item Every nonzero prime ideal of $R$ is maximal.
        \item If $I$ is a nonzero ideal, then $R /I$ is finite.
        \item $R$ is Noetherian.
    \end{enumerate}
    
    \subsection{Some Useful Ring Theory}
    
    \begin{prop}{}
        Let $R$ be a ring.\footnotemark[1] The following are equivalent.
        \begin{enumerate}
            \item $R$ is Noetherian.
            \item Every ascending chain of ideals stabilizes.\footnotemark[2] \hfill\textit{ascending chain condition (acc)}
            \item Every nonempty collection of ideals of $R$ has a maximal (with respect to inclusion) element.
        \end{enumerate}
        
        \noindent
        \begin{minipage}{\textwidth}
            \footnotetext[1]{Let us recall that a ring is always commutative and unital in our course.}
            \footnotetext[2]{This is the \textit{usual} definition of Noetherian ring in commutative algebra.}
        \end{minipage}
    \end{prop}
    
    \placeqed[Proof is left as an exercise]

    \np The idea for (b) $\implies$ (a) is that, given an ascending chain of ideals, the union is also an ideal. For this ideal to be finitely generated, it must be the case that the chain stabilizes. 

    For (b) $\implies$ (c), if we assume (c) is false, then we can construct an ascending chain of ideals that does not stabilize.
    
    \begin{prop}{A Glimpse of Prime Factorization}
        Let $R$ be a Noetherian ring and let $I$ be a proper ideal of $R$. Then there exists prime ideals $P_1,\ldots,P_n$ of $R$ such that
        \begin{enumerate}
            \item $I\subseteq P_i$ for $i$;
            \item $P_1\cdots P_n \subseteq I$.
        \end{enumerate}
    \end{prop}

    \rruleline

    \np We know that prime factorization of numbers does not work well in a ring of integers. After all, a ring of integers need not be a UFD! Hence, instead of factoring numbers, we are going to \textit{factor ideals} in $\mO_K$. This will work well, and introduce us the notion of \textit{Dedekind domains}. 
    
    Note that Proposition 3.2 is bit more general than we require, that it works for any \textit{Noetherian ring}. Indeed, any ring of integer is a Noetherian ring (Corollary 1.10.4, \textit{the} result of Chapter 1).

    \begin{boxyproof}{Proof of Proposition 3.2}
        Let $X$ be the collection of proper ideals of $R$ not having the property. Assume for contradiction that $X$ is nonempty. Let $I\in X$ be an maximal \textit{element} of $X$ (we do not insist that $I$ is a maximal \textit{ideal} in $R$).

        Clearly $I$ is not prime. If not, then take $P_1=I$ and observe that $I$ has the property. Since $I$ is not prime, we may find $a,b\in R$ such that $ab\in I$ but $a,b\notin I$. By maximality of $I$, $I+\left< a \right>,I+\left< b \right>\notin X$. Note that, for any ideal $J$, $IJ\subseteq I$ (this is a property of ideal product; check this!). Moreover, $ab\in I$ and $\left< a \right>,\left< b \right>$ are principal ideals, so that $\left< a \right>\left< b \right> = \left< ab \right> \subseteq I$. Hence it follows that     
        \begin{equation*}
            \left( I+\left< a \right>  \right)\left( I+\left< b \right>  \right) \subseteq I.
        \end{equation*}
        Hence $I+\left< a \right>, I+\left< b \right>\neq R$ (since $JR=RJ=J$ for any ideal $J$). Therefore, there are prime ideals $P_1,\ldots,P_n,Q_1,\ldots,Q_m$ such that
        \begin{enumerate}
            \item $I+\left< a \right>\subseteq P_i, I+\left< b \right>\subseteq Q_j$ for $i,j$ $\implies$ $I\subseteq I+\left< a \right>\subseteq P_i$, $I\subseteq I+\left< b \right>\subseteq Q_j$ for $i,j$; and
            \item $P_1\cdots P_n\subseteq I+\left< a \right>, Q_1\cdots Q_m\subseteq I+\left< b \right>$ $\implies$ $P_1\cdots P_nQ_1\cdots Q_m\subseteq\left( I+\left< a \right>  \right)\left( I+\left< b \right>  \right)\subseteq I$.
        \end{enumerate}
        Thus $I\notin X$, which is a contradiction.
    \end{boxyproof}
    
    \begin{definition}{\textbf{Coprime} Ideals}
        Let $R$ be a ring and let $I,J\subseteq R$ be prime ideals. We say $I,J$ are \emph{coprime} if and only if $I+J=R$.
    \end{definition}

    \np A motivation for the above definition comes from the Bezout lemma.

    \begin{prop}{}
        Let $R$ be a ring and let $I,J$ be coprime ideals of $R$. Then for any $n,m\in\N$, $I^n,J^m$ are coprime.
    \end{prop}

    \begin{proof}
        Since $I,J$ are proper, so are $I^n\subseteq I, J^m\subseteq J$. Suppose for contradiction that
        \begin{equation*}
            I^n+J^m \neq R.
        \end{equation*}
        Then $I^n+J^m\subseteq M$ for some maximal ideal $M$, which means $I^n,J^m\subseteq M$. But any maximal ideal is a prime ideal, so that $M$ is a prime ideal. Recall that, 
        \begin{equation*}
            \text{\textit{given two ideals $\tilde{I},\tilde{J}$ and a prime ideal $P$ such that $\tilde{I},\tilde{J}\subseteq P$, $\tilde{I}\subseteq P$ or $\tilde{J}\subseteq P$.}}
        \end{equation*}
        In particular, $I,J\subseteq M$. This means $I+J\subseteq M\neq R$, a contradiction.
    \end{proof}

    \np Recall the following theorem from ring theory.

    \begin{theorem}{Chinese Remainder Theorem}
        Let $R$ be a ring and let $I,J$ be coprime ideals of $R$. Then $R /IJ \iso R /I\times R /J$.
    \end{theorem}

    \begin{proof}
        "\textit{When we want two algebraic objects to be isomorphic, 99.9\% of the time we want to find an isomorphism.}" - Blake

        Since we are working with quotient rings, we resort to the first isomorphism theorem. Let
        \begin{equation*}
            \begin{aligned}
                \phi:R&\to R /I\times R /J \\
                x & \mapsto \left( x+I,x+J \right).
            \end{aligned} 
        \end{equation*}
        Then
        \begin{equation*}
            \ker\left( \phi \right) = I\cap J.
        \end{equation*}
        Now observe that,
        \begin{equation*}
            IJ\subseteq I\cap J = \left( I\cap J \right)R = \left( I\cap J \right)\left( I+J \right) = \underbrace{\left( I\cap J \right)I}_{\subseteq IJ} + \underbrace{\left( I\cap J \right)J}_{\subseteq IJ} \subseteq IJ,\footnotemark[1]
        \end{equation*} 
        so that
        \begin{equation*}
            IJ \subseteq I.
        \end{equation*}
        Hence we conclude
        \begin{equation*}
            \ker\left( \phi \right) = IJ.
        \end{equation*}
        To invoke the first isomorphism theorem, we want to show that $\phi$ is surjective. Take $a\in I, b\in J$ such that $a+b = 1$ (since $I+J=R$). For $x,y\in R$
        \begin{equation*}
            \begin{aligned}
                \phi\left( ax+by \right) & = \left( \underbrace{ax}_{\in I}+by+I,ax+\underbrace{by}_{\in J}+J \right) = \left( by+I,ax+J \right) \\
                                         & = \left( b+I,a+J \right)\left( y+I,x+J \right) = \left( 1+I,1+J \right)\left( y+I,x+J \right) = \left( y+I,x+J \right).
            \end{aligned} 
        \end{equation*}
        Note that we are using $a+b=1$ but $a+I=0+I, b+J=0+J$ to obtain the second-last equality.

        Thus $\phi$ is surjective and
        \begin{equation*}
            R /IJ \iso R /I\times R /J
        \end{equation*}
        by the first isomorphism theorem.
        
        \noindent
        \begin{minipage}{\textwidth}
            \footnotetext[1]{Note that the above argument worked because of the \textit{coprimeness} of $I,J$: $R=I+J$.}
        \end{minipage}
    \end{proof}
    
    \clearpage
    
    \begin{theorem}{Generalized Chinese Remainder Theorem}
        Let $R$ be a ring and let $I_1,\ldots,I_n$ be \textit{pairwise} coprime ideals. Then $R /I_1\cdots I_n \iso R /I_1 \times \cdots \times R /I_n$.
    \end{theorem}

    \rruleline

    \begin{prop}{}
        Let $R$ be a finite ring. Then 
        \begin{equation*}
            R \iso R /P_1^{n_1} \times \cdots \times R /P_m^{n_m}
        \end{equation*}
        for some distinct prime ideals $P_1,\ldots,P_m$ and $n_1,\ldots,n_m\in\N$.
    \end{prop}

    \rruleline

    \np In case $R$ is an integral domain, we can simply take $P_1 = \left\lbrace 0 \right\rbrace$ and \textit{call it a day!} In fact, the key idea for the general case is to identify $R$ with $R / \left\lbrace 0 \right\rbrace$.

    \begin{boxyproof}{Proof of Proposition 3.6}
        Note that
        \begin{equation*}
            R\text{ is finite} \implies R\text{ is Noetherian}.\footnotemark[1]
        \end{equation*}
        So we may find prime ideals $Q_1,\ldots,Q_k\subseteq R$ such that $Q_1\cdots Q_k=\left\lbrace 0 \right\rbrace$. \textit{Graping} the $Q_i$'s we obtain distinct prime ideals $P_1,\ldots,P_m$ such that
        \begin{equation*}
            P_1^{n_1}\cdots P_m^{n_m} = \left\lbrace 0 \right\rbrace.
        \end{equation*}
        For each $P_i$,
        \begin{equation*}
            R\text{ is finite and }P_i\text{ is prime}\implies R /P_i\text{ is finite integral domain} \implies R /P_i\text{ is a field}.
        \end{equation*}
        Hence each $P_i$ is maximal, which imply
        \begin{equation*}
            P_i+P_j = R, \hspace{2cm}\forall i\neq j.
        \end{equation*}
        It follows $P_i^{n_i}+P_j^{n_j}=R$. Hence $P_1,\ldots,P_m$ are pairwise coprime ideals, so by the generalized Chinese remainder theorem,
        \begin{equation*}
            R \iso R /\left\lbrace 0 \right\rbrace = R /P_1^{n_1}\cdots P_m^{n_m} \iso R /P_1^{n_1}\times\cdots R /P_m^{n_m}.
        \end{equation*}

        \noindent
        \begin{minipage}{\textwidth}
            \footnotetext[1]{"\textit{Good luck in finding an infinite ascending chain in a finite ring!}" - Blake}
        \end{minipage}
    \end{boxyproof}

    \subsection{Prime Ideals of a Ring of Integers}

    \begin{boxyrecall}{}
    Once again, let $K$ be a number field of degree $n$ and let $R = \mO_K$. 
        \begin{enumerate}
            \item $R$ is Noetherian. 
            \item $R /I$ is finite for any nonzero proper ideal $I$.
            \item Every ideal $\overline{J}$ of $R /I$ is of the form $\overline{J} = J /I$, where $J\subseteq R$ is an ideal such that $I\subseteq J$; moreover, $\overline{J}$ is prime if and only if $J$ is prime.\footnote{In fact, this is true for any ring!} \hfill\textit{correspondence theorem}
            \item $R /I \iso \left( R /I \right) / \left( P_1^{n_1} / I \right) \times \cdots \times \left( R /I \right) / \left( P_m^{n_m} / I \right) \iso R /P^{n_1}\times\cdots\times R /P_m^{n_m}$, where each $P_i\subseteq R$ is prime with $I\subseteq P_i$.
        \end{enumerate}
    \end{boxyrecall}
    Here are some bing ideas for this section:
    \begin{enumerate}
        \item To understand $I$, we study the prime ideals $P$ containing $I$. Turns out, for a prime ideal $P$,
            \begin{equation*}
                I\subseteq P \iff P\text{ is a \textit{prime factor} of $I$}.
            \end{equation*}
        \item The prime ideals of $R /I$ are $P /I$, where $P$ is a prime ideal containing $I$.
        \item Say $P$ is a prime ideal containing $I$. Then $\left| R /P \right| = p^m$ for some prime $p$ and $m\in\N$. Now,
            \begin{equation*}
                p^m + P = p^m\left( 1+P \right) = 0+P
            \end{equation*}
            by Lagrange's theorem, which imply that $p^m\in P$. Since $P$ is a prime ideal, it follows $p\in P$. Hence we have
            \begin{equation*}
                \left< p \right> \subseteq P. 
            \end{equation*}
            That is, any prime ideal containing $I$ also contains a principal ideal generated by \textit{an old-school prime number}. Because of this, we first search for ideals of the form $\left< p \right>$ to find candidates for prime factorization of $I$. 
    \end{enumerate}
    
    \begin{example}{}
        Let $K=\Q\left( \sqrt{2} \right), R = \mO_K = \Z\left[ \sqrt{2} \right]$. Find all prime ideals $P$ of $R$ containing $\left< 5 \right>$. 
    \end{example}

    \begin{answer}
        Observe that
        \begin{equation*}
            R /\left< 5 \right> = \Z\left[ \sqrt{2} \right] /\left< 5 \right> \iso \Z\left[ x \right] / \left< x^{2}-2,5 \right>  = \Z\left[ x \right] / \left< 5,x^{2}-2 \right> \iso \Z_5\left[ x \right] / \left< x^{2}-2 \right> .
        \end{equation*}
        But $x^{2}-2$ is irreducible over $\Z_5$, which means $\left< x^{2}-2 \right>$ is a maximal ideal of $\Z_5\left[ x \right]$. Therefore, $\Z_5\left[ x \right] /\left< x^{2}-2 \right>$ is a field, and so is $R /\left< 5 \right>$. Hence $\left< 5 \right>$ is a maximal ideal of $R$, which means the only prime ideal containing $\left< 5 \right>$ is $\left< 5 \right>$ itself.  
    \end{answer}

    \begin{example}{}
        Let $K=\Q\left( \sqrt{2} \right), R = \mO_K = \Z\left[ \sqrt{2} \right]$. Find all prime ideals $P$ of $R$ containing $\left< 7 \right>$. 
    \end{example}

    \begin{answer}
        Observe
        \begin{equation*}
            R /\left< 7 \right> = \Z\left[ x \right] / \left< x^{2}-2,7 \right> = \Z_7\left[ x \right] / \left< x^{2}-2 \right>.   
        \end{equation*}
        But $x^{2}-2$ is reducible over $\Z_7$, namely
        \begin{equation*}
            x^{2}-2 = \left( x+3 \right)\left( x+4 \right).
        \end{equation*}
        It follows $\left< x^{2}-2 \right> = \left< x+3 \right>\left< x+4 \right>$, and the two ideals $\left< x+3 \right>,\left< x+4 \right>$ are coprime. It follows by the Chinese remainder theorem that
        \begin{equation}
            \Z_7\left[ x \right] / \left< x^{2}-2 \right> \iso \Z_7\left[ x \right] / \left< x+3 \right> \times \Z_7\left[ x \right] / \left< x+4 \right> \iso \Z_7\times\Z_7,  
        \end{equation}
        where the last isomorphism is due to the first isomorphism theorem (or, we can intuitively think that we can replace $x$ by $-3,-4$ and retain every element of $\Z_7$ from $\Z_7\left[ x \right]$, respectively).

        The prime ideals of $\Z_7\times\Z_7$ are
        \begin{equation*}
            P_1 = \left< \left( 1,0 \right) \right>, P_2 = \left< \left( 0,1 \right) \right>. 
        \end{equation*}
        Now, given an isomorphism $\phi$, $\phi\left( \left< a \right>  \right) = \left< \phi\left( a \right) \right>$. Hence we have to \textit{undo} isomorphisms in [3.1] with elements $\left( 1,0 \right),\left( 0,1 \right)$ to figure out the prime ideals containing $\left< 7 \right>$:
        \begin{flalign*}
            && \left( 1,0 \right) & \mapsto \left( 1+\left< x+3 \right>,0+\left< x+4 \right>   \right) &&\\
            &&                   & \mapsto x+4+\left< x^{2}-2 \right> && \text{since $x+4$ is $1$ modulo $x+3$ and $0$ modulo $x+4$} \\
            &&                   & \mapsto x+4+\left< x^{2}-2,7 \right> && \\
            &&                   & \mapsto \sqrt{2}+4+\left< 7 \right>
        \end{flalign*}
        and
        \begin{equation*}
            \left( 0,1 \right) \mapsto \left( 0+\left< x+3 \right>,1+\left< x+4 \right>   \right)  
                               \mapsto \left( -x-3 \right) + \left< x^{2}-2 \right> 
                               \mapsto -x-3 + \left< x^{2},7 \right> 
                               \mapsto -\sqrt{2}-3 + \left< 7 \right> .
        \end{equation*}
        Therefore, the prime ideals in $R$ containing $7$ are $Q_1=\left< \sqrt{2}+4,7 \right>, Q_2=\left< -\sqrt{2}-3,7 \right>$. Note that we are including $7$ in each ideal in addition to $\sqrt{2}+4,-\sqrt{2}-3$, respectively, in order to mod out by $\left< 7 \right>$.  In fact, $\left< -\sqrt{2}-3,7 \right> = \left< \sqrt{2}+3,7 \right>$ and $\left( \sqrt{2}+3 \right)\left( \sqrt{2}-3 \right) = -7$, so that $Q_2 =  \left< \sqrt{2}+3 \right>$. 
        
        Note that $\left( \sqrt{2}+3 \right)\left( \sqrt{4} \right) = 14+7\sqrt{2}\in\left< 7 \right>$, so that $Q_1Q_2 = \left< 7 \right>$. That is, we factored $\left< 7 \right>$ into prime ideals!  
    \end{answer}

    \clearpage

    \begin{example}{}
        Let $K = \Q\left( \sqrt{2} \right), R=\mO_K=\left[ \sqrt{2} \right]$. Find all prime ideals $P$ of $R$ containing $\left< 2 \right>$.  
    \end{example}
    
    \begin{answer}
        We have
        \begin{equation*}
            R /\left< 2 \right> \iso \Z\left[ x \right] / \left< x^{2}-2,2 \right> \iso \Z_2\left[ x \right] / \left< x^{2}-2 \right> = \Z_2\left[ x \right] / \left< x^{2} \right>,
        \end{equation*}
        since $x^{2}-2 \equiv x^{2} \mod 2$. Since $\Z_2\left[ x \right] / \left< x^{2} \right>$ is very small,
        \begin{equation*}
            \Z_2\left[ x \right] / \left< x^{2} \right> = \left\lbrace 0+\left< x^{2} \right>,1+\left< x^{2} \right>,x+\left< x^{2} \right>,x+1+\left< x^{2} \right> \right\rbrace ,
        \end{equation*}
        given an ideal of $\Z_2\left[ x \right] / \left< x^{2} \right>$, we can explicitly write down the elements. 
        
        Let $P$ be a prime ideal of $\Z_2\left[ x \right] /\left< x^{2} \right>$. Since $P$ is an ideal, $0+\left< x^{2} \right>\in P$. Since $P$ is prime so proper, $1+\left< x^{2} \right>\notin P$. Also,
        \begin{equation*}
            \left( x+1+\left< x^{2} \right>  \right)^{2} = \left( x^{2}+2x+1+\left< x^{2} \right>  \right) = 1+\left< x^{2} \right>\notin P,
        \end{equation*}
        so that $x+1+\left< x^{2} \right>\notin P$, since $P$ is prime. Hence $P = \left< 0+\left< x^{2} \right>  \right>$ or $P = \left< x+\left< x^{2} \right>  \right>$. But $\Z_2\left[ x \right] / \left< x^{2} \right>$ is not an integral domain, since $x+\left< x^{2} \right> $ is a zero divisor. It follows that
        \begin{equation*}
            P = \left< x+\left< x^{2} \right>  \right>. 
        \end{equation*}

        Retracing the isomorphisms,
        \begin{equation*}
            x+\left< x^{2} \right> \mapsto x+\left< x^{2}-2,2 \right> \mapsto \sqrt{2} + \left< 2 \right>.   
        \end{equation*}
        Hence the only prime $Q\subseteq R$ with $2\in Q$ is
        \begin{equation*}
            Q = \left< \sqrt{2},2 \right> = \left< \sqrt{2} \right>.  
        \end{equation*}
        Note that
        \begin{equation*}
            \left< 2 \right> = \left< \sqrt{2} \right>^{2}.  
        \end{equation*}
        Hence we have a prime factorization of $\left< 2 \right>$ with \textit{multiplicity}. 
    \end{answer}
    
    \begin{prop}{}
        Let $K$ be a number field with $\left[ K:\Q \right]$ with $K=\Q\left( \alpha \right)$ such that $\mO_K=\Z\left[ \alpha \right]$.\footnotemark[1] Let $m\in\Z\left[ x \right]$ be the minimal polynomial for $\alpha$. If $p$ is prime and
        \begin{equation*}
            m = q_1^{n_1}\cdots q_k^{n_k}\in\Z_p\left[ x \right]\footnotemark[2]
        \end{equation*}
        for some distinct irreducible $q_1,\ldots,q_k\in\Z_p\left[ x \right]$, then
        \begin{enumerate}
            \item the prime ideals $P\subseteq\mO_K$ such that $p\in P$ are exactly of the form $P=\left< q_i\left( \alpha \right),p \right>$; and
            \item $\left< p \right> = \left< q_1\left( \alpha \right),p \right>^{n_1} \cdots \left< q_k\left( \alpha \right),p \right>^{n_k}$ in $\mO_K$.
        \end{enumerate}
        
        \noindent
        \begin{minipage}{\textwidth}
            \footnotetext[1]{Observe that $K=\Q\left( \alpha \right)$ does not add any assumption, since every number field is of the form due to the primitive element theorem.}
            \footnotetext[2]{To be more precise, we are referring to the polynomial $\overline{m}\in\Z_p\left[ x \right]$ we obtain by replacing every coefficient of $m$ by its equivalence class in $\Z_p$.}
        \end{minipage}
    \end{prop}

    \placeqed[We shall treat this as a fact for now!]
    
    \begin{example}{}
        Consider $\alpha\in\CC$ with $\alpha^{2}+\alpha+1 = 0$. Then $m = x^{2}+x+1$ is the minimal polynomial for $\alpha$ over $\Q$ and $\mO_{\Q\left( \alpha \right)} = \Z\left[ \alpha \right]$.

        Over $\Z_3$,
        \begin{equation*}
            m = \left( x+2 \right)\left( x+2 \right),
        \end{equation*}
        so that
        \begin{equation*}
            \left< 3 \right> = \left< \alpha+2,3 \right>^{2}.  
        \end{equation*}

        On the other hand, over $\Z_2$, $m$ is irreducible, so that
        \begin{equation*}
            \left< 2 \right> = \left< \alpha^{2}+\alpha+1,2 \right>. 
        \end{equation*}
    \end{example}

    \rruleline
    
    \subsection{Dedekind Domains}

    Dedekind domains are the rings where the ideal prime factorization happens.

    \begin{boxyrecall}{}
        Let $R,S$ be integral domains, $R\subseteq S$.
        \begin{enumerate}
            \item Let $\alpha\in S$. Then
                \begin{equation*}
                    \alpha\text{ is integral over $R$} \iff \text{there is monic $f\in R\left[ x \right]$ such that $f\left( \alpha \right)=0$} \iff \text{$R\left[ \alpha \right]$ is a finitely generated $R$-module}.
                \end{equation*}

            \item We say $S$ is integral over $R$ if and only if every element of $S$ is integral over $R$.
        \end{enumerate}
    \end{boxyrecall}
    
    \begin{definition}{\textbf{Integral Closure}}
        Let $R,S$ be integral domains, $R\subseteq S$.
        \begin{enumerate}
            \item The \emph{integral closure} of $R$ in $S$ is
                \begin{equation*}
                    \left\lbrace \alpha\in S: \alpha\text{ integral over $R$} \right\rbrace.
                \end{equation*}
            \item $R$ is \emph{integrally closed} if and only if the integral closure of $R$ in its field of fractions is $R$.
        \end{enumerate}
    \end{definition}

    \begin{example}{}
        $\Z$ is integrally closed.
    \end{example}

    \rruleline

    \np Let $K$ be a number field and let $R=\mO_K$. Let $F$ be the field of fractions of $R$. Given $\alpha\in K$, since $\alpha$ is an algebraic number, there is a polynomial $f\in\Z\left[ x \right]$ annihilating $\alpha$. Taking the leading coefficient $N\in\Z$ of $f$, it follows $N\alpha\in R$. Hence $\alpha\in F$, which imply that $K\subseteq F$.

    But $F$ is the smallest field containing $R$, so that $K=F$.

    \begin{prop}{}
        Let $K$ be a number field. Then $\mO_K$ is algebraically closed.
    \end{prop}
    
    \begin{proof}
        Let
        \begin{equation*}
            f = x^n+a_{n-1}x^{n-1}+\cdots+a_0\in\mO_K\left[ x \right]
        \end{equation*}
        and supose $f\left( \alpha \right) = 0$ for some $\alpha\in K$. Then each $a_i$ is an algebraic integer, so $\Z\left[ a_i \right]$ is a finitely generated $\Z$-module. Hence $\Z\left[ a_{n-1},\ldots,a_0 \right]$ is also finitely generated. Also,
        \begin{equation*}
            \alpha^n = -\sum^{n-1}_{j=0} a_j\alpha^j.
        \end{equation*}
        It follows that $\Z\left[ \alpha,a_{n-1},\ldots,a_0 \right]$ is finitely generated. Since $\Z$ is Noetherian and $\Z\left[ \alpha \right]\subseteq\Z\left[ \alpha,a_{n-1},\ldots,a_0 \right]$, $\Z\left[ \alpha \right]$ is finitely generated. Thus $\alpha$ is an algebraic integer, as required.
    \end{proof}
    
    \begin{definition}{\textbf{Dedekind Domain}}
        Let $R$ be an integral domain. We say $R$ is a \emph{Dedekind domain} if
        \begin{enumerate}
            \item $R$ is Noetherian;
            \item $R$ is integrally closed; and
            \item every nonzero prime ideal of $R$ is maximal.
        \end{enumerate}
    \end{definition}
    
    \begin{example}{}
        Let $K$ be a number field. Then $\mO_K$ is a Dedekind domain.
    \end{example}

    \rruleline

    \clearpage
    
    \np Here is a question for the section:
    \begin{equation*}
        \text{\textit{why is Def'n 3.3 the right definition for prime factorization?}}
    \end{equation*}
    It turns out (\textit{spoiler alert})\ldots
    \begin{enumerate}
        \item $\implies$ existence of prime factorization;
        \item $\implies$ prime ideals cannot be factored further; and
        \item $\implies$ uniqueness of prime factorization.
    \end{enumerate}
    Let us first explore the third implication. The following lemma will be \textit{the contradiction getter}, according to Blake.

    \begin{lemma}{}
        Let $R$ be a Dedekind domain and let $I$ be a proper nontrivial ideal of $R$. Let $F$ be the field of fractions of $R$. Then there is $\lambda\in F\setminus R$ such that $\lambda I\subseteq R$.
    \end{lemma}

    \begin{proof}
        Let $a\in I$ be nonzero. Since $R$ is Noetherian, we may find nonzero prime ideals $P_1,\ldots,P_r$ such that $P_1\cdots P_r\subseteq\left< a \right>$ by Proposition 3.2. Moreover, assume $r$ is minimal (i.e. there does not exist fewer prime ideals $Q_1,\ldots,Q_k$ such that $Q_1\cdots Q_k\subseteq\left< a \right>$). Let $M$ be a maximal ideal containing $I$.

        Since $P_1\cdots P_r\subseteq\left< a \right>\subseteq I\subseteq M$ and $M$ is prime, some $P_i$ is contained in $M$. Without loss of generality, suppose $P_1\subseteq M$. Since $P_1$ is a nonzero prime ideal of a Dedekind domain, it is maximal. Hence $P_1 = M$.

        \begin{case}
            \textit{$r=1$.}

            In this case,
            \begin{equation*}
                P_1\subseteq \left< a \right> \subseteq I \subseteq M = P_1, 
            \end{equation*}
            so that $I = P_1$ is a prime ideal. Take $\lambda = a^{-1}$, so that
            \begin{equation*}
                \lambda\left< a \right> = a^{-1}\left< a \right> = R \subseteq R.  
            \end{equation*}
            A quick note: $a^{-1}\notin R$, since if $a^{-1}\in R$, then $a$ is a unit in $R$, so that the principal ideal $\left< a \right>$ \textit{blows up to} $R$, contradicting the fact that $\left< a \right> \subseteq I\neq R$.  
        \end{case}

        \begin{case}
            \textit{$r>1$.}

            By minimality of $r$, $P_2\cdots P_r\nsubseteq\left< a \right>$, so choose
            \begin{equation*}
                b \in P_2\cdots P_r \setminus \left< a \right>. 
            \end{equation*}
            Note that $bP_1\subseteq\left< a \right>$, since, given any $c\in P_1$, $bc \in \left( P_2\cdots P_r \right)P_1 = P_1\cdots P_r\subseteq\left< a \right>$. Then
            \begin{equation}
                bI \subseteq bM = bP_1 \subseteq \left< a \right>. 
            \end{equation}
            Since $b\notin\left< a \right>$, $\lambda = \frac{b}{a}\notin R$. By [3.2], given any $x\in I$, $bx = ar$ for some $r\in R$, so that
            \begin{equation*}
                \lambda x = \frac{b}{a}x = \frac{ar}{a} = r\in R.
            \end{equation*}
        \end{case}
    \end{proof}
    
    \begin{prop}{Invertibility of the Ideals of a Dedekind Domain}
        Let $R$ be a Dedekind domain and let $I$ be an ideal of $R$. Then there exists a nonzero ideal $J\subseteq R$ such that $IJ$ is principal.
    \end{prop}
    
    \begin{proof}
        The case where $I = \left\lbrace 0 \right\rbrace$ or $I=R$ is trivial. Hence suppose $I$ is a nontrivial proper ideal.

        Let $a\in I$ be nonzero. Consider
        \begin{equation*}
            J = \left\lbrace x\in R: xI\subseteq\left< a \right>  \right\rbrace,
        \end{equation*}
        which is a nonzero ideal of $R$ (check this!). Note $IJ\subseteq\left< a \right>$ by definition. 

        Let
        \begin{equation*}
            A = \frac{1}{a}IJ.
        \end{equation*}
        Since $IJ\subseteq\left< a \right>$, it follows $A\subseteq R$. 

        Suppose for contradiction $A\neq R$. Observe that $A$ is a nonzero ideal of $R$ (again, check this!). From Lemma 3.9, \textit{the contradiction getter}, there is $\lambda\in F\setminus R$ such that $\lambda A\subseteq R$. Here $F$ is the field of fractions of $R$. We note two things.
        \begin{enumerate}
            \item \textit{Stupidly}, $J = \frac{1}{a}aJ$. Since $a\in I$ and $A=\frac{1}{a}IJ$, this means $J\subseteq A$, so that
                \begin{equation*}
                    \lambda J\subseteq\lambda A\subseteq R.
                \end{equation*}
            \item Observe that $\lambda A = \frac{\lambda}{a}IJ\subseteq R$. This means $\lambda IJ \subseteq aR = \left< a \right>$. 
        \end{enumerate}
        But by the definition of $J$,
        \begin{equation*}
            J = \left\lbrace x\in R: xI\subseteq\left< a \right>  \right\rbrace,
        \end{equation*}
        it follows $\lambda J\subseteq J$. Say $J$ is generated by $\alpha_1,\ldots,\alpha_m$. Then we may find $B\in R^{m\times m}$ such that
        \begin{equation*}
            \begin{bmatrix} \lambda\alpha_1 \\ \vdots \\ \lambda\alpha_m \end{bmatrix} = B \begin{bmatrix} \alpha_1 \\ \vdots \\ \alpha_m \end{bmatrix}.
        \end{equation*}
        That is, every $\lambda\alpha_j$ can be written as a $R$-linear combination of $\alpha_1,\ldots,\alpha_m$. This means
        \begin{equation*}
            \left( \lambda I-B \right) \begin{bmatrix} \alpha_1 \\ \vdots \\ \alpha_m \end{bmatrix} = 0,
        \end{equation*}
        where at least one of $\alpha_j$ is nonzero as $J = \left< \alpha_1,\ldots,\alpha_m \right>$. Hence
        \begin{equation*}
            \det\left( \lambda I-B \right) = 0.
        \end{equation*}
        This means $\lambda$ is a root of a monic polynomial over $R$, which contradicts the fact that $R$ is integrally closed and $\lambda\notin R$.

        Thus $A=R$, so that
        \begin{equation*}
            IJ = aR = \left< a \right>,
        \end{equation*}
        as required.
    \end{proof}
    
    \begin{cor}{}
        Let $R$ be a Dedekind domain and let
        \begin{equation*}
            X = \left\lbrace I\subseteq R:\text{$I$ is a nonzero ideal of $R$} \right\rbrace.
        \end{equation*}
        Define an equivalence relation $\sim$ on $X$ by
        \begin{equation*}
            I\sim J\iff \exists \alpha,\beta\in R\setminus \left\lbrace 0 \right\rbrace \left[ \alpha I = \beta J \right].
        \end{equation*}
        Then
        \begin{equation*}
            \mG = \left\lbrace \left[ I \right]_{\sim}: I\in X \right\rbrace
        \end{equation*}
        is a group with multiplication
        \begin{equation*}
            \left[ I \right]\left[ J \right] = \left[ IJ \right].
        \end{equation*}
    \end{cor}	

    \begin{proof}
        This follows from Proposition 3.10 and Assignment 2.
    \end{proof}

    \begin{definition}{\textbf{Ideal Class Group} of a Dedekind Domain}
        Consider the setting of Corollary 3.10.1. We call $\mG$ the \emph{ideal class group} of $R$.
    \end{definition}

    \clearpage
    
    \begin{prop}{Cancellation of Ideals of Dedekind Domains}
        Let $R$ be a Dedekind domain and let $A,B,C\subseteq R$ be nontrivial ideals. Then
        \begin{equation*}
            AB = AC \implies B=C.
        \end{equation*}
    \end{prop}

    \vspace{-\preskip}
    
    \begin{proof}
        Let $J$ be a nontrivial ideal of $R$ such that
        \begin{equation*}
            JA = \left< a \right> 
        \end{equation*}
        for some nonzero $a\in A$. Then
        \begin{equation*}
            AB = AC \implies JAB = JAC \implies \left< a \right>B = \left< a \right>C \implies aB = aC \implies B=C,  
        \end{equation*}
        where the last implication uses the fact that $R$ is an integral domain.
    \end{proof}
    
    \begin{definition}{\textbf{Ideal Divisibility}}
        Let $R$ be a ring and let $AB$ be ideals of $R$. We say $A$ \emph{divides} $B$, denoted as $A|B$, if and only if there is an ideal $C$ of $R$ such that $B= AC$.
    \end{definition}
    
    \begin{prop}{Characterization of Ideal Divisibility for Dedekind Domains}
        Let $R$ be a Dedekind domain and let $A,B$ be ideals of $R$. Then
        \begin{equation*}
            A|B \iff B\subseteq A.
        \end{equation*}
    \end{prop}

    \vspace{-\preskip}

    \begin{proof}
        The case involving $\left\lbrace 0 \right\rbrace$ or $R$ is trivial. Hence assume $A,B\neq\left\lbrace 0 \right\rbrace,R$.
        
        ($\implies$) Clearly $B=AC\subseteq A$.

        ($\impliedby$) Suppose $B\subseteq A$. Let $J$ be a nonzero ideal such that $JA = \left< a \right>$ for some $a\in A$. Then $JB \subseteq \left< a \right>$, which means
        \begin{equation*}
            C = \frac{1}{a}JB
        \end{equation*}
        is an ideal of $R$ (again, we can \textit{multiply} by $\frac{1}{a}$ since $JB\subseteq\left< a \right>$). This means
        \begin{equation*}
            JAC = \left< a \right> \frac{1}{a}JB = JB.  
        \end{equation*}
        Using cancellation (Proposition 3.11), we obtain
        \begin{equation*}
            AC = B.
        \end{equation*}
        That is, $A|B$, as required.
    \end{proof}
    
    \np Proposition 3.12 is \textit{nice}, since checking containment is easier than checking divisibility.
    
    \begin{theorem}{Prime Factorization of Ideals of a Dedekind Domain}
        Let $R$ be a Dedekind domain and let $I$ be a proper nontrivial\footnotemark[1] ideal of $R$. Then $I$ can be uniquely\footnotemark[2] written as a product of prime ideals.
        
        \noindent
        \begin{minipage}{\textwidth}
            \footnotetext[1]{"\textit{With $R$ we can never get existence and with $\left\lbrace 0 \right\rbrace$ we can never get uniqueness, so we rule those cases out.}" - Blake}
            \footnotetext[2]{Unique up to reordering.}
        \end{minipage}
    \end{theorem}

    \begin{proof}[Proof of Existence]\qedplacedtrue
        Let $X$ be the set of proper nontrivial ideals of $R$ which cannot be written as a product of prime ideals. For contradiction, $X\neq\emptyset$. Let $I\in X$ be an maximal element of $X$. We know $I$ is not prime, so is not maximal, since $R$ is a Dedekind domain. Let $P$ be a maximal ideal containing $I$. Since $P$ is prime, $I\neq P$. Hence there is a proper ideal $J$ such that $I = PJ$. Then
        \begin{equation*}
            I = PJ \subseteq J.
        \end{equation*}
        If $I = J$, then observe that
        \begin{equation*}
            RJ = RI = I = PJ,
        \end{equation*}
        so by cancelling $J$, we obtain $R=P$, which is a contradiction. Hence $I\neq J$, so that $J\notin X$. This means $J$ is a product of prime ideals, so that $I = PJ$ is also a product of prime ideals, which is a contradiction.

        Thus we conclude $X=\emptyset$, which means every proper nontrivial ideal of $R$ can be written as a product of prime ideals.
    \end{proof}

    \begin{proof}[Proof of Uniqueness]
        Suppose we have two factorizations of a proper nontrivial ideal $I$,
        \begin{equation*}
            I = P_1\cdots P_n = Q_1\cdots Q_m,
        \end{equation*}
        where $P_1,\ldots,P_n,Q_1,\ldots,Q_m$ are prime. This means
        \begin{equation*}
            Q_1\cdots Q_m \subseteq P_1.
        \end{equation*}
        Since $P_1$ is prime, it follows one of $Q_j$'s is contained in $P_1$. Without loss of generality, assume $Q_1\subseteq P_1$. But $Q_1$ is also prime and $R$ is a Dedekind domain, so that $Q_1$ is maximal. This means $P_1 = Q_1$. So by cancellation,
        \begin{equation*}
            P_2\cdots P_n = Q_2\cdots Q_m.
        \end{equation*}
        By induction, we obtain uniqueness.
    \end{proof}

    \np Now that we know prime factorization exists and is unique, our next question is
    \begin{equation*}
        \text{\textit{how do we actually factor an ideal?}}
    \end{equation*}
    This question will be answered in the following two sections.

    \subsection{Ideal Norm}
    
    \begin{definition}{\textbf{Norm} of an Ideal}
        Let $K$ be a number ring and let $R=\mO_K$. If $I$ is a nontrivial ideal of $R$, then we define the \emph{norm} of $I$ as
        \begin{equation*}
            N\left( I \right) = \left| R /I \right|.
        \end{equation*}
    \end{definition}
    
    \np Let's see where definition can be handy. \textit{Assume} that the norm is multiplicative:
    \begin{equation*}
        N\left( IJ \right) = N\left( I \right)N\left( J \right).
    \end{equation*}
    Let $I$ be a nontrivial proper ideal of $R$ and let
    \begin{equation*}
        n = N\left( I \right) = \left| R /I \right|.
    \end{equation*}
    We know that $I$ can be factored into product of prime ideals
    \begin{equation*}
        I = P_1^{n_1}\cdots P_k^{n_k}.
    \end{equation*}
    This means
    \begin{equation}
        N\left( I \right) = N\left( P_1 \right)^{n_1}\cdots N\left( P_k \right)^{n_k}.
    \end{equation}
    Recall that
    \begin{equation*}
        N\left( P_i \right) = \left| R /P_i \right| = p_i^{m_i}
    \end{equation*}
    where $p_i\in P_i$ is prime and $m_i\in\N$. Consequently,
    \begin{equation*}
        n = p_1^{n_1m_1}\cdots p_k^{n_km_k},
    \end{equation*}
    implying that
    \begin{equation*}
        p\in\N\text{ is prime with }p|n \implies p = p_i\text{ for some $i$}.
    \end{equation*}
    But
    \begin{equation*}
        p = p_i\in P_i \implies \left< p \right>\subseteq P_i \implies P_i|\left< p \right>.  
    \end{equation*}
    Hence \textit{if} we can factor each $\left< p_i \right>$, then we can find the candidates for $P_i$'s and hence factor $I$. Also, due to [3.3], $N\left( I \right)$ helps us find $n_i$ as well.

    \np Therefore, here are the goals for this section in order for the above story to work out.
    \begin{formula}{Goals}
        \begin{enumerate}
            \item Prove that ideal norm is multiplicative.
            \item Show $\left< p \right>$ is easily factored for \textit{almost all}\footnotemark[1] prime $p\in\N$. 
        \end{enumerate}
        
        \noindent
        \begin{minipage}{\textwidth}
            \footnotetext[1]{What does \textit{almost all} mean? We shall see this later.}
        \end{minipage}
    \end{formula}

    \np Suppose
    \begin{equation*}
        I = P_1^{n_1}\cdots P_k^{n_k}\subseteq\mO_K
    \end{equation*}
    with $P_i\neq P_j$ for $i\neq j$. Since $P_i$'s are coprime, it follows that
    \begin{equation*}
        R /I \iso R /P_1^{n_1} \times \cdots \times R /P_k^{n_k}
    \end{equation*}
    by the Chinese remainder theorem. Hence
    \begin{equation*}
        N\left( I \right) = N\left( P_1^{n_1} \right)\cdots N\left( P_k^{n_k} \right).
    \end{equation*}
    Hence it suffices to show that
    \begin{equation}
        N\left( P^n \right) = N\left( P \right)^n \text{ for $n\in\N$, prime $P$}.
    \end{equation}

    \np Here are the tools to prove [3.4]:
    \begin{enumerate}
        \item localization;
        \item local rings; and
        \item discrete valuation ring.
    \end{enumerate}
    
    
    
    
    
    
    
    
    
    
    
    
    
    
    
    
    
    
    
    
    
    
    
    
    
    
    
    
    
    
    
    
    
    
    
    
    
    

\end{document}
