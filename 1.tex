\documentclass[pmath441]{subfiles}

%% ========================================================
%% document

\begin{document}

    \section{Algebraic Integers}
    
    \subsection{Motivation}
    
    At its most elementary, number theory is the study of integers. Few topics:
    \begin{itemize}
        \item primes;
        \item integer equations;
        \item divisibility;
        \item gcd; and
        \item prime factorization.
    \end{itemize} 
    The goal is to generalize these topics with \textit{commutative algebra}.

    Naive approach is to use UFD's. A problem with this is that there are many \textit{integer-like} integral domains, such as $\Z\left[ \sqrt{5} \right]$, that are not UFD's.

    Let us do some \textit{random} math and see where it goes. Consider
    \begin{equation*}
        \alpha = \frac{1+\sqrt{5}}{2}.
    \end{equation*}
    Note that $\alpha\in\Q\left[ \sqrt{5} \right]$. In fact, observe that $\alpha$ is the root of the polynomial $x^{2}-x-1$, so that
    \begin{equation}
        \alpha^{2} = \alpha + 1.
    \end{equation}
    
    \begin{definition}{$\Z\left[ \alpha \right]$}
        Given $\alpha\in\CC$, define
        \begin{equation*}
            \Z\left[ \alpha \right] = \left\lbrace f\left( \alpha \right): f\in\Z\left[ x \right] \right\rbrace.
        \end{equation*}
    \end{definition}

    For the specific $\alpha = \frac{1+\sqrt{5}}{2}$, observe that [1.1] tells us that we can replace any $\alpha^{2}$ with a linear polynomial in $\alpha$, so that
    \begin{equation*}
        \Z\left[ \alpha \right] = \left\lbrace a+b\alpha : a,b\in\Z \right\rbrace.
    \end{equation*}

    This simplification worked because
    \begin{equation*}
        \text{\textit{there is a monic $f\in\Z\left[ x \right]$ such that $f\left( \alpha \right) = 0$}}.
    \end{equation*}
    In fact, observe that $\alpha = \frac{1+\sqrt{5}}{2}$ implies that
    \begin{equation*}
        \left( 2\alpha-1 \right)^{2} = 5,
    \end{equation*}
    which means if we have any other number \textit{congruent to $5$ mod $4$} in place of $5$, we would still get a polynomial of the form
    \begin{equation*}
        4\alpha^{2}-4\alpha-b = 0,
    \end{equation*}
    where $b\equiv 0\mod 4$.

    The last thing we note about $\Z\left[ \alpha \right]$ is that
    \begin{equation*}
        \Z\left[ \alpha \right] = \Z+\Z\alpha.
    \end{equation*}
    In general, we want to have
    \begin{equation*}
        \Z\left[ \alpha \right] = \Z+\Z\alpha+\cdots+\Z\alpha^{n-1}
    \end{equation*}
    which allows us to do \textit{$\Z$-module theory}.

    \subsection{Algebraic Integers}

    \begin{definition}{\textbf{Algebraic Integer}}
        We say $\alpha\in\CC$ is an \emph{algebraic integer} if and only if there exists a monic $f\in\Z\left[ x \right]$ such that
        \begin{equation*}
            f\left( \alpha \right) = 0.
        \end{equation*}
    \end{definition}

    We do not insist that $f$ is irreducible. For instance, $7, \sqrt{5}, \frac{1+\sqrt{5}}{2}, i, 1+i, \zeta_n$ are all algebraic integers, where $\zeta_n$ is an $n$th root of unity.

    How do we tell if an \textit{algebraic number} $\alpha\in\CC$ (i.e. $\alpha$ is a root of a not-necessarily monic polynomial over $\Z$) is an algebraic integer?
    
    \begin{theorem}{}
        An algebraic number $\alpha\in\CC$ is an algebraic integer if and only if its minimal polynomial over $\Q$ is over $\Z$.
    \end{theorem}

    \placeqed[Postponed]

    \begin{cor}{}
        The only algebraic integers in $\Q$ are integers.
    \end{cor}	

    \rruleline

    \begin{example}{}
        Consider
        \begin{equation*}
            \beta = \frac{1+\sqrt{3}}{2}.
        \end{equation*}
        Then $\left( 2\beta-1 \right)^{2}=3$, so that $\beta$ is a root for
        \begin{equation*}
            f = x^{2}-x-\frac{1}{2}.
        \end{equation*}
        But $f$ is a monic polynomial with $\deg\left( f \right)=2$ and a root $\beta$ of $f$ is irrational, it follows that $f$ is the minimal polynomial for $\beta$ over $\Q$. Thus $\beta$ is not an algebraic integer.
    \end{example}

    \rruleline

    \np Suppose that
    \begin{equation*}
        f = \sum^{n}_{j=0} a_jx^j\in\Z\left[ x \right].
    \end{equation*}
    Then the \emph{content} of $f$ is
    \begin{equation*}
        \text{content}\left( f \right) = \gcd\left( a_0,\ldots,a_n \right)
    \end{equation*}
    and we say that
    \begin{equation*}
        \text{$f$ is \emph{primitive}} \iff \text{content}\left( f \right) = 1.
    \end{equation*}
    In this setting, Gauss's lemma can be stated as following.

    \begin{lemma}{Gauss's Lemma}
        Let $f,g\in\Z\left[ x \right]$. If $f,g$ are primitive, then so is $fg$.
    \end{lemma}

    \rruleline

    \begin{boxyproof}{Proof of Theorem 1.1}
        ($\impliedby$) This direction is trivial, as any minimal polynomial is monic.

        ($\implies$) Let $\alpha\in\CC$ be an algebraic integer and let $m\in\Q\left[ x \right]$ be its minimal polynomial. Let $f\in\Z\left[ x \right]$ be monic such that $f\left( \alpha \right) = 0$. Since $m$ is the minimal polynomial,
        \begin{equation*}
            f = mg
        \end{equation*}
        for some $g\in\Q\left[ x \right]$. 

        Take $N_1,N_2\in\N$ be the smallest positive integers such that $N_1m, N_2g\in\Z\left[ x \right]$. If $p\in\N$ is a prime dividing all coefficients of $N_1m$, then $\frac{N_1}{p}m\in\Z\left[ x \right]$. In fact, $\frac{N_1}{p}\in\Z$, since $m$ is monic so that the leading coefficient of $N_1m$ is $N_1$. This leads to a contradiction, as $\frac{N_1}{p} < N_1$ violates minimality of $N_1$.

        Also note that $f,m$ are monic, so that $g$ is monic as well. Hence by following a similar argument, $N_2g$ is primitive.

        Now,
        \begin{equation*}
            N_1N_2f = \left( N_1m \right)\left( N_2g \right)
        \end{equation*}
        Since $f$ is monic, observe that the content of $N_1N_2f$ is $N_1N_2$. But $N_1m, N_2g$ are primitive, so by Gauss's lemma, $\left( N_1m \right)\left( N_2g \right)$ is primitive. Therefore
        \begin{equation*}
            N_1N_2 = \text{content}\left( N_1N_2f \right) = \text{content}\left( \left( N_1m \right)\left( N_2g \right) \right) = 1,
        \end{equation*}
        which means $N_1=N_2=1$. Thus $m\in\Z\left[ x \right]$.
    \end{boxyproof}

    \subsection{Ring of Integers}
    
    \begin{example}{}
        Let $d\in\Z$ be \emph{square-free} and $d\neq 1$. That is, in the prime factorization of $d$, there are no multiplicities. Consider
        \begin{equation*}
            K = \Q\left( \sqrt{d} \right) = \left\lbrace a+b\sqrt{d}: a,b\in\Q \right\rbrace.
        \end{equation*}
        Then we know that
        \begin{equation*}
            K /\Q\text{ is finite}\implies K /\Q\text{ is algebraic}.
        \end{equation*}

        We are going to find all algebraic integers in $K$. Let
        \begin{equation*}
            \alpha = a+b\sqrt{d}\in K
        \end{equation*}
        be an algebraic integer. Consider the conjugate
        \begin{equation*}
            \overline{\alpha} = a-b\sqrt{d}.
        \end{equation*}
        Then
        \begin{equation*}
            m = \left( x-\alpha \right)\left( x-\overline{\alpha} \right) = x^{2}-2ax+a^{2}-db^{2}
        \end{equation*}
        is the minimal polynomial for $\alpha$ over $\Q$. By Theorem 1.1, it follows that $2a,a^{2}-db^{2}\in\Z$. Now,
        \begin{equation*}
            4\left( a^{2}-db^{2} \right) = \left( 2a \right)^{2} - d\left( 2b \right)^{2}
        \end{equation*}
        but $a^{2}-db^{2}, \left( 2a \right)^{2}\in\Z$, so that
        \begin{equation*}
            d\left( 2b \right)^{2}\in\Z.
        \end{equation*}
        Since $d$ is square-free, it follows that $2b\in\Z$. If not, then the denominator of $2b$ is not $1$. This means the denominator of $\left( 2b \right)^{2}$ has a square of a prime as a factor, which contradicts the fact that $d$ is square-free. Hence $\gamma = 2a, \delta = 2b\in\Z$. This means
        \begin{equation*}
            a^{2}-db^{2} = \left( \frac{\gamma}{2} \right)^{2} - d\left( \frac{\delta}{2} \right)^{2} = \frac{\gamma^{2}-d\delta^{2}}{4}\in\Z.
        \end{equation*}
        It follows $\gamma^{2}-d\delta^{2} \equiv 0 \mod 4$.

        We have few cases.

        \begin{case}
            $d\equiv 1\mod 4$.

            It follows that
            \begin{equation*}
                \gamma^{2} \equiv \delta^{2} \mod 4.
            \end{equation*}
            But even numbers square to $0$ mod $4$ and odd numbers square to $1$ mod $4$. Hence
            \begin{equation*}
                \gamma \equiv \delta \mod 2.
            \end{equation*}
            It follows that $\alpha$ is of the form
            \begin{equation*}
                \alpha = a+b\sqrt{d} = \frac{\gamma+\delta\sqrt{d}}{2}
            \end{equation*}
            for some $\gamma,\delta\in\Z$.
        \end{case}

        \begin{case}
            $d\equiv 2\mod 4$ or $d\equiv 3\mod 4$.

            It is a routine exercise to show that
            \begin{equation*}
                \gamma^{2}-d\delta^{2} \equiv 0 \mod 4 \iff \gamma \equiv \delta \equiv 0 \mod 2.
            \end{equation*}
            Hence
            \begin{equation*}
                \alpha = \frac{\gamma}{2} + \frac{\delta}{2} \sqrt{d}
            \end{equation*}
            but $\gamma,\delta$ are even numbers, so that $a=\frac{\gamma}{2},b=\frac{\delta}{2}\in\Z$ and
            \begin{equation*}
                \alpha = a+b\sqrt{d}.
            \end{equation*}
        \end{case}

        Exercise: check these conditions are also sufficient.
    \end{example}

    \rruleline
    
    \np The above example gives the following idea.
    \begin{equation*}
        \text{\textit{Given a finite extension $K /\Q$, we find all algebraic integers in $K$.}}
    \end{equation*}
    This motivates the following definitions.

    \begin{definition}{\textbf{Number Field}, \textbf{Ring of Integers} of a Number Field}
        We call a finite extension $K$ of $\Q$ a \emph{number field}.

        Given a number field $K$, we call
        \begin{equation*}
            \mO_K = \left\lbrace \alpha\in K: \alpha \text{ is an algebraic integer} \right\rbrace
        \end{equation*}
        the \emph{ring of integers} of $K$.
    \end{definition}
    
    \np We are going to prive that $\mO_K$ is a ring.\footnote{We are going to assume that every ring is unital and commutative throughout, if not stated otherwise.} To do so, we first show
    \begin{equation*}
        \A = \left\lbrace z\in\CC: z\text{ is an algebraic integer} \right\rbrace
    \end{equation*}
    is a ring, so that
    \begin{equation*}
        \mO_K = \A\cap K
    \end{equation*}
    is also a ring.
    
    \np Recall that, given $\alpha\in\A$, we have
    \begin{equation*}
        \Z\left[ \alpha \right] = \Z + \Z\alpha + \cdots + \Z\alpha^{n-1}.
    \end{equation*}
    This allows us to do module theory over $\Z$.

    \begin{definition}{\textbf{$R$-module}}
        Let $R$ be a ring. An \emph{$R$-module} is an abelian group $\left( M,+ \right)$ with a left $R$-action on $M$ such that
        \begin{enumerate}
            \item $1m = m$ for $m\in M$;
            \item $\left( r_1+r_2 \right)m = r_1m+r_2m$ for $r_1,r_2\in R, m\in M$;
            \item $r\left( m_1+m_2 \right) = rm_1+rm_2$ for $r\in R, m_1,m_2\in M$; and
            \item $\left( r_1r_2 \right)m = r_1\left( r_2m \right)$ for $r_1,r_2\in R, m\in M$.
        \end{enumerate}
    \end{definition}

    \clearpage
    
    \begin{example}{Examples of $R$-modules}
        Given a ring $R$, $R$ is an $R$-module with left action
        \begin{equation*}
            r\cdot m = rm, \hspace{2cm}\forall r,m\in R.
        \end{equation*}
        In fact, given any subring $S\subseteq\R$, $R$ is an $S$-module with
        \begin{equation*}
            s\cdot r = sr, \hspace{2cm}\forall s\in S, r\in R.
        \end{equation*}
        Similar to $\R^n$ which is a $\R$-vector space, $R^n$ is an $R$-module with
        \begin{equation*}
            r\begin{bmatrix} x_1 \\ \vdots \\ x_n \end{bmatrix} = \begin{bmatrix} rx_1 \\ \vdots \\ rx_n \end{bmatrix}, \hspace{2cm}\forall r\in R, \begin{bmatrix} x_1 \\ \vdots \\ x_n \end{bmatrix}\in R^n.
        \end{equation*}
    \end{example}

    \rruleline

    \begin{example}{}
        Consider $R=\Z$ and consider an $R$-module $M$. Then given $n\in\N,m\in M$,
        \begin{equation*}
            n\cdot m = \left( 1+\cdots+1 \right)\cdot m = 1\cdot m + \cdots + 1\cdot m = m + \cdot + m = nm.
        \end{equation*}
        That is, the $\Z$-module on an abelian group $M$ \textit{does not impose any additional structure on $M$}; a $\Z$-module is simply an abelian group.

        As an exercise, we can also check that
        \begin{equation*}
            \left( -n \right)\cdot m = -nm
        \end{equation*}
        for $n\in\N, m\in M$.
    \end{example}

    \rruleline
    
    \begin{definition}{\textbf{$R$-submodule}, \textbf{Homomorphism} of $R$-modules, \textbf{Finitely Generated} $R$-module}
        Let $R$ be a ring and let $M$ be an $R$-module. We say $N\subseteq M$ is an $R$-submodule of $M$ if $N$ is an $R$-module using the same operations as $M$.

        Given $R$-modules $M,N$, we say $f:M\to N$ is a \emph{homomorphism} if and only if
        \begin{equation*}
            f\left( rm_1+m_2 \right) = rf\left( m_1 \right) + f\left( m_2 \right), \hspace{2cm}\forall r\in R, m_1,m_2\in M.
        \end{equation*}
        In case $f$ is bijective, we say $f$ is an \emph{isomorphism}.

        We say an $R$-module is \emph{finitely generated} if there are $m_1,\ldots,m_n\in M$ such that
        \begin{equation*}
            M = Rm_1 + \cdots + Rm_n.
        \end{equation*}
        That is, for any $m\in M$, there exists $r_1,\ldots,r_n\in R$ such that
        \begin{equation*}
            m = \sum^{n}_{j=1}r_jm_j.
        \end{equation*}
        In other words, finite number of elements $m_1,\ldots,m_n$ \emph{generate} $M$.
    \end{definition}

    \np Observe that
    \begin{equation*}
        \text{$N\subseteq M$ is an $R$-submodule} \iff \text{$N$ is subgroup of $M$ closed under $R$-left action}.
    \end{equation*}
    
    \begin{example}{}
        Given a ring $R$, as an $R$-module, the only $R$-submodules are the ideals of $R$.
    \end{example}

    \rruleline

    \begin{definition}{\textbf{Integral} over $R$}
        Let $R,S$ be integral domains, such that $R$ is a subring of $S$. We say $\alpha\in S$ is \emph{integral} over $R$ if there is monic $f\in R\left[ x \right]$ such that $f\left( \alpha \right) = 0$.
    \end{definition}

    \begin{example}{}
        In case $R=\Z, S=\CC$, given $\alpha\in S$,
        \begin{equation*}
            \text{$\alpha$ is integral} \iff \text{$\alpha$ is algebraic integer}.
        \end{equation*}
        That is, being integral over $R$ is a generalization of being an algebraic integer.
    \end{example}

    \rruleline
    
    \begin{theorem}{}
        Let $R,S$ be integral domains where $R$ is a subring of $S$ and let $\alpha\in S$. Then
        \begin{equation*}
            \text{$\alpha$ is integral over $R$} \iff R\left[ \alpha \right] = \left\lbrace f\left( \alpha \right):f\in R\left[ x \right] \right\rbrace \text{ is a finitely generated $R$-module}.
        \end{equation*}
    \end{theorem}
    
    \begin{proof}
        ($\implies$) Suppose $\alpha$ is integral over $R$. Then there is a polynomial relation
        \begin{equation*}
            \alpha^n + a_{n-1}\alpha^{n-1} + \cdots + a_1\alpha + a_0 = 0
        \end{equation*}
        for some $a_0,\ldots,a_{n-1}\in R$. Rearranging for $\alpha^n$,
        \begin{equation*}
            \alpha^n = - \left( a_{n-1}\alpha^{n-1} + \cdots + a_1\alpha + a_0 \right).
        \end{equation*}
        This means, given any $f\in R\left[ x \right]$, every powers $\alpha^n,\alpha^{n+1},\ldots$ in $f\left( \alpha \right)$ can be replaced by lower powers of $\alpha$, so that
        \begin{equation*}
            f\left( \alpha \right) = g\left( \alpha \right)
        \end{equation*}
        for some $g\in R\left[ x \right]$ with $\deg\left( g \right) \leq n-1$. That is,
        \begin{equation*}
            R\left[ \alpha \right]\subseteq R+R\alpha+\cdots+R\alpha^{n-1}.
        \end{equation*}
        But the reverse containment is trivial, so that $R\left[ \alpha \right]$ is finitely generated.

        ($\impliedby$) Suppose $R\left[ \alpha \right]$ is finitely generated, say
        \begin{equation*}
            R\left[ \alpha \right] = Rf_1\left( \alpha \right) + \cdots + Rf_n\left( \alpha \right)
        \end{equation*}
        with $f_1,\ldots,f_n\in R\left[ x \right]$. Take $N = \max_{1\leq j\leq n} \deg\left( f_j \right)$. Then $\alpha^{N+1}\in R\left[ x \right]$ as a polynomial of $\alpha$, so that
        \begin{equation*}
            \alpha^{N+1} = \sum^{n}_{j=1} r_jf_j\left( \alpha \right)
        \end{equation*}
        for some $r_1,\ldots,r_n\in R$. 

        Now consider
        \begin{equation*}
            g = x^{N+1} - \sum^{n}_{j=1}r_jf_j \in R\left[ x \right].
        \end{equation*}
        Then $g\left( \alpha \right) = 0$. But $\deg\left( x^{N+1} \right) = N+1 > N = \max_{1\leq j\leq n} \deg\left( f_j \right)$, so that $g$ is monic as well. Thus $\alpha$ is algebraic over $R$.
    \end{proof}
    
    \np The big idea for Theorem 1.3 is that
    \begin{equation*}
        \text{\textit{showing $\Z\left[ \alpha \right]$ is finitely generated is often \textbf{easier} than finding monic $f\in\Z\left[ x \right]$ with $f\left( \alpha \right) = 0$.}}
    \end{equation*}
    "Let's work with generators instead of polynomials" - Blake.
    
    \begin{theorem}{}
        Let
        \begin{equation*}
            \A = \left\lbrace z\in\CC: z\text{ is an algebraic integer} \right\rbrace.
        \end{equation*}
        Then $\A$ is a subring of $\CC$.
    \end{theorem}

    \begin{proof}[Proof Attempt]
        If we are in PMATH 348, proving something is \textit{easy}; we simply apply the subring test. Let's see how it fails here.

        Let $\alpha,\beta\in\A$. We must show that $\alpha-\beta, \alpha\beta\in\A$. That is, we must show
        \begin{equation*}
            \Z\left[ \alpha-\beta \right],\Z\left[ \alpha\beta \right]\text{ are finitely generated $\Z$-modules}.
        \end{equation*}
        Since $\alpha,\beta$ are algebraic integers, write
        \begin{equation*}
            \Z\left[ \alpha \right] = \sum^{n}_{j=1} \Z\alpha_j, \hspace{0.5cm} \Z\left[ \beta \right] = \sum^{m}_{j=1} \Z\beta_j.
        \end{equation*}
        Therefore,
        \begin{equation*}
            \Z\left[ \alpha,\beta \right] = \left\lbrace f\left( \alpha,\beta \right): \Z\left[ x,y \right] \right\rbrace
        \end{equation*}
        is also finitely generated. In fact, it is generated by $\left\lbrace \alpha_i\beta_j \right\rbrace^{}_{1\leq i\leq n, 1\leq j\leq m}$. Hence $\Z\left[ \alpha,\beta \right]$ is finitely generated as a $\Z$-module.

        We have that $\Z\left[ \alpha-\beta \right], \Z\left[ \alpha\beta \right]$ are $\Z$-submodules of the $fg$ module $\Z\left[ \alpha,\beta \right]$.

        Now, if we use the intuition from linear algebra, we should be done here. Recall that subspaces of a finite-dimensional vector space are finite-dimensional. But this is not the case for modules!

        \placeqed[Proof Failed]
    \end{proof}

    \begin{example}{Submodule of a Finitely Generated Module That Is Not Finitely Generated}
        Consider
        \begin{equation*}
            R = \left[ x_1,x_2,\ldots \right].
        \end{equation*}
        Then $R$ is a finitely generated $R$-module (i.e. $R = R 1$). But observe that
        \begin{equation*}
            I = \left< x_1,x_2,\ldots \right> 
        \end{equation*}
        is not finitely generated. 

        To see this, observe that elements of $R$ are polynomials in $x_1,x_2,\ldots$, which has \textit{only finitely many indeterminates}. So having finitely many polynomials does not give enough number of indeterminates to generate $I$.
    \end{example}

    \rruleline

    \np To resolve this issue, we consider the following definition.
    
    \begin{definition}{\textbf{Noetherian} Ring}
        Let $R$ be a ring. We say $R$ is \emph{Noetherian} if every $R$-submodule (i.e. ideal) of $R$ is finitely generated.
    \end{definition}

    \begin{example}{}
        Observe that $\Z$ is Noetherian, as it is a PID (i.e. every ideal of $\Z$ is generated by \textit{an} element).
    \end{example}

    \rruleline

    \begin{theorem}{}
        Let $R$ be a Noetherian ring and let $M$ be a finitely generated $R$-module. Then every $R$-submodule of $M$ is finitely generated.
    \end{theorem}

    \rruleline

    \np Theorem 1.5 resolves the issue we left in Theorem 1.4, since $\Z$ is Noetherian.

    \np Let us reduce Theorem 1.5 a bit. Consider a finitely generated $R$-module
    \begin{equation*}
        M = R\alpha_1 + \cdots + R\alpha_n
    \end{equation*}
    and an epimorphism of $R$-modules
    \begin{equation*}
        \begin{aligned}
            f:R^n & \to M \\
            \left( r_1,\ldots,r_n \right) & \mapsto r_1\alpha_1 + \cdots + r_n\alpha_n
        \end{aligned} .
    \end{equation*}
    That is, every finitely generated $R$-module can be viewed as an $R$-submodule of $R^n$.

    Moreover, for any $R$-submodule $N\subseteq M$,
    \begin{equation*}
        f^{-1}\left( N \right)\subseteq R^n.
    \end{equation*}
    If $f^{-1}\left( N \right) = R\beta_1 + \cdots + R\beta_m$, then
    \begin{equation*}
        N = Rf\left( \beta_1 \right) + \cdots + Rf\left( \beta_m \right).
    \end{equation*}
    Hence it remains to show that every $R$-submodule $N$ of $M$ satisfy $f^{-1}\left( N \right) = R\beta_1+\cdots+R\beta_m$ for some $\beta_1,\ldots,\beta_m\in R$.

    \begin{boxyproof}{Proof of Theorem 1.5}
        We may assume $M=R^n$. If $n=1$, then $R$ is Noetherian and we are done.

        Suppose that the result holds for some $n\geq 1$ and consider $M = R^{n+1}$. Consider the epimorphism
        \begin{equation*}
            \begin{aligned}
                \pi:R^{n+1}&\to R \\
                \left( r_1,\ldots,r_{n+1} \right) & \mapsto r_{n+1}
            \end{aligned} .
        \end{equation*}
        Let $N$ be an $R$-submodule of $M$. Consider
        \begin{equation*}
            N_1 = \left\lbrace \left( r_1,\ldots,r_{n+1} \right)\in N: r_{n+1} = 0 \right\rbrace
        \end{equation*}
        which is isomorphic to an $R$-submodule of $R^n$. Hence by inductive hypothesis, $N_1$ is finitely generated. Moreover,
        \begin{equation*}
            N_2 = \pi\left( N \right)
        \end{equation*}
        is an $R$-submodule of $R$, so is finitely generated (by inductive hypothesis).

        Say
        \begin{equation*}
            \begin{aligned}
                N_1 & = Rx_1 + \cdots + Rx_p \\
                N_2 & = R\pi\left( y_1 \right) + \cdots + R\pi\left( y_q \right)
            \end{aligned} 
        \end{equation*}
        for some $x_1,\ldots,x_p,y_1,\ldots,y_q\in R$. Let $x\in N$. Then
        \begin{equation*}
            \pi\left( x \right) = r_1\pi\left( y_1 \right) + \cdots + r_q\pi\left( y_q \right)
        \end{equation*}
        for some $r_1,\ldots,r_q\in R$. But $\pi$ is a homomorphism of $R$-modules, so that
        \begin{equation*}
            \pi\left( x - \sum^{q}_{j=1} r_jy_j \right) = 0.
        \end{equation*}
        This means the $\left( n+1 \right)$th entry of $x-\sum^{q}_{j=1}r_jy_j$ is $0$, so that $x-\sum^{q}_{j=1}r_jy_j\in N_1$. That is,
        \begin{equation*}
            x - \sum^{q}_{j=1}r_jy_j = \sum^{p}_{k=1} s_kx_k
        \end{equation*}
        for some $s_1,\ldots,s_p\in R$.

        Thus
        \begin{equation*}
            x = \sum^{q}_{j=1}r_jy_j + \sum^{p}_{k=1} s_kx_k,
        \end{equation*}
        so that
        \begin{equation*}
            N = \sum^{q}_{j=1}Ry_j + \sum^{p}_{k=1} Rx_k,
        \end{equation*}
        as required.
    \end{boxyproof}
    
    \subsection{Additive Structure}

    So far, it has been very useful to consider $\mO_K$ as a $\Z$-module. Let us investigate this $\Z$-module as an abelian group
    \begin{equation*}
        \left( \mO_K,+ \right)
    \end{equation*}
    \textit{without multiplication structure}, where $K$ is a number ring (i.e. $\left[ K:\Q \right]<\infty$).

    The next definition will make it clear the kind of \textit{linear algebraic} approach we are going to take.

    \begin{definition}{\textbf{Linearly Independent} Subset of an $R$-module, \textbf{Basis} for an $R$-module, \textbf{Free} $R$-module}
        Let $R$ be a ring and let $M$ be an $R$-module. Let $B\subseteq M$.
        \begin{enumerate}
            \item Say $B$ is \emph{linearly independent} if and only if for all $m_1,\ldots,m_n\in B, r_1,\ldots,r_n\in R$,
                \begin{equation*}
                    r_1m_1 + \cdots + r_nm_n = 0 \implies r_1 = \cdots = r_n = 0.
                \end{equation*}
            \item Say $B$ \emph{spans} $M$ if for all $x\in M$, there are $b_1,\ldots,b_n\in B, r_1,\ldots,r_n\in R$ such that
                \begin{equation*}
                    x = r_1b_1 + \cdots + r_nb_n.
                \end{equation*}
            \item Say $B$ is a \emph{basis} for $M$ if $B$ is linearly independent and spans $M$. In case $M$ admits a basis, we call it a \emph{free} $R$-module.
        \end{enumerate}
    \end{definition}

    \np In case there is a basis $B$ for $M$, the size of any other basis for $M$ is $\left| B \right|$.

    \begin{definition}{\textbf{Rank} of a Free $R$-module}
        Let $R$ be a ring and let $M$ be a free $R$-module. Then the size of a basis for $M$ is called the \emph{rank} of $M$, denoted as $\rank\left( M \right)$.
    \end{definition}
    
    \begin{prop}{}
        Let $R$ be a ring and let $M$ be an $R$-module. Let $B\subseteq M$. Then
        \begin{equation*}
            \text{$B$ is a basis} \iff \text{every $x\in M$ can be uniquely written as an $R$-linear combination of elements of $B$}.
        \end{equation*}
        In particular,
        \begin{equation*}
            \text{$M$ is free with }\rank\left( M \right) = n < \infty \iff M\iso R^n \text{ by } \left( r_1,\ldots,r_n \right)\leftrightarrow r_1b_1+\cdots+r_nb_n \text{ for some $b_1,\ldots,b_n\in M$},
        \end{equation*}
        in which case $\left\lbrace b_1,\ldots,b_n \right\rbrace$ is a basis for $M$.
    \end{prop}

    \rruleline

    \begin{example}{Free but not Finitely Generated}
        Consider $R=\Z,M=\Z\left[ x \right],B=\left\lbrace 1,x,x_2,\ldots \right\rbrace$. Then $M$ is a free module generated by $B$ but is not finitely generated.
    \end{example}

    \rruleline
    
    \begin{example}{Finitely Generated but not Free}
        Consider $R=\Z, M=\Z_2$. Then $2\cdot 1 =0$ but $2\neq 0$ in $R$. So the only $R$-linearly independent subset of $M$ is the emptyset $\emptyset$, so that $M$ is fintely generated but not free.
    \end{example}

    \rruleline
    
    \begin{example}{}
        Consider $R=\Z, M=\Z\times\Z, N=\Z\times 2\Z$. Then $M$ is free with a basis
        \begin{equation*}
            B_1 = \left\lbrace \left( 1,0 \right),\left( 0,1 \right) \right\rbrace,
        \end{equation*}
        so that $\rank\left( M \right) = 2$. Also, $N$ is free with a basis
        \begin{equation*}
            B_2 = \left\lbrace \left( 1,0 \right),\left( 0,2 \right) \right\rbrace,
        \end{equation*}
        so that $\rank\left( N \right) = 2$. However, observe that $B_2$ is an $R$-linearly independent subset of $M$ with $\rank\left( M \right)$ elements!

        This particular example shows that it is possible for modules of rank $n$ to have a linearly independent subset of $n$ elements which does not span the whole module, unlike the case in linear algebra.
    \end{example}

    \rruleline

    \clearpage
    
    \np We are going to present two facts without proof. Fix a PID $R$ and a free $R$-module $M$ with $\rank\left( M \right) = n<\infty$.

    \begin{fact}{}
        For an $R$-submodule $N\subseteq M$, $N$ is free with $\rank\left( N \right)\leq n$.
    \end{fact}

    \begin{fact}{}
        Any maximal linearly independent subset of $M$ has $n$ elements.
    \end{fact}

    \np The next goal is to show that ring of integers is a free module. That is, given a number field $K$ with $\left[ K:\Q \right] = n$, our goal is 
    \begin{equation*}
        \text{\textit{to find an embedding (i.e. monomorphism) $\phi:\mO_K\to\Z^n$ such that $\rank\left( \phi\left( \mO_K \right) \right) = n$.}}
    \end{equation*}
    This will tell us $\mO_k\iso\Z^n$ as $\Z$-modules. In particular, $\left( \mO_K,+ \right)$ is a free module with rank $n$.
    
    \begin{definition}{\textbf{Integral Basis}}
        Given a free $\Z$-module $M$, a basis for $M$ is called an \emph{integral basis}.
    \end{definition}

    \np We introduce two useful tools in algebraic number theory.

    \begin{definition}{\textbf{Trace}, \textbf{Norm} of an Element of a Number Field}
        Let $K$ be a number field with $\left[ K:\Q \right]=n<\infty$. Let $\alpha\in K$ and consider
        \begin{equation*}
            \begin{aligned}
                T_{\alpha}:K&\to K \\
                x&\mapsto \alpha x
            \end{aligned} ,
        \end{equation*}
        which is a $\Q$-linear operator.
        \begin{enumerate}
            \item The \emph{trace} of $\alpha$ relative to $K /\Q$, denoted as $\tr_{K /\Q}\left( \alpha \right)$, is
                \begin{equation*}
                    \tr_{K /\Q}\left( \alpha \right) = \tr\left( T_{\alpha} \right).
                \end{equation*}
            \item The \emph{norm} of $\alpha$ relative to $K /\Q$, denoted as $N_{K /\Q}\left( \alpha \right)$, is
                \begin{equation*}
                    N_{K /\Q}\left( \alpha \right) = \det\left( T_{\alpha} \right).
                \end{equation*}
        \end{enumerate}
    \end{definition}
    
    \np Note that $\tr_{K /\Q}\left( \alpha \right), N_{K /\Q}\left( \alpha \right)\in\Q$, since $T_{\alpha}$ is a $\Q$-linear operator (hence the entries of any matrix representation of $T_{\alpha}$ are rational).

    \np Let $\alpha\in K$. Let $\beta$ be a $\Q$-basis for $K$ and let $A = \left[ T_{\alpha} \right]_{\beta}$. Consider the characteristic and minimal polynomials $f,p\in\Q\left[ x \right]$, respectively, of $A$. Notice that, for $g\in\Q\left[ x \right]$ and $v\in K$,
    \begin{equation*}
        g\left( T_{\alpha} \right)v = g\left( \alpha \right)v,
    \end{equation*}
    since $T_{\alpha}^mv = \alpha^mv$ for $m\in\N\cup\left\lbrace 0 \right\rbrace$. In particular,
    \begin{equation*}
        g\left( \alpha \right) = 0 \iff g\left( T_{\alpha} \right) = 0,
    \end{equation*}
    so that $p$ is the minimal polynomial for $\alpha$ over $\Q$. By the Cayley-Hamilton theorem, $p|f$. However,
    \begin{equation*}
        \deg\left( f \right) = \left[ K:\Q \right] = n.
    \end{equation*}
    We consider the following particular case.

    \clearpage

    \begin{case}
        \textit{Suppose}
        \begin{equation*}
            K = \Q\left( \alpha \right).
        \end{equation*}
        On the other hand, since $p$ is the minimal polynomial of $\alpha$,
        \begin{equation*}
            \deg\left( p \right) = \left[ \Q\left( \alpha \right):\Q \right] = \left[ K:\Q \right] = n.
        \end{equation*}
        Hence $p|f$, $\deg\left( f \right) = \deg\left( p \right)$, and $f,p$ are monic, so that $f=p$.

        Let $\alpha = \alpha_1, \alpha_2, \ldots, \alpha_n$ be the conjugates of $\alpha$ (i.e. the roots of $p$ in $\CC$). But the roots of the characteristic polynomial of an operator are the eigenvalues (with multiplicity) and $f=p$, so that
        \begin{equation*}
            \tr_{K /\Q}\left( \alpha \right) = \tr\left( T_{\alpha} \right) = \sum^{n}_{j=1} \alpha_j
        \end{equation*}
        and
        \begin{equation*}
            N_{K /\Q}\left( \alpha \right) = \det\left( T_{\alpha} \right) = \prod^{n}_{j=1} \alpha_j.
        \end{equation*}

        Also note that
        \begin{equation*}
            \sum^{n}_{j=1} \alpha_j = -\left[ x^{n-1} \right]p 
        \end{equation*}
        and
        \begin{equation*}
            \left( -1 \right)\left[ x^0 \right]p = \left( -1 \right)^np\left( 0 \right).
        \end{equation*}

        Recall from the field theory that the embeddings of $K=\Q\left( \alpha \right)$ in $\CC$ are exactly given by $\sigma_j\left( \alpha \right) = \alpha_j$ for $j\in\left\lbrace 1,\ldots,n \right\rbrace$. That is,
        \begin{equation*}
            \tr_{K /\Q}\left( \alpha \right) = \sum^{n}_{j=1}\alpha_j = \sum^{n}_{j=1} \sigma_j\left( \alpha \right)
        \end{equation*}
        and
        \begin{equation*}
            N_{K /\Q}\left( \alpha \right) = \prod^{n}_{j=1}\alpha_j = \sum^{n}_{j=1}\sigma_j\left( \alpha \right).
        \end{equation*}
    \end{case}
    
    \np Apart from Case 1, we want to compute $\tr_{K /\Q}\left( \alpha \right), N_{K /\Q}\left( \alpha \right)$ \textit{in general}. To do so, we introduce the following lemma with a technical proof.

    \begin{lemma}{}
        Suppose that $K$ is a number field with $\left[ K:\Q \right] = n$ and let $\alpha\in K$ with $\left[ K:\Q\left( \alpha \right) \right] = m$. Consider
        \begin{equation*}
            \begin{aligned}
                T_{\alpha}:K&\to K \\
                x & \mapsto \alpha x
            \end{aligned} .
        \end{equation*}
        Let $f\in\Q\left[ x \right]$ be the characteristic polynomial of $T_{\alpha}$ and let $p\in\Q\left[ x \right]$ be the minimal polynomial for $\alpha$. Then
        \begin{equation*}
            f = p^m.
        \end{equation*}
    \end{lemma}

    \np Note that we recover Case 1 when $m=1$ (i.e. $K=\Q\left( \alpha \right)$).
    
    \begin{proof}
        Let
        \begin{equation*}
            \beta = \left\lbrace y_1,\ldots,y_d \right\rbrace
        \end{equation*}
        be a $\Q$-basis for $\Q\left( \alpha \right)$ and let
        \begin{equation*}
            \beta' = \left\lbrace z_1,\ldots,z_m \right\rbrace
        \end{equation*}
        be a $\Q\left( \alpha \right)$-basis for $K$. By the tower theorem, we have that
        \begin{equation*}
            \left\lbrace y_jz_k \right\rbrace_{1\leq j\leq d, 1\leq k\leq m}
        \end{equation*}
        is a $\Q$-basis for $K$.

        Let $A = \left[ T_{\alpha} \right]_{\beta}\in\Q^{d\times d}$ (where we consider the restriction $T_{\alpha}:\Q\left( \alpha \right)\to\Q\left( \alpha \right)$). Recall from linear algebra that
        \begin{equation*}
            \alpha y_j = T_{\alpha}\left( y_j \right) = \left(A\left[ y_j \right]_{\beta}\right)^{T} 
            \begin{bmatrix} y_1 & \cdots & y_d^{T} \end{bmatrix}
            = \left( Ae_j \right)^{T}
            \begin{bmatrix} y_1 & \cdots & y_d^{T} \end{bmatrix}
            = \sum^{d}_{k=1} a_{k,i} y_k,
        \end{equation*}
        where $A = \left[ a_{k,i} \right]^{d}_{k,i=1}$. This implies
        \begin{equation}
            \alpha y_iz_j = \sum^{d}_{k=1} a_{ki}y_kz_j.
        \end{equation}
        Consider the ordered basis
        \begin{equation*}
            \gamma = \left( y_1z_1,\ldots,y_dz_1,y_1z_2,\ldots,y_dz_2,\ldots,y_1z_m,\ldots,y_dz_m \right).
        \end{equation*}
        Then [1.2] gives (exercise)
        \begin{equation*}
            \left[ T_{\alpha} \right]_{\gamma} =
            \begin{bmatrix}
            	A &  &  &  \\
            	 & A &  &  \\
            	 &  & \ddots &  \\
            	 &  &  & A \\
            \end{bmatrix}.
        \end{equation*}
        Immediately,
        \begin{equation*}
            f = \det\left( xI-A \right)^m = p^m,
        \end{equation*}
        where the last equality follows from Case 1.
    \end{proof}
    
    \np Consider the setting of Lemma 1.9. Observe that
    \begin{equation*}
        \tr_{K /\Q}\left( \alpha \right) = \tr\left( T_{\alpha} \right) = \sum^{}_{j} \lambda_j,
    \end{equation*}
    where $\lambda_j$'s are the eigenvalues of $T_{\alpha}$. But $f$ is the characteristic polynomial for $T_{\alpha}$ and $f=p^m$, so that
    \begin{equation*}
        \tr_{K /\Q}\left( \alpha \right) = m\sum^{\frac{m}{n}}_{j=1}\alpha_j.
    \end{equation*}
    Similarly,
    \begin{equation*}
        N_{K /\Q}\left( \alpha \right) = \left( \alpha_1\cdots\alpha_{\frac{n}{m}} \right)^m.
    \end{equation*}
    
    \np The embeddings of $\Q\left( \alpha \right)$ in $\CC$ are determined by $\sigma_j\left( \alpha \right)=\alpha_j$ for $j\in\left\lbrace 1,\ldots,\frac{n}{m} \right\rbrace$. By Assignment 1, each $\sigma_j$ extends to exactly $m$ embeddings of $K$ in $\CC$. If $\rho_1,\ldots,\rho_n$ are the embeddings of $K$ in $\CC$, them
    \begin{equation*}
        \tr_{K /Q}\left( \alpha \right) = m\sum^{n}_{j=1} \sigma_j\left( \alpha \right) = \sum^{n}_{j=1}\rho_n\left( \alpha \right).
    \end{equation*}
    Similarly,
    \begin{equation*}
        N_{K /\Q}\left( \alpha \right) = \prod^{n}_{j=1} \rho_j\left( \alpha \right).
    \end{equation*}
    
    \clearpage

    \np Let $K$ be a number field with $\left[ K:\Q \right]=n$ and let $\alpha,\beta\in K, q\in\Q$. Then
    \begin{equation*}
        \tr_{K /\Q}\left( q\alpha+\beta \right) = \sum^{n}_{j=1}\sigma_j\left( q\alpha+\beta \right) = q\sum^{n}_{j=1}\sigma_j\left( \alpha \right) + \sum^{n}_{j=1}\sigma_j\left( \beta \right) = q\tr_{K /\Q}\left( \alpha \right) + \tr_{K /\Q}\left( \beta \right).
    \end{equation*}
    That is, $\tr_{K /\Q}$ is a linear map.

    On the other hand,
    \begin{equation*}
        N_{K /\Q}\left( q\alpha\beta \right) = \prod^{n}_{j=1} \sigma_j\left( q\alpha\beta \right) = \prod^{n}_{j=1} q\sigma_j\left( \alpha \right)\sigma_j\left( \beta \right) = q^nN_{K /\Q}\left( \alpha \right)N_{K /\Q}\left( \beta \right).
    \end{equation*}
    
    Now suppose $\alpha\in\mO_K$. Then
    \begin{equation*}
        \tr_{K /\Q}\left( \alpha \right) = \sum^{n}_{j=1} \sigma_j\left( \alpha \right).
    \end{equation*}
    If $\alpha$ is the root of a monic $f\in K\left[ x \right]$, then so are $\sigma_j\left( \alpha \right)$'s, since the minimal polynomial for $\alpha$ divides $f$. Hence $\tr_{K /\Q}\left( \alpha \right)\in\mO_K$. But the trace is always a rational number, so that
    \begin{equation*}
        \tr_{K /\Q}\left( \alpha \right)\in\Z.
    \end{equation*}
    In a similar manner,
    \begin{equation*}
        N_{K /\Q}\left( \alpha \right)\in\Z.
    \end{equation*}
    
    \begin{example}{}
        Consider $K=\Q\left( \sqrt{d} \right)$, where $d\in\N$ is squarefree and $d\neq 1$. Let
        \begin{equation*}
            \alpha = a+b\sqrt{d}
        \end{equation*}
        for some $a,b\in\Z$ with $b\neq 0$. Then
        \begin{equation*}
            \tr_{K /\Q}\left( \alpha \right) = \left( a+b\sqrt{d} \right) + \left( a-b\sqrt{d} \right) = 2a
        \end{equation*}
        and
        \begin{equation*}
            N_{K /\Q}\left( \alpha \right) = \left( a+b\sqrt{d} \right)\left( a-b\sqrt{d} \right) = a^{2}-db^{2}.
        \end{equation*}
    \end{example}

    \rruleline

    \np Recall that $a^{2}-db^{2}$ is frequently used in (elementary) ring theory! That is
    \begin{equation*}
        \text{$a+b\sqrt{d}$ is a unit in $\Q\left( \sqrt{d} \right)$} \iff a^{2}-db^{2} = 1 \text{ or } a^{2}-db^{2} = -1.
    \end{equation*}
    We have the following generalization, left as an exercise.

    \begin{exercise}{}
        Consider a number field $K$ and let $R = \mO_K$. Prove that for $\alpha\in R$,
        \begin{equation*}
            \alpha\in R^{\times}\iff N_{K /\Q}\left( \alpha \right) = 1 \text{ or } N_{K /\Q}\left( \alpha \right) = -1.
        \end{equation*}
    \end{exercise}

    \rruleline
    
    \np This concludes every properties of trace and norm for the course. As a first application, we are going to prove that every $\mO_K$ is a free $\Z$-module.

    \clearpage

    Here we prove a very powerful theorem with a cascade of useful corollaries. Fix
    \begin{equation*}
        K\text{ a number field with }\left[ K:\Q \right]=n.
    \end{equation*}

    \begin{theorem}{}
        $\left( \mO_K,+ \right)\iso\Z^n$.
    \end{theorem}

    \begin{proof}
        Let $\left\lbrace x_1,\ldots,x_n \right\rbrace$ be a $\Q$-basis for $K$. By Assignment 1, we may assume each $x_j\in\mO_K$. Let
        \begin{equation*}
            \begin{aligned}
                \phi:K&\to\Q^n \\
                x & \mapsto \left( \tr\left( xx_1 \right), \ldots, \tr\left( xx_n \right) \right),
            \end{aligned} 
        \end{equation*}
        where $\tr$ is the shorthand for $\tr_{K /\Q}$.

        Since $\tr$ is $\Q$-linear, so that $\phi$ is $\Q$-linear. Moreover, if for $x\in K$,
        \begin{equation*}
            \phi\left( x \right) = 0,
        \end{equation*}
        then
        \begin{equation*}
            \tr\left( xx_j \right) = 0, \hspace{2cm}\forall j\in\left\lbrace 1,\ldots,n \right\rbrace.
        \end{equation*}
        But $\left\lbrace x_1,\ldots,x_n \right\rbrace$ is a $\Q$-basis for $K$, so that
        \begin{equation}
            \tr\left( xy \right) = 0, \hspace{2cm}\forall y\in K.
        \end{equation}
        For contradiction, suppose $x\neq 0$. Since $x\in K$ is nonzero and $K$ is a field, we have $x^{-1}\in K$. But
        \begin{equation*}
            \tr\left( xx^{-1} \right) = \tr\left( 1 \right) = \tr\left( I_{n\times n} \right) = n \neq 0.
        \end{equation*}
        This contradicts [1.3], so we conclude $x=0$. Hence $\phi$ has trivial kernel, which means $\phi$ is a monomorphism of $\Q$-vector spaces.

        Since we know that $\phi\left( \alpha \right)\in\Z$ for $\alpha\in\mO_K$, it follows that
        \begin{equation*}
            \mO_K \overset{\phi}{\iso} \phi\left( \mO_K \right)\subseteq\Z^n.
        \end{equation*}
        That is, $\mO_K$ isomorphic to a $\Z$-submodule of $\Z^n$.

        By Fact 1.7, it follows that $\mO_K$ is a free $\Z$-module with $\rank\left( \mO_K \right) \leq n$, since $\Z$ is a PID. But we have a $\Q$-linearly independent, hence $\Z$-linearly independent, set $\left\lbrace x_1,\ldots,x_n \right\rbrace$ contained in $\mO_K$, so that $\rank\left( \mO_K \right)\geq n$. Thus we conclude
        \begin{equation*}
            \rank\left( \mO_K \right) = n
        \end{equation*}
        by Fact 1.8.
    \end{proof}

    \begin{example}{Warning Example}
        Consider $\left\lbrace 1,\sqrt{5} \right\rbrace\subseteq\Q\left( \sqrt{5} \right)$, which is a $\Q$-basis for $\Q\left( \sqrt{5} \right)$. However, it is not an \textit{integral basis} for $\Q\left( \sqrt{5} \right)$ over $\Q$.

        Theorem 1.10 only shows that \textit{integral basis exists}, but it hasn't constructed one!
    \end{example}

    \rruleline

    \begin{cor}{}
        If $I$ is a nonzero ideal of $\mO_K$, then $\left( I,+ \right)\iso\Z^n$.
    \end{cor}	

    \begin{proof}
        Let $\left\lbrace x_1,\ldots,x_n \right\rbrace$ be an integral basis for $\mO_K$ and let $a\in I$ be nonzero. Then $\left\lbrace ax_1,\ldots,ax_n \right\rbrace$ is a $\Z$-linearly independent subset of $I$, so that $n\leq\rank\left( I \right)$.
    \end{proof}

    \begin{cor}{}
        If $I$ is a nonzero ideal of $\mO_K$, then $\mO_K /I$ is finite.
    \end{cor}	

    \rruleline

    \np To prove Corollary 1.10.2, here is the last fact we steal from commutative algebra.

    \begin{fact}{}
        If $M$ is a finitely generated $\Z$-module, then $M\iso\Z^n\times T$, where is $T$ is a finite $\Z$-module.
    \end{fact}

    \np Fact 1.11 is a consequence of the unfamous \textit{structure theorem for finitely generated modules over a PID}.

    \begin{boxyproof}{Proof of Corollary 1.10.2}
        By Fact 1.11, we know
        \begin{equation*}
            \mO_K /I\iso Z^k\times T
        \end{equation*}
        as $Z$-modules, where $T$ is finite. We are going to show that $k=0$. To do so, observe that for $k\geq 1$, there is an element of infinite order in $\Z^k$. Hence it suffices to show that there is no element of infinite order in $\mO_K /I$.

        Suppose
        \begin{equation*}
            \overline{x} = x+I \in \mO_K /I
        \end{equation*}
        is an element of infinite order for contradiction. Let $\left\lbrace x_1,\ldots,x_n \right\rbrace$ be an integral basis for $I$. We note that, since $x_1,\ldots,x_n\in I$ but $x+I$ has infinite order, so that $x\notin I$. 

        \begin{claim}
            \textit{$\left\lbrace x,x_1,\ldots,x_n \right\rbrace$ is linearly independent.}

            Suppose
            \begin{equation*}
                cx + \sum^{n}_{j=1} c_jx_j = 0
            \end{equation*}
            for some $c,c_1,\ldots,c_n\in\Z$. Then
            \begin{equation*}
                c\overline{x} = 0 + I.
            \end{equation*}
            But $\overline{x}$ has an infinite order, so that $c=0$. But $x_1,\ldots,x_n$ are linearly independent, so that $c_1,\ldots,c_n=0$ as well.
        \end{claim}

        Note that the conclusing of Claim 1 contradicts the fact that $I\iso\Z^n$. Thus we conclude that
        \begin{equation*}
            \mO_K /I\iso T.
        \end{equation*}
    \end{boxyproof}
    
    \setcounter{stcounter}{10}
    \setcounter{corcounter}{2}
    \begin{cor}{}
        Every nonzero prime ideal of $\mO_K$ is maximal.
    \end{cor}	

    \begin{proof}
        Since $P$ is a prime ideal, $\mO_K /P$ is an integral domain. By Corollary 1.10.2, $\mO_K /P$ is a finite integral domain, so it is a field. Hence $P$ is maximal.
    \end{proof}

    \begin{cor}{}
        $\mO_K$ is Noetherian.
    \end{cor}	

    \begin{proof}
        Let $I$ be an ideal of $\mO_K$. Then $I$ is a free $\Z$-module with finite rank $n$, which means $I$ is a finitely generated $\Z$-module. Since $\Z$ is a submodule of $\mO_K$, $I$ is also a finitely generated $\mO_K$.
    \end{proof}

    
    
    
    
    
    
    
    
    
    
    
    
    
    
    
    
    
    
    
    
    
    
    
    
    
    
    
    
    
    
    
    
    
    
    
    
    
    

\end{document}
