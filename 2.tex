\documentclass[pmath441]{subfiles}

%% ========================================================
%% document

\begin{document}

    \section{Discriminant}

    Suppose we have a number field $K$ with $\left[ K:\Q \right]=n$ and let $R=\mO_K$. Given $\left\lbrace v_1,\ldots,v_n \right\rbrace\subseteq R$, we desire to find a way to \textit{discriminate} whether or not $\left\lbrace v_1,\ldots,v_n \right\rbrace$ is an integral basis for $R$.

    Fix $K,R$ throughout.
    
    \subsection{Elementary Properties of Discriminant}

    \np We first record the definition of discriminant and than investigate many importnat properties of it.

    \begin{definition}{\textbf{Discriminant} of Finite Subset of $K$}
        Let $\sigma_1,\ldots,\sigma_n$ be embeddings of $K$ in $\CC$. The \emph{discriminant} of $\left\lbrace a_1,\ldots,a_n \right\rbrace\subseteq K$, denoted as $\disc\left( a_1,\ldots,a_n \right)$, is
        \begin{equation*}
            \disc\left( a_1,\ldots,a_n \right) = \det \left(\left[ \sigma_i\left( a_j \right) \right]^n_{i,j=1}\right)^{2}.
        \end{equation*}
    \end{definition}

    \np Because of the presenece of the power $2$, Def'n 2.1 is \textit{independnet} of choice of ordering of the $\sigma_i$'s and $a_j$'s.

    \np Consider
    \begin{equation*}
        B = \left[ \sigma_i\left( a_j \right) \right]^n_{i,j}
    \end{equation*}
    and let $A = B^{T}$. Since determinant is multiplicative and is invariant under transpose, it follows
    \begin{equation*}
        \det\left( a_1,\ldots,a_n \right) = \det\left( AB \right).
    \end{equation*}
    However, the $\left( i,j \right)$th entry of $AB$ is
    \begin{equation*}
        \begin{bmatrix} \sigma_1\left( a_i \right) & \cdots & \sigma_n\left( a_i \right) \end{bmatrix} \begin{bmatrix} \sigma_1\left( a_j \right)\\\vdots\\\sigma_n\left( a_j \right) \end{bmatrix} 
        = \sum^{n}_{k=1} \sigma_k\left( a_i \right)\sigma_k\left( a_j \right)
        = \sum^{n}_{k=1} \sigma_k\left( a_ia_j \right)
        = \tr_{K /\Q}\left( a_ia_j \right).
    \end{equation*}
    Therefore,
    \begin{equation*}
        \disc\left( a_1,\ldots,a_n \right) = \det \left[ \tr_{K /\Q}\left( a_ia_j \right) \right]^n_{i,j=1}.
    \end{equation*}
    Some texts use the above formula as the definition.

    Since we know that $\tr_{K /\Q}\left( a \right)$ is a rational number for $a\in K$,
    \begin{equation*}
        \disc\left( a_1,\ldots,a_n \right) \in \Q.
    \end{equation*}
    In particular, when $a_1,\ldots,a_n\in\mO_K$,
    \begin{equation*}
        \disc\left( a_1,\ldots,a_n \right)\in\Z.
    \end{equation*}

    \np Consider $v,w\in K^n$ and $A\in\Q^{n\times n}$ such that
    \begin{equation*}
        Av = w.
    \end{equation*}
    Then, for $i\in\left\lbrace 1,\ldots,n \right\rbrace$,
    \begin{equation*}
        A\sigma_i\left( v \right) =
        \begin{bmatrix}
            A_{1,1} & \cdots & A_{1,n} \\
        	\vdots & \ddots & \vdots \\
                A_{n,1} & \cdots & A_{n,n} \\
        \end{bmatrix}
        \begin{bmatrix} \sigma_i\left( v_1 \right) \\ \vdots \\ \sigma_i\left( v_n \right) \end{bmatrix} =
        \begin{bmatrix} \sigma_i\left( \sum^{n}_{j=1} A_{1,j}v_j \right) \\ \vdots \\ \sigma_i\left( \sum^{n}_{j=1} A_{n,j}v_j \right) \end{bmatrix} =
        \begin{bmatrix} \sigma_i\left( w_1 \right) \\ \vdots \\ \sigma_i\left( w_n \right) \end{bmatrix}.
    \end{equation*}
    Therefore,
    \begin{equation*}
        A \left[ \sigma_i\left( v_j \right) \right]^n_{i,j=1} = \left[ \sigma_i\left( w_j \right) \right]^n_{i,j=1}.
    \end{equation*}
    Thus we conclude
    \begin{equation*}
        \det\left( A^{2} \right)\disc\left( v \right) = \disc\left( w \right).
    \end{equation*}
    
    \np Let $\left\lbrace v_1,\ldots,v_n \right\rbrace\subseteq\mO_K$ be an integral basis for $\mO_K$ and let $\left\lbrace w_1,\ldots,w_n \right\rbrace\subseteq\mO_K$. Then there is $\left\lbrace C_{i,j} \right\rbrace^n_{i,j}\subseteq\Z$ such that
    \begin{equation*}
        w_i = \sum^{n}_{j=1}C_{i,j}v_j, \hspace{2cm}\forall i\in\left\lbrace 1,\ldots,n \right\rbrace.
    \end{equation*}
    That is,
    \begin{equation*}
        w = Cv,
    \end{equation*}
    where $C = \left[ C_{i,j} \right]^n_{i,j=1}$. Hence
    \begin{equation*}
        \disc\left( w \right) = \det\left( C^{2} \right)\disc\left( v \right).
    \end{equation*}
    Let $\beta = \left\lbrace v_1,\ldots,v_n \right\rbrace$ and
    \begin{equation*}
        \begin{aligned}
            T:\mO_K&\to\mO_K \\
            v_i&\mapsto w_i, \hspace{2cm}\forall i\in\left\lbrace 1,\ldots,n \right\rbrace
        \end{aligned} ,
    \end{equation*}
    which is a $\Z$-linear homomorphism. Then
    \begin{equation*}
        \left[ T \right]_{\beta} = \begin{bmatrix} \left[ T\left( v_1 \right) \right]_{\beta} & \cdots & \left[ T\left( v_n \right) \right]_{\beta} \end{bmatrix} = \begin{bmatrix} \left[ w_1 \right]_{\beta} & \cdots & \left[ w_n \right]_{\beta} \end{bmatrix} = C^{T}.
    \end{equation*}
    
    
    
    
    
    
    
    
    
    
    
    
    
    
    
    
    
    
    
    
    
    
    
    
    
    
    
    
    
    
    
    
    
    
    
    
    
    

\end{document}
