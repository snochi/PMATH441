\documentclass[pmath441]{subfiles}

%% ========================================================
%% document

\begin{document}

    \section{Discriminant}

    Suppose we have a number field $K$ with $\left[ K:\Q \right]=n$ and let $R=\mO_K$. Given $\left\lbrace v_1,\ldots,v_n \right\rbrace\subseteq R$, we desire to find a way to \textit{discriminate} whether or not $\left\lbrace v_1,\ldots,v_n \right\rbrace$ is an integral basis for $R$.

    Fix $K,R$ throughout.
    
    \subsection{Elementary Properties of Discriminant}

    \np We first record the definition of discriminant and than investigate many importnat properties of it.

    \begin{definition}{\textbf{Discriminant} of Finite Subset of $K$}
        Let $\sigma_1,\ldots,\sigma_n$ be embeddings of $K$ in $\CC$. The \emph{discriminant} of $\left\lbrace a_1,\ldots,a_n \right\rbrace\subseteq K$, denoted as $\disc\left( a_1,\ldots,a_n \right)$, is
        \begin{equation*}
            \disc\left( a_1,\ldots,a_n \right) = \det \left(\left[ \sigma_i\left( a_j \right) \right]^n_{i,j=1}\right)^{2}.
        \end{equation*}
    \end{definition}

    \np Because of the presenece of the power $2$, Def'n 2.1 is \textit{independnet} of choice of ordering of the $\sigma_i$'s and $a_j$'s.

    \np Consider
    \begin{equation*}
        B = \left[ \sigma_i\left( a_j \right) \right]^n_{i,j}
    \end{equation*}
    and let $A = B^{T}$. Since determinant is multiplicative and is invariant under transpose, it follows
    \begin{equation*}
        \det\left( a_1,\ldots,a_n \right) = \det\left( AB \right).
    \end{equation*}
    However, the $\left( i,j \right)$th entry of $AB$ is
    \begin{equation*}
        \begin{bmatrix} \sigma_1\left( a_i \right) & \cdots & \sigma_n\left( a_i \right) \end{bmatrix} \begin{bmatrix} \sigma_1\left( a_j \right)\\\vdots\\\sigma_n\left( a_j \right) \end{bmatrix} 
        = \sum^{n}_{k=1} \sigma_k\left( a_i \right)\sigma_k\left( a_j \right)
        = \sum^{n}_{k=1} \sigma_k\left( a_ia_j \right)
        = \tr_{K /\Q}\left( a_ia_j \right).
    \end{equation*}
    Therefore,
    \begin{equation*}
        \disc\left( a_1,\ldots,a_n \right) = \det \left[ \tr_{K /\Q}\left( a_ia_j \right) \right]^n_{i,j=1}.
    \end{equation*}
    Some texts use the above formula as the definition.

    Since we know that $\tr_{K /\Q}\left( a \right)$ is a rational number for $a\in K$,
    \begin{equation*}
        \disc\left( a_1,\ldots,a_n \right) \in \Q.
    \end{equation*}
    In particular, when $a_1,\ldots,a_n\in\mO_K$,
    \begin{equation*}
        \disc\left( a_1,\ldots,a_n \right)\in\Z.
    \end{equation*}

    \np Consider $v,w\in K^n$ and $A\in\Q^{n\times n}$ such that
    \begin{equation*}
        Av = w.
    \end{equation*}
    Then, for $i\in\left\lbrace 1,\ldots,n \right\rbrace$,
    \begin{equation*}
        A\sigma_i\left( v \right) =
        \begin{bmatrix}
            A_{1,1} & \cdots & A_{1,n} \\
        	\vdots & \ddots & \vdots \\
                A_{n,1} & \cdots & A_{n,n} \\
        \end{bmatrix}
        \begin{bmatrix} \sigma_i\left( v_1 \right) \\ \vdots \\ \sigma_i\left( v_n \right) \end{bmatrix} =
        \begin{bmatrix} \sigma_i\left( \sum^{n}_{j=1} A_{1,j}v_j \right) \\ \vdots \\ \sigma_i\left( \sum^{n}_{j=1} A_{n,j}v_j \right) \end{bmatrix} =
        \begin{bmatrix} \sigma_i\left( w_1 \right) \\ \vdots \\ \sigma_i\left( w_n \right) \end{bmatrix}.
    \end{equation*}
    Therefore,
    \begin{equation*}
        A \left[ \sigma_i\left( v_j \right) \right]^n_{i,j=1} = \left[ \sigma_i\left( w_j \right) \right]^n_{i,j=1}.
    \end{equation*}
    Thus we conclude
    \begin{equation*}
        \det\left( A^{2} \right)\disc\left( v \right) = \disc\left( w \right).
    \end{equation*}
    
    \np Let $\left\lbrace v_1,\ldots,v_n \right\rbrace\subseteq\mO_K$ be an integral basis for $\mO_K$ and let $\left\lbrace w_1,\ldots,w_n \right\rbrace\subseteq\mO_K$. Then there is $\left\lbrace C_{i,j} \right\rbrace^n_{i,j}\subseteq\Z$ such that
    \begin{equation*}
        w_i = \sum^{n}_{j=1}C_{i,j}v_j, \hspace{2cm}\forall i\in\left\lbrace 1,\ldots,n \right\rbrace.
    \end{equation*}
    That is,
    \begin{equation*}
        w = Cv,
    \end{equation*}
    where $C = \left[ C_{i,j} \right]^n_{i,j=1}$. Hence
    \begin{equation*}
        \disc\left( w \right) = \det\left( C^{2} \right)\disc\left( v \right).
    \end{equation*}
    Let $\beta = \left\lbrace v_1,\ldots,v_n \right\rbrace$ and
    \begin{equation*}
        \begin{aligned}
            T:\mO_K&\to\mO_K \\
            v_i&\mapsto w_i, \hspace{2cm}\forall i\in\left\lbrace 1,\ldots,n \right\rbrace
        \end{aligned} ,
    \end{equation*}
    which is a $\Z$-linear homomorphism. Then
    \begin{equation*}
        \left[ T \right]_{\beta} = \begin{bmatrix} \left[ T\left( v_1 \right) \right]_{\beta} & \cdots & \left[ T\left( v_n \right) \right]_{\beta} \end{bmatrix} = \begin{bmatrix} \left[ w_1 \right]_{\beta} & \cdots & \left[ w_n \right]_{\beta} \end{bmatrix} = C^{T}.
    \end{equation*}
    
    \np Let $A\in\Z^{n\times n}$. If $\det\left( A \right)\neq 0$, then recall that
    \begin{equation*}
        A^{-1} = \frac{1}{\det\left( A \right)}\adj\left( A \right).
    \end{equation*}
    Since $A\in\Z^{n\times n}$, every cofactor of $A$ is in $\Z$, so that $\adj\left( A \right)\in\Z^{n\times n}$. Thus,
    \begin{equation*}
        A^{-1}\in\Z^{n\times n}\iff\det\left( A \right) = 1 \text{ or }\det\left( A \right) = -1.
    \end{equation*}

    \np Let $\left\lbrace v_1,\ldots,v_n \right\rbrace\mO_K$ be an integral basis and suppose 
    \begin{equation*}
        \disc\left( v \right) = \disc\left( w \right)
    \end{equation*}
    for some $\left\lbrace w_1,\ldots,w_n \right\rbrace\in\mO_K$. Then
    \begin{equation*}
        Cv = w
    \end{equation*}
    for some $C\in\Z^{n\times n}$. This implies that
    \begin{equation*}
        \det\left( C^{2} \right)\disc\left( v \right)=\disc\left( w \right),
    \end{equation*}
    so that
    \begin{equation*}
        \left( \det\left( C \right) \right)^{2} = 1.\footnote{Note the degenerate case where $\disc\left( v \right)=\disc\left( w \right)=0$. We will show that this never happens.}
    \end{equation*}
    Hence $\det\left( C \right) = 1$ or $\det\left( C \right) = -1$, which means $C$ is invertible with $C^{-1}\in\Z^{n\times n}$. This implies that $C^{T}$ is invertible with integer inverse, so that
    \begin{equation*}
        \begin{aligned}
            T:\mO_K&\to\mO_K
        \end{aligned} 
    \end{equation*}

    Therefore, given an integral basis $\left\lbrace v_1,\ldots,v_n \right\rbrace$, we can search for other integral basis by looking at subsets $\left\lbrace w_1,\ldots,w_n \right\rbrace$ whose discriminant agrees with $\disc\left( v \right)$.

    Conversely, if
    \begin{equation*}
        \left\lbrace v_1,\ldots,v_n \right\rbrace,\left\lbrace w_1,\ldots,w_n \right\rbrace\subseteq\mO_K
    \end{equation*}
    are integral bases, then $Av=w, Bw=v$ for some $A,B\in\Z^{n\times n}$. It follows that $\det\left( A \right)^{2}\disc\left( v \right)=\disc\left( w \right)$ and $\det\left( B \right)^{2}\disc\left( w \right)=\disc\left( v \right)$. Thus we have that
    \begin{equation*}
        \disc\left( v \right) = \disc\left( w \right).
    \end{equation*}

    \np Let $\left\lbrace a_1,\ldots,a_n \right\rbrace\subseteq K$. Suppose there is nonzero $\left( c_1,\ldots,c_n \right)\in\Q^n$ such that
    \begin{equation*}
        \sum^{n}_{j=1} c_ja_j = 0.
    \end{equation*}
    This means
    \begin{equation*}
        \sum^{n}_{j=1} c_j\sigma_i\left( a_j \right) = 0
    \end{equation*}
    for any embedding $\sigma_i$ of $K$ in $\CC$, so that $\left[ \sigma_i\left( a_j \right) \right]^{n}_{i,j}$ is not invertible. It follows that
    \begin{equation*}
        \disc\left( a_1,\ldots,a_n \right) = 0.
    \end{equation*}
    Conversely, suppose that $\disc\left( a_1,\ldots,a_n \right) = 0$. Then the columns of $\left[ \sigma_{i}\left( a_j \right) \right]^{n}_{i,j=1}$ are linearly dependent. That is,
    \begin{equation*}
        \sum^{n}_{j=1}c_j\sigma_i\left( a_j \right) = 0,\hspace{2cm}\forall i,
    \end{equation*}
    for some nonzero $\left( c_1,\ldots,c_n \right)\in\Q^n$. By considering $\sigma_i = \iota : K\to\CC$ by $k\mapsto k$, we observe that $\sum^{n}_{j=1} a_j = 0$. Thus $\left\lbrace a_1,\ldots,a_n \right\rbrace$ is $\Q$-linearly dependent.
    
    \subsection{Discriminant of Number Fields}

    Fix a number field $K$ with $\left[ K:\Q \right] = n$.

    \begin{definition}{\textbf{Discriminant} of a Number Field}
        We define the \emph{discriminant} of $K$, $\disc\left( K \right)$, as
        \begin{equation*}
            \disc\left( K \right) = \disc\left( v_1,\ldots,v_n \right),
        \end{equation*}
        where $v_1,\ldots,v_n$ is an integral basis for $\mO_K$.
    \end{definition}
    
    \begin{example}{}
        Consider $K = \Q\left( \sqrt{d} \right)$, where $d\neq 1$ is squarefree.
        
        \begin{case}
            \textit{$d\equiv 1\mod 4$.}

            We claim that $\left\lbrace 1,\frac{1+\sqrt{d}}{2} \right\rbrace$ is an integral basis (check this; exercise!). Then
            \begin{equation*}
                \disc\left( K \right) = \det
                \begin{bmatrix}
                	1 & \frac{1+\sqrt{d}}{2} \\
                	1 & \frac{1-\sqrt{d}}{2} \\
                \end{bmatrix}^{2} =
                \left( \frac{1-\sqrt{d}}{2}-\frac{1+\sqrt{d}}{2} \right)^{2} = \left( -\sqrt{d} \right)^{2} = d.
            \end{equation*}
        \end{case}

        \begin{case}
            \textit{$d\equiv 2,3\mod 4$.}

            In this case, $\left\lbrace 1,\sqrt{d} \right\rbrace$ is an integral basis, so that
            \begin{equation*}
                \disc\left( K \right) = \det
                \begin{bmatrix}
                	1 & \sqrt{d} \\
                	1 & -\sqrt{d} \\
                \end{bmatrix}^{2} = 4d.
            \end{equation*}
        \end{case}
    \end{example}
    
    \rruleline
    
    \subsection{Computational Considerations}

    \begin{recall}{\textbf{Discriminant} of a Polynomial}
        Let $p\in\CC\left[ x \right]$ and let $\alpha_1,\ldots,\alpha_n\in\CC$ be the roots of $p$. Then we define the \textit{discriminant} of $p$, $\disc\left( p \right)$, by
        \begin{equation*}
            \disc\left( p \right) = \prod^{}_{i<j} \left( \alpha_i-\alpha_j \right)^{2}.
        \end{equation*}
    \end{recall}

    \begin{example}{Discriminant of Quadratic, Cubic Polynomials}
        For a quadratic $x^{2}+bx+c$,
        \begin{equation*}
            \disc\left( x^{2}+bx+c \right) = b^{2}-4c.
        \end{equation*}
        For a \textit{depressed} cubic $x^{3}+bx+c$,
        \begin{equation*}
            \disc\left( x^{3}+bx+c \right) = -4b^{3}-27c^{2}.
        \end{equation*}
        To turn a general cubic $x^{3}+ax^{2}+bx+c$ into a depressed cubic, substitute $x$ by $x-\frac{a}{3}$ which \textit{eliminates} $x^{2}$ term. Since every root is \textit{shifted by the same amout $\frac{a}{3}$}, it follows that the discriminant is the same:
        \begin{equation*}
            \disc\left( x^{3}+ax^{2}+bx+c \right) = -4b^{3}-27c^{2}.
        \end{equation*}
    \end{example}

    \rruleline
    
    \begin{definition}{\textbf{Discriminant} of an Algebraic Number}
        Suppose $\alpha\in\CC$ is such that $\left[ \Q\left( \alpha \right):\Q \right] = n$. Then we define the \emph{discriminant} of $\alpha$, $\disc\left( \alpha \right)$, to be
        \begin{equation*}
            \disc\left( \alpha \right) = \disc\left( 1,\alpha,\ldots,\alpha^{n-1} \right).
        \end{equation*}
    \end{definition}

    \np Observe that $\left\lbrace 1,\alpha,\ldots,\alpha^{n-1} \right\rbrace$ is an integral basis for $\Z\left[ \alpha \right]$. Moreover,
    \begin{equation*}
        \disc\left( \alpha \right) = \det
        \begin{bmatrix}
            1 & \alpha_1 & \alpha_1^{2} & \cdots & \alpha_1^{n-1} \\
            1 & \alpha_2 & \alpha_2^{2} & \cdots & \alpha_2^{n-1} \\
            1 & \alpha_3 & \alpha_3^{2} & \cdots & \alpha_3^{n-1} \\
            \vdots & \vdots & \vdots & \cdots & \vdots \\
            1 & \alpha_n & \alpha_n^{2} & \cdots & \alpha_n^{n-1} \\
        \end{bmatrix}^{2}.
    \end{equation*}
    Observe that we have a Vandermonde matrix, whose determinant is famously 
    \begin{equation*}
        \det
        \begin{bmatrix}
            1 & \alpha_1 & \alpha_1^{2} & \cdots & \alpha_1^{n-1} \\
            1 & \alpha_2 & \alpha_2^{2} & \cdots & \alpha_2^{n-1} \\
            1 & \alpha_3 & \alpha_3^{2} & \cdots & \alpha_3^{n-1} \\
            \vdots & \vdots & \vdots & \cdots & \vdots \\
            1 & \alpha_n & \alpha_n^{2} & \cdots & \alpha_n^{n-1} \\
        \end{bmatrix} = \prod^{}_{i<j} \left( \alpha_i-\alpha_j \right)
    \end{equation*}
    Since we have the square term, it follows that
    \begin{equation*}
        \disc\left( \alpha \right) = \prod^{}_{i<j}\left( \alpha_i-\alpha_j \right)^{2} = \disc\left( p \right),
    \end{equation*}
    where $p$ is the minimal polynomial of $\alpha$. Thus the discriminant of an algebraic number and its minimal polynomial coincides.
    
    \np Suppose $\left\lbrace v_1,\ldots,v_n \right\rbrace$ is an integral basis for $\mO_{\Q\left( \alpha \right)}$. Then
    \begin{equation*}
        \begin{bmatrix} 1\\\cdots\\\alpha^{n-1} \end{bmatrix} = A \begin{bmatrix} v_1\\\cdots\\v_n \end{bmatrix}
    \end{equation*}
    for some invertible $A\in\Z^{n\times n}$. Therefore,
    \begin{equation*}
        \disc\left( \alpha \right) = \det\left( A \right)^{2}\disc\left( \Q\left( \alpha \right) \right) = \left[ \mO_{\Q\left( \alpha \right)}:\Z\left[ \alpha \right] \right]^{2}\disc\left( \Q\left( \alpha \right) \right)
    \end{equation*}
    by Assignment 2.

    \np As a corollary, if $\disc\left( \alpha \right)$ is squarefree, then
    \begin{equation*}
        \mO_{\Q\left( \alpha \right)} = \Z\left[ \alpha \right].
    \end{equation*}

    \begin{example}{}
        Suppose $\alpha\in\CC$ is such that $p\left( \alpha \right) = 0$, where
        \begin{equation*}
            p = x^{3}+x+1.
        \end{equation*}
        Note that $p$ is irreducible over $\Q$, so it is the minimal polynomial for $\alpha$. Then $\disc\left( \alpha \right) = \disc\left( p \right) = -4-27 = -31$, which is prime so is squarefree.

        Thus
        \begin{equation*}
            \mO_{\Q\left( \alpha \right)} = \Z\left[ \alpha \right] = \left\lbrace a+b\alpha+c\alpha^{2} \right\rbrace.
        \end{equation*}
    \end{example}

    \rruleline
    
    \np Let $\alpha$ be an algebraic number with minimal polynomial $p\in\Q\left[ x \right]$ and $\left[ \Q\left( \alpha \right):\Q \right]=n$. Let $\alpha_1=\alpha$ and let $\alpha_2,\ldots,\alpha_n$ be the conjugates of $\alpha$. Then
    \begin{equation*}
        p = \left( x-\alpha_1 \right)\cdots\left( x-\alpha_n \right).
    \end{equation*}
    Consider the \textit{formal derivative} of $p$, which we can find using the product rule:
    \begin{equation*}
        p' = \sum^{n}_{i=1} \prod^{n}_{j=1,j\neq i} \left( x-\alpha_j \right) .
    \end{equation*}
    Then
    \begin{equation*}
        p'\left( \alpha_i \right) = \prod^{n}_{j=1,j\neq i} \left( \alpha_i-\alpha_j \right) , \hspace{2cm}\forall i.
    \end{equation*}
    Now, given the embeddings $\sigma_1,\ldots,\sigma_n:\Q\left( \alpha \right)\to\CC$,
    \begin{equation*}
        \begin{aligned}
            N_{K /\Q}\left( p'\left( \alpha \right) \right) & = \prod^{n}_{i=1} \sigma_r\left( p'\left( \alpha \right) \right) = \prod^{n}_{i=1} p'\left( \sigma_i\left( \alpha \right) \right) && \text{since $\sigma_i$ fix each element in $\Q$} \\
                                                            & = \prod^{n}_{i=1} p'\left( \alpha_i \right) = \prod^{n}_{i\neq j} \left( \alpha_i-\alpha_j \right) = \left( -1 \right)^{\binom{n}{2}} \prod^{n}_{i<j} \left( \alpha_i-\alpha_j \right)^{2} \\
                                                            & = \left( -1 \right)^{\binom{n}{2}} \disc\left( p \right) = \left( -1 \right)^{\binom{n}{2}} \disc\left( \alpha \right) .
        \end{aligned} 
    \end{equation*}

    \begin{definition}{\textbf{Resultant} of Polynomials}
        Let $f=\sum^{n}_{i=0}a_ix^i, g=\sum^{m}_{j=0}b_jx^j\in\CC\left[ x \right]$. Then we define the \emph{resultant} of $f,g$, denoted as $\res\left( f,g \right)$, is the determinant of
        \begin{equation*}
            \begin{bmatrix} 
                a & 0 & \cdots & \cdots & \cdots & \cdots & 0\\
                0 & a & 0 & \cdots & \cdots & \cdots & 0 \\
                  & \ddots & \ddots & \ddots & \ddots & \ddots & 0 \\
                0 & \cdots & 0 & a & 0 & \cdots & 0 \\
                b & 0 & \cdots & \cdots & \cdots & \cdots & 0  \\
                0 & b & 0 & \cdots & \cdots & \cdots & 0 \\
                  & \ddots & \ddots & \ddots & \ddots & \ddots & 0 \\
                0 & \cdots & 0 & b & 0 & \cdots & 0 \\
            \end{bmatrix} \in\Q^{\left( n+m \right)\times\left( n+m \right)} ,
        \end{equation*}
        where $a = \left( a_n,\ldots,a_0 \right), b = \left( b_m,\ldots,b_0 \right)$.
    \end{definition}

    \begin{example}{}
        We have
        \begin{equation*}
            \res\left( x^3+x+2,x^{2}+4x-1 \right) =
            \det
            \begin{bmatrix}
            	1 & 0 & 1 & 2 & 0 \\
            	0 & 1 & 0 & 1 & 2 \\
            	1 & 4 & -1 & 0 & 0 \\
            	0 & 1 & 4 & -1 & 0 \\
            	0 & 0 & 1 & 4 & -1 \\
            \end{bmatrix}.
        \end{equation*}
    \end{example}

    \rruleline
    
    \begin{fact}{}
        Let $\alpha\in\CC$ be an algebraic number with the minimal polynomial $p\in\Q\left[ x \right]$ such that $\alpha\in\mO_{\Q\left( \alpha \right)}$ and $\left[ \Q\left( \alpha \right):\Q \right]=n$. Then
        \begin{equation*}
            \disc\left( \alpha \right) = \left( -1 \right)^{\binom{n}{2}}\res\left( p,p' \right).
        \end{equation*}
    \end{fact}

    \begin{example}{}
        Let $\alpha\in\CC$ be such that $p\left( \alpha \right) = 0$, where
        \begin{equation*}
            p = x^3-x^2-1.
        \end{equation*}
        Since $p\left( 1 \right),p\left( -1 \right)\neq 0$, so $p$ is irreducible over $\Q$. Hence $\left[ \Q\left( \alpha \right):\Q \right] = 3$. 

        Note that
        \begin{equation*}
            p' = 3x^{2}-2x.
        \end{equation*}
        It follows that
        \begin{equation*}
            \disc\left( \alpha \right) = \left( -1 \right)^{\binom{3}{2}} \det
            \begin{bmatrix}
            	1 & -1 & 0 & -1 & 0 \\
            	0 & 1 & -1 & 0 & -1 \\
            	3 & -2 & 0 & 0 & 0 \\
            	0 & 3 & -2 & 0 & 0 \\
            	0 & 0 & 3 & -2 & 0 \\
            \end{bmatrix} = 31.
        \end{equation*}
        Since $31$ is squarefree, so that
        \begin{equation*}
            \mO_K = \Z\left[ \alpha \right].
        \end{equation*}
    \end{example}
    
    \rruleline
    
    
    
    
    
    
    
    
    
    
    
    
    
    
    
    
    
    
    
    
    
    
    
    
    
    
    
    
    
    
    
    
    
    
    
    

\end{document}
